%\documentclass{article}
\documentclass{book}

%\usepackage[cm]{fullpage}
\usepackage{fullpage}
%\usepackage{hyperref}
\usepackage[colorlinks=true,citecolor=brown,pagebackref=true,backref=true,hyperfigures=true,hyperfootnotes=true,hyperindex=true]{hyperref}

\parindent 0in
\parskip 0.1in

\setcounter{secnumdepth}{3}
\setcounter{tocdepth}{3}

\begin{document}
\title{64-bit ISA extensions to the AJIT processor}
\author{Madhav Desai}
\maketitle
\newpage
\parskip=0em
\tableofcontents
\newpage
\listoftables
\newpage
\parskip 0.1in

\chapter{The ISA Specification from IITB}
\label{chap:from:mpd:at:iitb}

\section{Overview}
\label{sec:Overview}

The AJIT processor implements the  Sparc-V8 ISA.  We propose to extend
this ISA to provide support for a native 64-bit integer datatype.  The
proposed  extensions use  the  existing instruction  encodings to  the
maximum extent possible.

All proposed extensions are:

\centerline{Register $\times$ Register $\rightarrow$ Register,Condition-codes}

type instructions.  The load/store instructions are not modified.

We list  the additional instructions  in the subsequent  sections.  In
each  case,  only the  differences  in  the  encoding relative  to  an
existing Sparc-V8 instruction are provided.

\newpage
\section{ISA Extensions}
\label{sec:isa:extns}

The extensions to SPArc V8 for AJIT are described in this section.

\subsection{Integer-Unit Extensions: Arithmetic-Logic Instructions}
\label{sec:integer-unit-extns:arith-logic-insns}

These  instructions provide  64-bit  arithmetic/logic  support in  the
integer unit.  The instructions work  on 64-bit register pairs in most
cases.  Register-pairs are  identified by a 5-bit  even number (lowest
bit     must     be     0).      See     Tables~\ref{tab:arith:insns},
\ref{tab:shift:insns},            \ref{tab:muldiv:insns}           and
\ref{tab:64bit:logical:insns}.

\begin{table}[p]
  \centering
  \begin{tabular}[p]{|l|l|}
    \hline
\multicolumn{2}{|l|}{	\textbf{ADDD}			} \\ 
\hline
 		  same as ADD, but with Instr[13]=0 (i=0), and Instr[5]=1. & 
 		rd(pair) $\leftarrow$ rs1(pair) + rs2(pair)\\
\hline
\hline
\multicolumn{2}{|l|}{	\textbf{ADDDCC}} \\ 
\hline
 		  same as ADDCC, but with Instr[13]=0 (i=0), and Instr[5]=1. & 
 		rd(pair) $\leftarrow$ rs1(pair) + rs2(pair), set Z,N\\
\hline
\hline
\multicolumn{2}{|l|}{	\textbf{SUBD}} \\ 
\hline
 		  same as SUB, but with Instr[13]=0 (i=0), and Instr[5]=1. & 
 		rd(pair) $\leftarrow$ rs1(pair) - rs2(pair)\\
\hline
\hline
\multicolumn{2}{|l|}{	\textbf{SUBDCC}} \\ 
\hline
 		  same as SUBCC, but with Instr[13]=0 (i=0), and Instr[5]=1. & 
 		rd(pair) $\leftarrow$ rs1(pair) - rs2(pair), set Z,N\\
\hline
  \end{tabular}
  \caption{Addition and Subtraction Instructions}
  \label{tab:arith:insns}
\end{table}

\begin{table}[p]
  \centering
  \begin{tabular}[p]{|p{.45\textwidth}|p{.45\textwidth}|}
    \hline
\multicolumn{2}{|l|}{	\textbf{SLLD}} \\ 
 \hline 
    same as SLL, but with Instr[6:5]=2.
    if imm bit (Instr[13]) is 1, then Instr[5:0] is the shift-amount.
    else shift-amount is the lowest 5 bits of rs2. Note that rs2
    is a 32-bit register. & 
    rd(pair) $\leftarrow$  rs1(pair) $<<$ shift-amount\\
\hline
\hline
\multicolumn{2}{|l|}{	\textbf{SRLD}} \\ 
 \hline 
    same as SRL, but with Instr[6:5]=2.
    if imm bit (Instr[13]) is 1, then Instr[5:0] is the shift-amount.
    else shift-amount is the lowest 5 bits of rs2. Note that rs2
    is a 32-bit register. & 
    rd(pair) $\leftarrow$  rs1(pair) $>>$ shift-amount\\
\hline
\hline
\multicolumn{2}{|l|}{	\textbf{SRAD}} \\ 
 \hline 
    same as SRA, but with Instr[6:5]=2.
    if imm bit (Instr[13]) is 1, then Instr[5:0] is the shift-amount.
    else shift-amount is the lowest 5 bits of rs2. Note that rs2
    is a 32-bit register. & 
    rd(pair) $\leftarrow$  rs1(pair) $>>$ shift-amount (with sign extension).\\
\hline
  \end{tabular}
  \caption{Shift instructions}
  \label{tab:shift:insns}
\end{table}

\begin{table}[p]
  \centering
  \begin{tabular}[p]{|p{.45\textwidth}|p{.45\textwidth}|}
    \hline
\multicolumn{2}{|l|}{\textbf{	UMULD}} \\ 
 \hline 
 		  same as UMUL, but with Instr[13]=0 (i=0), and Instr[5]=1. & 
 		rd(pair) $\leftarrow$ rs1(pair) * rs2(pair)\\
\hline
\hline
\multicolumn{2}{|l|}{\textbf{	UMULDCC}} \\ 
 \hline 
 		  same as UMULCC, but with Instr[13]=0 (i=0), and Instr[5]=1. & 
 		rd(pair) $\leftarrow$ rs1(pair) * rs2(pair), sets Z,\\
\hline
\hline
\multicolumn{2}{|l|}{\textbf{	SMULD}} \\ 
 \hline 
 		  same as SMULD, but with Instr[13]=0 (i=0), and Instr[5]=1. & 
 		rd(pair) $\leftarrow$ rs1(pair) * rs2(pair) (signed)\\
\hline
\hline
\multicolumn{2}{|l|}{\textbf{	SMULDCC}} \\ 
 \hline 
 		  same as SMULCC, but with Instr[13]=0 (i=0), and Instr[5]=1. & 
 		\parbox{\linewidth}{rd(pair) $\leftarrow$ rs1(pair) *
                  rs2(pair) (signed)\\	sets condition codes Z,N,Ovflow}\\ 
\hline
\hline
\multicolumn{2}{|l|}{\textbf{	UDIVD}} \\ 
 \hline 
 		  same as UDIV, but with Instr[13]=0 (i=0), and Instr[5]=1. & 
 		\parbox{\linewidth}{rd(pair) $\leftarrow$ rs1(pair) /
                 rs2(pair)\\
    \textbf{Note:} can generate div-by-zero trap.}\\
\hline
\hline
\multicolumn{2}{|l|}{\textbf{	UDIVDCC}} \\ 
 \hline 
 		  same as UDIVCC, but with Instr[13]=0 (i=0), and Instr[5]=1. & 
 		\parbox{\linewidth}{rd(pair) $\leftarrow$ rs1(pair) /
                 rs2(pair),\\ sets condition codes Z,Ovflow \\ \textbf{Note:} can generate div-by-zero trap.}\\
\hline
\hline
\multicolumn{2}{|l|}{\textbf{	SDIVD}} \\ 
 \hline 
 		  same as SDIV, but with Instr[13]=0 (i=0), and Instr[5]=1. & 
 		rd(pair) $\leftarrow$ rs1(pair) / rs2(pair) (signed)\\
\hline
\hline
\multicolumn{2}{|l|}{\textbf{	SDIVDCC}} \\ 
 \hline 
 		  same as SDIVCC, but with Instr[13]=0 (i=0), and Instr[5]=1. & 
 		\parbox{\linewidth}{rd(pair) $\leftarrow$ rs1(pair) /
                  rs2(pair) (signed),\\ sets condition codes
                  Z,N,Ovflow,\\ \textbf{Note:} can generate
                  div-by-zero trap.}\\ 
\hline
  \end{tabular}
  \caption{Multiplication and Division Instructions}
  \label{tab:muldiv:insns}
\end{table}

\begin{table}[p]
  \centering
  \begin{tabular}[p]{|p{.45\textwidth}|p{.45\textwidth}|}
    \hline
\multicolumn{2}{|l|}{\textbf{	ORD}} \\ 
 \hline 
 \parbox{\linewidth}{		  same as OR, but with Instr[13]=0 (i=0), and Instr[5]=1.} & 
 \parbox{\linewidth}{		rd(pair) $\leftarrow$ rs1(pair) $\vert$ rs2(pair)}\\
\hline
\hline
\multicolumn{2}{|l|}{\textbf{	ORDCC}} \\ 
 \hline 
 \parbox{\linewidth}{		  same as ORCC, but with Instr[13]=0 (i=0), and Instr[5]=1.} & 
 \parbox{\linewidth}{		rd(pair) $\leftarrow$ rs1(pair) $\vert$ rs2(pair),\\ sets Z.}\\
\hline
\hline
\multicolumn{2}{|l|}{\textbf{	ORDN}} \\ 
 \hline 
 \parbox{\linewidth}{		  same as ORN, but with Instr[13]=0 (i=0), and Instr[5]=1.} & 
 \parbox{\linewidth}{		rd(pair) $\leftarrow$ rs1(pair) $\vert$ ($\sim$rs2(pair))}\\
\hline
\hline
\multicolumn{2}{|l|}{\textbf{	ORDNCC}} \\ 
 \hline 
 \parbox{\linewidth}{		  same as ORNCC, but with Instr[13]=0 (i=0), and Instr[5]=1.} & 
 \parbox{\linewidth}{		rd(pair) $\leftarrow$ rs1(pair) $\vert$ ($\sim$rs2(pair)),\\ sets Z                 sets Z.}\\
\hline
\hline
\multicolumn{2}{|l|}{\textbf{	XORDCC}} \\ 
 \hline 
 \parbox{\linewidth}{		  same as XORCC, but with Instr[13]=0 (i=0), and Instr[5]=1.} & 
 \parbox{\linewidth}{		rd(pair) $\leftarrow$ rs1(pair) $\hat{}$ rs2(pair), \\sets Z		sets Z.}\\
\hline
\hline
\multicolumn{2}{|l|}{\textbf{	XNORD}} \\ 
 \hline 
 \parbox{\linewidth}{		  same as XNOR, but with Instr[13]=0 (i=0), and Instr[5]=1.} & 
 \parbox{\linewidth}{		rd(pair) $\leftarrow$ rs1(pair) $\hat{}$ rs2(pair)}\\
\hline
\hline
\multicolumn{2}{|l|}{\textbf{	XNORDCC}} \\ 
 \hline 
 \parbox{\linewidth}{		  same as XNORCC, but with Instr[13]=0 (i=0), and Instr[5]=1.} & 
 \parbox{\linewidth}{		rd(pair) $\leftarrow$ rs1(pair) $\hat{}$ rs2(pair),\\ sets Z}\\
\hline
\hline
\multicolumn{2}{|l|}{\textbf{	ANDD}} \\ 
 \hline 
 \parbox{\linewidth}{		  same as AND, but with Instr[13]=0 (i=0), and Instr[5]=1.} & 
 \parbox{\linewidth}{		rd(pair) $\leftarrow$ rs1(pair) . rs2(pair)}\\
\hline
\hline
\multicolumn{2}{|l|}{\textbf{	ANDDCC}} \\ 
 \hline 
 \parbox{\linewidth}{		  same as ANDCC, but with Instr[13]=0 (i=0), and Instr[5]=1.} & 
 \parbox{\linewidth}{		rd(pair) $\leftarrow$ rs1(pair) . rs2(pair),\\ sets Z}\\
\hline
\hline
\multicolumn{2}{|l|}{\textbf{	ANDDN}} \\ 
 \hline 
 \parbox{\linewidth}{		  same as ANDN, but with Instr[13]=0 (i=0), and Instr[5]=1.} & 
 \parbox{\linewidth}{		rd(pair) $\leftarrow$ rs1(pair) . ($\sim$rs2(pair))}\\
\hline
\hline
\multicolumn{2}{|l|}{\textbf{	ANDDNCC}} \\ 
 \hline 
 \parbox{\linewidth}{		  same as ANDNCC, but with Instr[13]=0 (i=0), and Instr[5]=1.} & 
 \parbox{\linewidth}{		rd $\leftarrow$ rs1 . ($\sim$rs2),\\ sets Z}\\
\hline
  \end{tabular}
  \caption{64 bit Logical Instructions}
  \label{tab:64bit:logical:insns}
\end{table}

% \subsection{Opcode Bit Patterns}
% \label{sec:opc:bits}

% \textbf{A. M. Vichare}

% Original SPArc V8 ADD is:  (See: SPArc v8 architecture manual, pg. 108
% (pg. 130 in PDF sequence)).

% Bit format: 32 bits in total

% \begin{tabular}[p]{|c|c|l|l|l|}
% \hline
%   \textbf{Start} & \textbf{End} & \textbf{Range} & \textbf{Meaning} &
%   \textbf{New Meaning}\\
% \hline
% 0 & 4 & 32 & Source register 2, rs2 & No change \\
% 5 & 12 & -- & \textbf{unused} & \textbf{Set bit 5 to ``1''} \\
% 13 & 13 & 0,1 & The \textbf{i} bit & \textbf{Set i to ``0''} \\
% 14 & 18 & 32 & Source register 1, rs1 & No change \\
% 19 & 24 & 000000 & ``\textbf{op3}'' & No change \\
% 25 & 29 & 32 & Destination register, rd & No change \\
% 30 & 31 & 4 & Always ``10'' & No change \\
% \hline
% \end{tabular}

% New addition:
% \textbf{ADDD}: same as ADD, but with Instr[13]=0 (i=0), and Instr[5]=1.
%  		rd(pair) $\leftarrow$ rs1(pair) + rs2(pair).\\
% \textbf{Syntax}: ``\texttt{addd  SrcReg1, SrcReg2, DestReg}''.

% Bits layout to determine the ``match'' etc. bit masks.

% Bit offset: 31          23          15           7       0 \newline
% Bit layout:  0000 0000   0000 0000   0000 0000   0000 0000 \newline
% Bits match:  10      0   0000 0        0           1       \newline
% Bit tomask:  1100 0001   1111 1000   0010 0000   0010 0000 \newline
% The bitmas:  C0   01     FF   80     02   00     02   00

% Final match: C001 FF80 0200 0200

% % \begin{tabular}[p]{|c|c|l|l|}
% % \hline
% %   \textbf{Start} & \textbf{End} & \textbf{Range} & \textbf{Meaning} \\
% % \hline
% % 0 & 4 & 32 & Source register 2, rs2 \\
% % 5 & 12 & -- & \textbf{unused} \textbf{-- Set bit 5 to ``1''}\\
% % 13 & 13 & 0,1 & The \textbf{i} bit \textbf{-- Set to ``0'' }\\
% % 14 & 18 & 32 & Source register 1, rs1 \\
% % 19 & 24 & 000000 & ``\textbf{op3}'' \\
% % 25 & 29 & 32 & Destination register, rd \\
% % 30 & 31 & 4 & Always ``10'' \\
% % \hline
% % \end{tabular}

% % \newpage
\subsection{Integer-Unit Extensions: SIMD Instructions}
\label{sec:integer-unit-extns:simd-instructions}

These instructions  are vector instructions  which work on  two source
registers (each a  64 bit register pair), and produce  a 64-bit vector
result.   The   vector  elements  can  be   8-bit/16-bit/32-bit.   See
Table~\ref{tab:simd:insns}.

\begin{table}[p]
  \centering
  \begin{tabular}[p]{|p{.45\textwidth}|p{.45\textwidth}|}
    \hline
\multicolumn{2}{|l|}{\textbf{	VADDD8, VADDD16, VADDD32}} \\ 
 \hline 
 \parbox{\linewidth}{        Same as ADDD, but with Instr[13]=0 (i=0),
    and Instr[6:5]=2. Bits Instr[9:7] are a 3-bit field, which specify
    the data type \\
\begin{tabular}[p]{|l|l|l|}
\hline
  001  &   byte			 & (VADDD8)\\
  010  &   half-word (16-bits)	 & (VADDD16)\\
  100  &   word (32-bits) 		 & (VADDD32)\\
\hline
\end{tabular}\\
} & 
 \parbox{\linewidth}{        Performs a vector operation by considering the 64-bit operands as a vector of objects with specified data-type.}\\
\hline
\hline
\multicolumn{2}{|l|}{\textbf{	VSUBD8, VSUBD16, VSUBD32}} \\ 
 \hline 
 \parbox{\linewidth}{        Same as ADDD, but with Instr[13]=0 (i=0),
    and Instr[6:5]=2.  Bits Instr[9:7] are a 3-bit field, which
    specify the data type\\
\begin{tabular}[p]{|l|l|l|}
\hline
  001  &   byte 			 & (VSUBD8)\\
  010  &   half-word (16-bits)	 & (VSUBD16)\\
  100  &   word (32-bits) 		 & (VSUBD32)\\
\hline
\end{tabular}\\
} & 
 \parbox{\linewidth}{        Performs a vector operation by considering the 64-bit operands as a vector of objects with specified data-type.}\\
\hline
\hline
\multicolumn{2}{|l|}{\textbf{	VUMULD8, VUMULD16, VUMULD32}} \\ 
 \hline 
 \parbox{\linewidth}{        Same as ADDD, but with Instr[13]=0 (i=0),
    and Instr[6:5]=2. Bits Instr[9:7] are a 3-bit field, which specify
    the data type\\
\begin{tabular}[p]{|l|l|l|}
\hline
  001  &   byte			 & (VMULD8)\\
  010  &   half-word (16-bits)	 & (VMULD16)\\
  100  &   word (32-bits) 		 & (VMULD32)\\
\hline
\end{tabular}\\
} & 
 \parbox{\linewidth}{	Performs a vector operation by considering the 64-bit operands as a vector of objects with specified data-type.}\\
\hline
\hline
\multicolumn{2}{|l|}{\textbf{	VSMULD8, VSUMLD16, VSMULD32}} \\ 
 \hline 
 \parbox{\linewidth}{        Same as ADDD, but with Instr[13]=0 (i=0),
    and Instr[6:5]=2. Bits Instr[9:7] are a 3-bit field, which specify
    the data type\\
\begin{tabular}[p]{|l|l|l|}
\hline
  001  &   byte			 & (VSMULD8)\\
  010  &   half-word (16-bits)	 & (VSMULD16)\\
  100  &   word (32-bits) 		 & (VSMULD32)\\
\hline
\end{tabular}\\
} & 
 \parbox{\linewidth}{	Performs a vector operation by considering the 64-bit operands as a vector of objects with specified data-type.}\\
\hline
  \end{tabular}
  \caption{SIMD Instructions}
  \label{tab:simd:insns}
\end{table}

% \newpage
\subsection{Integer-Unit Extensions: SIMD Instructions II}
\label{sec:integer-unit-extns:simd-instructions:2}

These  instructions  are vector  instructions  which  reduce a  source
register to a byte result.  See Table~\ref{tab:simd:2:insns}.

\begin{table}[p]
  \centering
  \begin{tabular}[p]{|p{.45\textwidth}|p{.45\textwidth}|}
    \hline
\multicolumn{2}{|l|}{\textbf{ORDBYTER} (Byte-Reduce OR)} \\ 
 \hline 
 \parbox{\linewidth}{op=2, op3[3:0]=0xe, op3[5:4]=0x2, contents[7:0]
    of rs2 specify a mask.\\

    Instr[31:30] (op) = 0x2\\
    Instr[29:25] (rd)    lowest bit assumed 0.\\
    Instr[24:19] (op3) = 111010\\
    Instr[18:14] (rs1)   lowest bit assumed 0.\\
    Instr[13]    (i)  = 0 (ignored)\\
    Instr[12:5]   (zero)\\
    Instr[4:0]   (rs2)   32-bit register is read.\\
} & 
 \parbox{\linewidth}{rd $\leftarrow$ (rs1\_7.m7 $\vert$ rs1\_6.m6 $\vert$ rs1\_5.m5 ... $\vert$ rs1\_0.m0)}\\
\hline
\hline
\multicolumn{2}{|l|}{\textbf{ANDDBYTER} (Byte-Reduce AND)} \\ 
 \hline 
 \parbox{\linewidth}{op=2, op3[3:0]=0xf, op3[5:4]=0x2, contents[7:0]
    of rs2 specify a mask.\\

    Instr[31:30] (op) = 0x2\\
    Instr[29:25] (rd)    lowest bit assumed 0.\\
    Instr[24:19] (op3) = 111110\\
    Instr[18:14] (rs1)   lowest bit assumed 0.\\
    Instr[13]    (i)  = 0 (ignored)\\
    Instr[12:5]   (zero)\\
    Instr[4:0]   (rs2)   32-bit register is read.\\
} & 
 \parbox{\linewidth}{rd $\leftarrow$ ( (m7 ? rs1\_7 : 0xff) . (m6 ? rs1\_6 : 0xff) \ldots (m0 ? rs1\_0 : 0xff))}\\
\hline
\hline
\multicolumn{2}{|l|}{\textbf{XORDBYTER} (Byte-Reduce XOR)} \\ 
 \hline 
 \parbox{\linewidth}{op=2, op3[3:0]=0xe, op3[5:4]=0x3, contents[7:0]
    of rs2 specify a mask.\\

    Instr[31:30] (op) = 0x2\\
    Instr[29:25] (rd)    lowest bit assumed 0.\\
    Instr[24:19] (op3) = 111011\\
    Instr[18:14] (rs1)   lowest bit assumed 0.\\
    Instr[13]    (i)  = 0 (ignored)\\
    Instr[12:5]   (zero)\\
    Instr[4:0]   (rs2)   32-bit register is read.\\
} & 
 \parbox{\linewidth}{rd $\leftarrow$ (rs1\_7.m7 $\hat{}$ rs1\_6.m6 $\hat{}$ rs1\_5.m5 ... $\hat{}$ rs1\_0.m0)}\\
\hline
\hline
\multicolumn{2}{|l|}{\textbf{ZBYTEDPOS} (Positions-of-Zero-Bytes in D-Word)} \\ 
 \hline 
 \parbox{\linewidth}{op=2, op3[3:0]=0xf, op3[5:4]=0x3, contents[7:0]
    of rs2/imm-value specify a mask.\\

    Instr[31:30] (op) = 0x2\\
    Instr[29:25] (rd)    lowest bit assumed 0.\\
    Instr[24:19] (op3) = 111011\\
    Instr[18:14] (rs1)   lowest bit assumed 0.\\
    Instr[13]    (i)  =  if 0, use rs2, else Instr[7:0]\\
    Instr[12:5]  = 0  (ignored if i=0)\\
    Instr[4:0]   (rs2, if i=0) 32-bit register is read.\\
} & 
 \parbox{\linewidth}{rd $\leftarrow$ [b7\_zero b6\_zero b5\_zero b4\_zero \ldots b0\_zero] (if mask-bit is zero then b$\star$\_zero is zero)}\\
\hline
  \end{tabular}
  \caption{SIMD Instructions II}
  \label{tab:simd:2:insns}
\end{table}

% \newpage
\subsection{Vector Floating Point Instructions}
\label{sec:vector-floating-point-instructions}

These are vector  float operations which work on  two single precision
operand  pairs   to  produce   two  single  precision   results.   See
Table~\ref{tab:simd:float:ops}.

\begin{table}[p]
  \centering
  \begin{tabular}[p]{|l|l|}
    \hline
    \textbf{VFADD} & {op=2, op3=0x34, opf=0x142} \\
    \hline
    \textbf{VFSUB} & {op=2, op3=0x34, opf=0x146} \\
    \hline
    \textbf{VFMUL} & {op=2, op3=0x34, opf=0x14a} \\
    \hline
    \textbf{VFDIV} & {op=2, op3=0x34, opf=0x14e} \\
    \hline
    \textbf{VFSQRT} & {op=2, op3=0x34, opf=0x12a} \\
    \hline      
  \end{tabular}
  \caption[SIMD Floating Point Operations]{SIMD Floating Point
    Operations.  NaN propagated, but no traps. For each of these,
    rs1,rs2,rd are considered even numbers pointing to.
  }
  \label{tab:simd:float:ops}
\end{table}

\subsection{CSWAP instructions}
\label{sec:cswap-instructions}

The Sparc-V8 ISA does not include a compare-and-swap (CAS) instruction
which is very  useful in achieving consensus  among distributed agents
when the number of agents is $>$ 2.  See Table~\ref{tab:cswap:insns}.

We introduce a CSWAP instruction in two flavours:
		% CSWAP64     rs1, rs2-pair/immediate, rd-pair
		% 	op=3
		% 	op3= 10 1111
		% 		(rest of instruction similar to SWAP)
			
		% CSWAP64A    rs1, rs2-pair/immediate, rd-pair, asi
		% 	op=3
		% 	op3= 11 1111
                %         (rest of instruction similar to SWAPA)

                %         // CSWAP64 has no explicit ASI, while CSWAP64A
                %         // does! Any ambiguity issues?

\begin{table}[p]
  \centering
  \begin{tabular}[p]{|p{.45\textwidth}|p{.45\textwidth}|}
    \hline
\multicolumn{2}{|l|}{\textbf{CSWAP64} (effective address in registers
    rs1 and rs2)} \\ 
 \hline 
 \parbox{\linewidth}{op=3, op3=10 1111, i=0.\\

    Instr[31:30] (op) = 0x3\\
    Instr[29:25] (rd)    lowest bit assumed 0.\\
    Instr[24:19] (op3) = 101111\\
    Instr[18:14] (rs1)   lowest bit assumed 0.\\
    Instr[13]    (i)  = 0 (registers based effective address)\\
    Instr[12:5]  (asi) = Address Space Identifier (See: Appendix G of V8)\\
    Instr[4:0]   (rs2)   32-bit register is read.\\
} & 
 \parbox{\linewidth}{~}\\
\hline
    \hline
\multicolumn{2}{|l|}{\textbf{CSWAP64} (immediate effective address)} \\ 
 \hline 
 \parbox{\linewidth}{op=3, op3=10 1111, i=1.\\

    Instr[31:30] (op) = 0x3\\
    Instr[29:25] (rd)    lowest bit assumed 0.\\
    Instr[24:19] (op3) = 101111\\
    Instr[18:14] (rs1)   lowest bit assumed 0.\\
    Instr[13]    (i)  = 1 (immediate effective address)\\
    Instr[12:0]  (simm13) 13-bit immediate address.\\
} & 
 \parbox{\linewidth}{~}\\
\hline
    \hline
\multicolumn{2}{|l|}{\textbf{CSWAP64A} (effective address in registers
    rs1 and rs2)} \\ 
 \hline 
 \parbox{\linewidth}{op=3, op3=10 1111, i=0.\\

    Instr[31:30] (op) = 0x3\\
    Instr[29:25] (rd)    lowest bit assumed 0.\\
    Instr[24:19] (op3) = 111111\\
    Instr[18:14] (rs1)   lowest bit assumed 0.\\
    Instr[13]    (i)  = 0 (registers based effective address)\\
    Instr[12:5]  (asi) = Address Space Identifier (See: Appendix G of V8)\\
    Instr[4:0]   (rs2)   32-bit register is read.\\
} & 
 \parbox{\linewidth}{~}\\
\hline
    \hline
\multicolumn{2}{|l|}{\textbf{CSWAP64A} (immediate effective address)} \\ 
 \hline 
 \parbox{\linewidth}{op=3, op3=10 1111, i=1.\\

    Instr[31:30] (op) = 0x3\\
    Instr[29:25] (rd)    lowest bit assumed 0.\\
    Instr[24:19] (op3) = 111111\\
    Instr[18:14] (rs1)   lowest bit assumed 0.\\
    Instr[13]    (i)  = 1 (immediate effective address)\\
    Instr[12:0]  (simm13) 13-bit immediate address.\\
} & 
 \parbox{\linewidth}{~}\\
\hline
  \end{tabular}
  \caption{CSWAP Instructions}
  \label{tab:cswap:insns}
\end{table}
  
\newpage
\chapter{AJIT Support for the GNU Binutils Toolchain}
\label{chap:amv:work}

\section{Towards a GNU Binutils Toolchain}
\label{sec:binutils:support}

This section describes the details  of adding the AJIT instructions to
SPARC v8 part of GNU Binutils 2.22.  We use the SPARC v8 manual to get
the details of  the sparc instruction.  It's bit  pattern is described
\emph{again}, and  the new  bit pattern  required for  AJIT is  set up
alongside.  Bit layouts to determine  the ``match'' etc.  of the sparc
port  are  also  laid  out.    The  SPARC  manual  also  contains  the
``suggested asm syntax''  that we adapt for the  new AJIT instruction.
The sections below follow the sections in chapter~\ref{sec:isa:extns}.
For each  instruction, we  need to  define its  bitfields in  terms of
macros  in  \texttt{\$BINUTILSHOME/include/opcode/sparc.h} and  define
the opcodes table in \texttt{\$BINUTILSHOME/opcodes/sparc-opc.c}.

The AJIT  instructions are  variations of  the corresponding  SPARC V8
instructions.  Please refer to the SPARC V8 manual for details of such
corresponding SPARC  instructions. For example,  the\texttt{ADD} insn,
pg. 108 (pg.  130 in PDF sequence) of the  manual.  Other instructions
can be similarly found, and will not be mentioned.

\subsection{Integer-Unit Extensions: Arithmetic-Logic Instructions}
\label{sec:integer-unit-extns:arith-logic-insns:impl}

The  integer  unit extensions  of  AJIT  are  based  on the  SPARC  V8
instructions.    See:  SPArc   v8  architecture   manual.   SPARC   v8
instructions  are  32   bits  long.   The  GNU   Binutils  2.22  SPARC
implementation defines a  set of macros to capture the  bits set by an
instruction.  These are the so called ``match'' masks.  Please see the
code     in     \texttt{\$BINUTILSHOME/include/opcode/sparc.h}     and
\texttt{\$BINUTILSHOME/opcodes/sparc-opc.c}.
\vfill
\newpage
\subsubsection{Addition and subtraction instructions:}
\label{sec:add:sub:insn:impl}
\begin{enumerate}
\item \textbf{ADDD}:\\
  \begin{center}
    \begin{tabular}[p]{|c|c|l|l|l|}
      \hline
      \textbf{Start} & \textbf{End} & \textbf{Range} & \textbf{Meaning} &
                                                                          \textbf{New Meaning}\\
      \hline
      0 & 4 & 32 & Source register 2, rs2 & No change \\
      5 & 12 & -- & \textbf{unused} & \textbf{Set bit 5 to ``1''} \\
      13 & 13 & 0,1 & The \textbf{i} bit & \textbf{Set i to ``0''} \\
      14 & 18 & 32 & Source register 1, rs1 & No change \\
      19 & 24 & 000000 & ``\textbf{op3}'' & No change \\
      25 & 29 & 32 & Destination register, rd & No change \\
      30 & 31 & 4 & Always ``10'' & No change \\
      \hline
    \end{tabular}
  \end{center}
  \begin{itemize}
  \item []\textbf{ADDD}: same as ADD, but with Instr[13]=0 (i=0), and
    Instr[5]=1.
  \item []\textbf{Syntax}: ``\texttt{addd  SrcReg1, SrcReg2, DestReg}''.
  \item []\textbf{Semantics}: rd(pair) $\leftarrow$ rs1(pair) + rs2(pair).
  \end{itemize}
  Bits layout:
\begin{verbatim}
    Offsets      : 31       24 23       16  15        8   7        0
    Bit layout   :  XXXX  XXXX  XXXX  XXXX   XXXX  XXXX   XXXX  XXXX
    Insn Bits    :  10       0  0000  0        0            1       
    Destination  :    DD  DDD                                       
    Source 1     :                     SSS   SS
    Source 2     :                                           S  SSSS
    Unused (0)   :                              U  UUUU   UU        
    Final layout :  10DD  DDD0  0000  0SSS   SS0U  UUUU   UU1S  SSSS
\end{verbatim}

  Hence the SPARC bit layout of this instruction is:

  \begin{tabular}[h]{lclcl}
    Macro to set  &=& \texttt{F4(x, y, z)} &in& \texttt{sparc.h}     \\
    Macro to reset  &=& \texttt{INVF4(x, y, z)} &in& \texttt{sparc.h}     \\
    x &=& 0x2      &in& \texttt{OP(x)  /* ((x) \& 0x3)  $<<$ 30 */} \\
    y &=& 0x00     &in& \texttt{OP3(y) /* ((y) \& 0x3f) $<<$ 19 */} \\
    z &=& 0x0      &in& \texttt{F3I(z) /* ((z) \& 0x1)  $<<$ 13 */} \\
    a &=& 0x1      &in& \texttt{OP\_AJIT\_BIT(a) /* ((a) \& 0x1)  $<<$ 5 */}
  \end{tabular}

  The AJIT bit  (insn[5]) is set internally by  \texttt{F4}, and hence
  there are only three arguments.

\item \textbf{ADDDCC}:\\
  \begin{center}
    \begin{tabular}[p]{|c|c|l|l|l|}
      \hline
      \textbf{Start} & \textbf{End} & \textbf{Range} & \textbf{Meaning} &
                                                                          \textbf{New Meaning}\\
      \hline
      0 & 4 & 32 & Source register 2, rs2 & No change \\
      5 & 12 & -- & \textbf{unused} & \textbf{Set bit 5 to ``1''} \\
      13 & 13 & 0,1 & The \textbf{i} bit & \textbf{Set i to ``0''} \\
      14 & 18 & 32 & Source register 1, rs1 & No change \\
      19 & 24 & 010000 & ``\textbf{op3}'' & No change \\
      25 & 29 & 32 & Destination register, rd & No change \\
      30 & 31 & 4 & Always ``10'' & No change \\
      \hline
    \end{tabular}
  \end{center}
  New addition:
  \begin{itemize}
  \item []\textbf{ADDDCC}: same as ADDCC, but with Instr[13]=0 (i=0), and
    Instr[5]=1.
  \item []\textbf{Syntax}: ``\texttt{adddcc  SrcReg1, SrcReg2, DestReg}''.
  \item []\textbf{Semantics}: rd(pair) $\leftarrow$ rs1(pair) + rs2(pair), set Z,N
  \end{itemize}
  Bits layout:
\begin{verbatim}
    Offsets      : 31       24 23       16  15        8   7        0
    Bit layout   :  XXXX  XXXX  XXXX  XXXX   XXXX  XXXX   XXXX  XXXX
    Insn Bits    :  10       0  1000  0        0            1       
    Destination  :    DD  DDD                                       
    Source 1     :                     SSS   SS
    Source 2     :                                           S  SSSS
    Unused (0)   :                              U  UUUU   UU        
    Final layout :  10DD  DDD0  1000  0SSS   SS0U  UUUU   UU1S  SSSS
\end{verbatim}

  Hence the SPARC bit layout of this instruction is:

  \begin{tabular}[h]{lclcl}
    Macro to set  &=& \texttt{F4(x, y, z)} &in& \texttt{sparc.h}     \\
    Macro to reset  &=& \texttt{INVF4(x, y, z)} &in& \texttt{sparc.h}     \\
    x &=& 0x2      &in& \texttt{OP(x)  /* ((x) \& 0x3)  $<<$ 30 */} \\
    y &=& 0x10     &in& \texttt{OP3(y) /* ((y) \& 0x3f) $<<$ 19 */} \\
    z &=& 0x0      &in& \texttt{F3I(z) /* ((z) \& 0x1)  $<<$ 13 */} \\
    a &=& 0x1      &in& \texttt{OP\_AJIT\_BIT(a) /* ((a) \& 0x1)  $<<$ 5 */}
  \end{tabular}

  The AJIT bit  (insn[5]) is set internally by  \texttt{F4}, and hence
  there are only three arguments.

\item \textbf{SUBD}:\\
  \begin{center}
    \begin{tabular}[p]{|c|c|l|l|l|}
      \hline
      \textbf{Start} & \textbf{End} & \textbf{Range} & \textbf{Meaning} &
                                                                          \textbf{New Meaning}\\
      \hline
      0 & 4 & 32 & Source register 2, rs2 & No change \\
      5 & 12 & -- & \textbf{unused} & \textbf{Set bit 5 to ``1''} \\
      13 & 13 & 0,1 & The \textbf{i} bit & \textbf{Set i to ``0''} \\
      14 & 18 & 32 & Source register 1, rs1 & No change \\
      19 & 24 & 000100 & ``\textbf{op3}'' & No change \\
      25 & 29 & 32 & Destination register, rd & No change \\
      30 & 31 & 4 & Always ``10'' & No change \\
      \hline
    \end{tabular}
  \end{center}
  New addition:
  \begin{itemize}
  \item []\textbf{SUBD}: same as SUB, but with Instr[13]=0 (i=0), and
    Instr[5]=1.
  \item []\textbf{Syntax}: ``\texttt{subd  SrcReg1, SrcReg2, DestReg}''.
  \item []\textbf{Semantics}: rd(pair) $\leftarrow$ rs1(pair) - rs2(pair).
  \end{itemize}
  Bits layout:
\begin{verbatim}
    Offsets      : 31       24 23       16  15        8   7        0
    Bit layout   :  XXXX  XXXX  XXXX  XXXX   XXXX  XXXX   XXXX  XXXX
    Insn Bits    :  10       0  0010  0        0            1       
    Destination  :    DD  DDD                                       
    Source 1     :                     SSS   SS
    Source 2     :                                           S  SSSS
    Unused (0)   :                              U  UUUU   UU        
    Final layout :  10DD  DDD0  0010  0SSS   SS0U  UUUU   UU1S  SSSS
\end{verbatim}

  Hence the SPARC bit layout of this instruction is:

  \begin{tabular}[h]{lclcl}
    Macro to set  &=& \texttt{F4(x, y, z)} &in& \texttt{sparc.h}     \\
    Macro to reset  &=& \texttt{INVF4(x, y, z)} &in& \texttt{sparc.h}     \\
    x &=& 0x2      &in& \texttt{OP(x)  /* ((x) \& 0x3)  $<<$ 30 */} \\
    y &=& 0x04     &in& \texttt{OP3(y) /* ((y) \& 0x3f) $<<$ 19 */} \\
    z &=& 0x0      &in& \texttt{F3I(z) /* ((z) \& 0x1)  $<<$ 13 */} \\
    a &=& 0x1      &in& \texttt{OP\_AJIT\_BIT(a) /* ((a) \& 0x1)  $<<$ 5 */}
  \end{tabular}

  The AJIT bit  (insn[5]) is set internally by  \texttt{F4}, and hence
  there are only three arguments.

\item \textbf{SUBDCC}:\\
  \begin{center}
    \begin{tabular}[p]{|c|c|l|l|l|}
      \hline
      \textbf{Start} & \textbf{End} & \textbf{Range} & \textbf{Meaning} &
                                                                          \textbf{New Meaning}\\
      \hline
      0 & 4 & 32 & Source register 2, rs2 & No change \\
      5 & 12 & -- & \textbf{unused} & \textbf{Set bit 5 to ``1''} \\
      13 & 13 & 0,1 & The \textbf{i} bit & \textbf{Set i to ``0''} \\
      14 & 18 & 32 & Source register 1, rs1 & No change \\
      19 & 24 & 010100 & ``\textbf{op3}'' & No change \\
      25 & 29 & 32 & Destination register, rd & No change \\
      30 & 31 & 4 & Always ``10'' & No change \\
      \hline
    \end{tabular}
  \end{center}
  New addition:
  \begin{itemize}
  \item []\textbf{SUBDCC}: same as SUBCC, but with Instr[13]=0 (i=0), and
    Instr[5]=1.
  \item []\textbf{Syntax}: ``\texttt{subdcc  SrcReg1, SrcReg2, DestReg}''.
  \item []\textbf{Semantics}: rd(pair) $\leftarrow$ rs1(pair) - rs2(pair), set Z,N
  \end{itemize}
  Bits layout:
\begin{verbatim}
    Offsets      : 31       24 23       16  15        8   7        0
    Bit layout   :  XXXX  XXXX  XXXX  XXXX   XXXX  XXXX   XXXX  XXXX
    Insn Bits    :  10       0  1010  0        0            1       
    Destination  :    DD  DDD                                       
    Source 1     :                     SSS   SS
    Source 2     :                                           S  SSSS
    Unused (0)   :                              U  UUUU   UU        
    Final layout :  10DD  DDD0  1010  0SSS   SS0U  UUUU   UU1S  SSSS
\end{verbatim}

  Hence the SPARC bit layout of this instruction is:

  \begin{tabular}[h]{lclcl}
    Macro to set  &=& \texttt{F4(x, y, z)} &in& \texttt{sparc.h}     \\
    Macro to reset  &=& \texttt{INVF4(x, y, z)} &in& \texttt{sparc.h}     \\
    x &=& 0x2      &in& \texttt{OP(x)  /* ((x) \& 0x3)  $<<$ 30 */} \\
    y &=& 0x14     &in& \texttt{OP3(y) /* ((y) \& 0x3f) $<<$ 19 */} \\
    z &=& 0x0      &in& \texttt{F3I(z) /* ((z) \& 0x1)  $<<$ 13 */} \\
    a &=& 0x1      &in& \texttt{OP\_AJIT\_BIT(a) /* ((a) \& 0x1)  $<<$ 5 */}
  \end{tabular}

  The AJIT bit  (insn[5]) is set internally by  \texttt{F4}, and hence
  there are only three arguments.
\end{enumerate}

\subsubsection{Shift instructions:}
\label{sec:shift:insn:impl}
The shift  family of instructions  of AJIT  may each be  considered to
have  two versions:  a direct  count version  and a  register indirect
count version.  In the direct count  version the shift count is a part
of the  instruction bits.   In the indirect  count version,  the shift
count is  found on the  register specified by  the bit pattern  in the
instruction  bits.   The direct  count  version  is specified  by  the
14$^{th}$  bit, i.e.  insn[13]  (bit  number 13  in  the  0 based  bit
numbering scheme), being set to 1.  If insn[13] is 0 then the register
indirect version is specified.

Similar to the addition and subtraction instructions, the shift family
of instructions of  SPARC V8 also do  not use bits from 5  to 12 (both
inclusive).  The AJIT processor uses bits  5 and 6.  In particular bit
6 is always 1.   Bit 5 may be used in the direct  version giving a set
of 6 bits  available for specifying the shift count.   The shift count
can have  a maximum  value of  64.  Bit  5 is  unused in  the register
indirect version, and is always 0 in that case.

These instructions  are therefore  worked out  below in  two different
sets: the direct and the register indirect ones.
\begin{enumerate}
\item The direct versions  are given by insn[13] = 1.  The 6 bit shift
  count  is directly  specified  in the  instruction bits.   Therefore
  insn[5:0] specify the  shift count.  insn[6] =  1, distinguishes the
  AJIT version from the SPARC V8 version.
  \begin{enumerate}
  \item \textbf{SLLD}:\\
    \begin{center}
      \begin{tabular}[p]{|c|c|l|p{.25\textwidth}|p{.3\textwidth}|}
        \hline
        \textbf{Start} & \textbf{End} & \textbf{Range} & \textbf{Meaning} & \textbf{New Meaning}\\
        \hline
        0 & 4 & 32 & Source register 2, rs2 & Lowest 5 bits of shift count \\
        \hline
        5 & 12 & -- & \textbf{Unused. Set to 0 by software.} &
                                        \begin{minipage}[h]{1.0\linewidth}
                                          \begin{itemize}
                                          \item \textbf{Use bit 5
                                              to specify the msb of
                                              shift count.}
                                          \item \textbf{Use bit 6 to
                                              distinguish AJIT from
                                              SPARC V8.}
                                          \item \textbf{Set bits 7:12
                                              to 0.}
                                          \end{itemize}
                                        \end{minipage}
        \\
        \hline
        13 & 13 & 0,1 & The \textbf{i} bit & \textbf{Set i to ``1''} \\
        14 & 18 & 32 & Source register 1, rs1 & No change \\
        19 & 24 & 100101 & ``\textbf{op3}'' & No change \\
        25 & 29 & 32 & Destination register, rd & No change \\
        30 & 31 & 4 & Always ``10'' & No change \\
        \hline
      \end{tabular}
    \end{center}
    \begin{itemize}
    \item []\textbf{SLLD}: same as SLL, but with Instr[13]=0 (i=0),
      and Instr[5]=1.
    \item []\textbf{Syntax}: ``\texttt{slld SrcReg1, 6BitShiftCnt,
        DestReg}''. \\
      (\textbf{Note:} In an assembly language program, when the second
      argument is a number, we have direct mode.  A register number is
      prefixed with  ``r'', and hence the  syntax itself distinguished
      between   direct  and   register   indirect   version  of   this
      instruction.)
    \item []\textbf{Semantics}: rd(pair) $\leftarrow$ rs1(pair) $<<$
      shift count.
    \end{itemize}
    Bits layout:
\begin{verbatim}
    Offsets      : 31       24 23       16  15        8   7        0
    Bit layout   :  XXXX  XXXX  XXXX  XXXX   XXXX  XXXX   XXXX  XXXX
    Insn Bits    :  10       1  0010  1        1           1        
    Destination  :    DD  DDD                                       
    Source 1     :                     SSS   SS
    Source 2     :                                           S  SSSS
    Unused (0)   :                              U  UUUU   UU        
    Final layout :  10DD  DDD1  0010  1SSS   SS1U  UUUU   U1II  IIII
\end{verbatim}

    This will need another macro that sets bits 5 and 6. Let's call it
    \texttt{OP\_AJIT\_BITS\_5\_AND\_6}.   Hence the  SPARC bit  layout of  this
    instruction is:

    \begin{tabular}[h]{lclcl}
      Macro to set  &=& \texttt{F5(x, y, z)} &in& \texttt{sparc.h}     \\
      Macro to reset  &=& \texttt{INVF5(x, y, z)} &in& \texttt{sparc.h}     \\
      x &=& 0x2      &in& \texttt{OP(x)  /* ((x) \& 0x3)  $<<$ 30 */} \\
      y &=& 0x25     &in& \texttt{OP3(y) /* ((y) \& 0x3f) $<<$ 19 */} \\
      z &=& 0x1      &in& \texttt{F3I(z) /* ((z) \& 0x1)  $<<$ 13 */} \\
      a &=& 0x2      &in& \texttt{OP\_AJIT\_BITS\_5\_AND\_6(a) /* ((a) \& 0x3  $<<$ 6 */}
    \end{tabular}

    The AJIT bits (insn[6:5]) is  set or reset internally by \texttt{F5}
    (just  like  in  \texttt{F4}),  and   hence  there  are  only  three
    arguments.

  \item \textbf{SRLD}:\\
    \begin{center}
      \begin{tabular}[p]{|c|c|l|l|p{.35\textwidth}|}
        \hline
        \textbf{Start} & \textbf{End} & \textbf{Range} & \textbf{Meaning} & \textbf{New Meaning}\\
        \hline
        0 & 4 & 32 & Source register 2, rs2 & Lowest 5 bits of shift count \\
        \hline
        5 & 12 & -- & \textbf{unused} &
                                        \begin{minipage}[h]{1.0\linewidth}
                                          \begin{itemize}
                                          \item \textbf{Use bit 5
                                              to specify the msb of
                                              shift count.}
                                          \item \textbf{Use bit 6 to
                                              distinguish AJIT from
                                              SPARC V8.}
                                          \end{itemize}
                                        \end{minipage}
        \\
        \hline
        13 & 13 & 0,1 & The \textbf{i} bit & \textbf{Set i to ``1''} \\
        14 & 18 & 32 & Source register 1, rs1 & No change \\
        19 & 24 & 100110 & ``\textbf{op3}'' & No change \\
        25 & 29 & 32 & Destination register, rd & No change \\
        30 & 31 & 4 & Always ``10'' & No change \\
        \hline
      \end{tabular}
    \end{center}
    \begin{itemize}
    \item []\textbf{SRLD}: same as SRL, but with Instr[13]=0 (i=0),
      and Instr[5]=1.
    \item []\textbf{Syntax}: ``\texttt{sral SrcReg1, 6BitShiftCnt,
        DestReg}''. \\
      (\textbf{Note:} In an assembly language program, when the second
      argument is a number, we have direct mode.  A register number is
      prefixed with  ``r'', and hence the  syntax itself distinguished
      between   direct  and   register   indirect   version  of   this
      instruction.)
    \item []\textbf{Semantics}: rd(pair) $\leftarrow$ rs1(pair) $>>$
      shift count.
    \end{itemize}
    Bits layout:
\begin{verbatim}
    Offsets      : 31       24 23       16  15        8   7        0
    Bit layout   :  XXXX  XXXX  XXXX  XXXX   XXXX  XXXX   XXXX  XXXX
    Insn Bits    :  10       1  0011  0        1           1        
    Destination  :    DD  DDD                                       
    Source 1     :                     SSS   SS
    Source 2     :                                           S  SSSS
    Unused (0)   :                              U  UUUU   UU        
    Final layout :  10DD  DDD1  0011  0SSS   SS1U  UUUU   U1II  IIII
\end{verbatim}

    This will need another macro that sets bits 5 and 6. Let's call it
    \texttt{OP\_AJIT\_BITS\_5\_AND\_6}.   Hence the  SPARC bit  layout of  this
    instruction is:

    \begin{tabular}[h]{lclcl}
      Macro to set  &=& \texttt{F5(x, y, z)} &in& \texttt{sparc.h}     \\
      Macro to reset  &=& \texttt{INVF5(x, y, z)} &in& \texttt{sparc.h}     \\
      x &=& 0x2      &in& \texttt{OP(x)  /* ((x) \& 0x3)  $<<$ 30 */} \\
      y &=& 0x26     &in& \texttt{OP3(y) /* ((y) \& 0x3f) $<<$ 19 */} \\
      z &=& 0x1      &in& \texttt{F3I(z) /* ((z) \& 0x1)  $<<$ 13 */} \\
      a &=& 0x2      &in& \texttt{OP\_AJIT\_BITS\_5\_AND\_6(a) /* ((a) \& 0x3  $<<$ 6 */}
    \end{tabular}

    The AJIT bits (insn[6:5]) is  set or reset internally by \texttt{F5}
    (just  like  in  \texttt{F4}),  and   hence  there  are  only  three
    arguments.
    
  \item \textbf{SRAD}:\\
    \begin{center}
      \begin{tabular}[p]{|c|c|l|l|p{.35\textwidth}|}
        \hline
        \textbf{Start} & \textbf{End} & \textbf{Range} & \textbf{Meaning} & \textbf{New Meaning}\\
        \hline
        0 & 4 & 32 & Source register 2, rs2 & Lowest 5 bits of shift count \\
        \hline
        5 & 12 & -- & \textbf{unused} &
                                        \begin{minipage}[h]{1.0\linewidth}
                                          \begin{itemize}
                                          \item \textbf{Use bit 5
                                              to specify the msb of
                                              shift count.}
                                          \item \textbf{Use bit 6 to
                                              distinguish AJIT from
                                              SPARC V8.}
                                          \end{itemize}
                                        \end{minipage}
        \\
        \hline
        13 & 13 & 0,1 & The \textbf{i} bit & \textbf{Set i to ``1''} \\
        14 & 18 & 32 & Source register 1, rs1 & No change \\
        19 & 24 & 100111 & ``\textbf{op3}'' & No change \\
        25 & 29 & 32 & Destination register, rd & No change \\
        30 & 31 & 4 & Always ``10'' & No change \\
        \hline
      \end{tabular}
    \end{center}
    \begin{itemize}
    \item []\textbf{SRAD}: same as SRA, but with Instr[13]=0 (i=0),
      and Instr[5]=1.
    \item []\textbf{Syntax}: ``\texttt{srad SrcReg1, 6BitShiftCnt,
        DestReg}''. \\
      (\textbf{Note:} In an assembly language program, when the second
      argument is a number, we have direct mode.  A register number is
      prefixed with  ``r'', and hence the  syntax itself distinguished
      between   direct  and   register   indirect   version  of   this
      instruction.)
    \item []\textbf{Semantics}: rd(pair) $\leftarrow$ rs1(pair) $>>$
      shift count (with sign extension).
    \end{itemize}
    Bits layout:
\begin{verbatim}
    Offsets      : 31       24 23       16  15        8   7        0
    Bit layout   :  XXXX  XXXX  XXXX  XXXX   XXXX  XXXX   XXXX  XXXX
    Insn Bits    :  10       1  0011  1        1           1        
    Destination  :    DD  DDD                                       
    Source 1     :                     SSS   SS
    Source 2     :                                           S  SSSS
    Unused (0)   :                              U  UUUU   UU        
    Final layout :  10DD  DDD1  0011  1SSS   SS1U  UUUU   U1II  IIII
\end{verbatim}

    This will need another macro that sets bits 5 and 6. Let's call it
    \texttt{OP\_AJIT\_BITS\_5\_AND\_6}.   Hence the  SPARC bit  layout of  this
    instruction is:

    \begin{tabular}[h]{lclcl}
      Macro to set  &=& \texttt{F5(x, y, z)} &in& \texttt{sparc.h}     \\
      Macro to reset  &=& \texttt{INVF5(x, y, z)} &in& \texttt{sparc.h}     \\
      x &=& 0x2      &in& \texttt{OP(x)  /* ((x) \& 0x3)  $<<$ 30 */} \\
      y &=& 0x27     &in& \texttt{OP3(y) /* ((y) \& 0x3f) $<<$ 19 */} \\
      z &=& 0x1      &in& \texttt{F3I(z) /* ((z) \& 0x1)  $<<$ 13 */} \\
      a &=& 0x2      &in& \texttt{OP\_AJIT\_BITS\_5\_AND\_6(a) /* ((a) \& 0x3  $<<$ 6 */}
    \end{tabular}

    The AJIT bits (insn[6:5]) is  set or reset internally by \texttt{F5}
    (just  like  in  \texttt{F4}),  and   hence  there  are  only  three
    arguments.

  \end{enumerate}
\item The register  indirect versions are given by insn[13]  = 0.  The
  shift count is indirectly specified in the 32 bit register specified
  in instruction bits.  Therefore  insn[4:0] specify the register that
  has the  shift count.  insn[6]  = 1, distinguishes the  AJIT version
  from the SPARC V8 version.  In this case, insn[5] = 0, necessarily.
  \begin{enumerate}
  \item \textbf{SLLD}:\\
    \begin{center}
      \begin{tabular}[p]{|c|c|l|l|p{.35\textwidth}|}
        \hline
        \textbf{Start} & \textbf{End} & \textbf{Range} & \textbf{Meaning} &
                                                                            \textbf{New Meaning}\\
        \hline
        0 & 4 & 32 & Source register 2, rs2 & Register number \\
        \hline
        5 & 12 & -- & \textbf{unused} &
                                        \begin{minipage}[h]{1.0\linewidth}
                                          \begin{itemize}
                                          \item \textbf{Set bit 5 to 0.}
                                          \item \textbf{Use bit 6 to
                                              distinguish AJIT from
                                              SPARC V8.}
                                          \end{itemize}
                                        \end{minipage}
        \\
        \hline
        13 & 13 & 0,1 & The \textbf{i} bit & \textbf{Set i to ``0''} \\
        14 & 18 & 32 & Source register 1, rs1 & No change \\
        19 & 24 & 100101 & ``\textbf{op3}'' & No change \\
        25 & 29 & 32 & Destination register, rd & No change \\
        30 & 31 & 4 & Always ``10'' & No change \\
        \hline
      \end{tabular}
    \end{center}
    \begin{itemize}
    \item []\textbf{SLLD}: same as SLL, but with Instr[13]=0 (i=0),
      and Instr[5]=1.
    \item []\textbf{Syntax}: ``\texttt{slld SrcReg1, SrcReg2,
        DestReg}''.
    \item []\textbf{Semantics}: rd(pair) $\leftarrow$ rs1(pair) $<<$
      shift count register rs2.
    \end{itemize}
    Bits layout:
\begin{verbatim}
    Offsets      : 31       24 23       16  15        8   7        0
    Bit layout   :  XXXX  XXXX  XXXX  XXXX   XXXX  XXXX   XXXX  XXXX
    Insn Bits    :  10       1  0010  1        0           10        
    Destination  :    DD  DDD                                       
    Source 1     :                     SSS   SS
    Source 2     :                                           S  SSSS
    Unused (0)   :                              U  UUUU   UU        
    Final layout :  10DD  DDD1  0010  1SSS   SS0U  UUUU   U10I  IIII
\end{verbatim}

    This will need another macro that sets bits 5 and 6. Let's call it
    \texttt{OP\_AJIT\_BITS\_5\_AND\_6}.   Hence the  SPARC bit  layout of  this
    instruction is:

    \begin{tabular}[h]{lclcl}
      Macro to set  &=& \texttt{F5(x, y, z)} &in& \texttt{sparc.h}     \\
      Macro to reset  &=& \texttt{INVF5(x, y, z)} &in& \texttt{sparc.h}     \\
      x &=& 0x2      &in& \texttt{OP(x)  /* ((x) \& 0x3)  $<<$ 30 */} \\
      y &=& 0x25     &in& \texttt{OP3(y) /* ((y) \& 0x3f) $<<$ 19 */} \\
      z &=& 0x0      &in& \texttt{F3I(z) /* ((z) \& 0x1)  $<<$ 13 */} \\
      a &=& 0x2      &in& \texttt{OP\_AJIT\_BITS\_5\_AND\_6(a) /* ((a) \& 0x3  $<<$ 6 */}
    \end{tabular}

    The AJIT bits (insn[6:5]) is  set or reset internally by \texttt{F5}
    (just  like  in  \texttt{F4}),  and   hence  there  are  only  three
    arguments.

  \item \textbf{SRLD}:\\
    \begin{center}
      \begin{tabular}[p]{|c|c|l|l|p{.35\textwidth}|}
        \hline
        \textbf{Start} & \textbf{End} & \textbf{Range} & \textbf{Meaning} &
                                                                            \textbf{New Meaning}\\
        \hline
        0 & 4 & 32 & Source register 2, rs2 & Register number \\
        \hline
        5 & 12 & -- & \textbf{unused} &
                                        \begin{minipage}[h]{1.0\linewidth}
                                          \begin{itemize}
                                          \item \textbf{Set bit 5 to 0.}
                                          \item \textbf{Use bit 6 to
                                              distinguish AJIT from
                                              SPARC V8.}
                                          \end{itemize}
                                        \end{minipage}
        \\
        \hline
        13 & 13 & 0,1 & The \textbf{i} bit & \textbf{Set i to ``0''} \\
        14 & 18 & 32 & Source register 1, rs1 & No change \\
        19 & 24 & 100110 & ``\textbf{op3}'' & No change \\
        25 & 29 & 32 & Destination register, rd & No change \\
        30 & 31 & 4 & Always ``10'' & No change \\
        \hline
      \end{tabular}
    \end{center}
    \begin{itemize}
    \item []\textbf{SRLD}: same as SRL, but with Instr[13]=0 (i=0),
      and Instr[5]=1.
    \item []\textbf{Syntax}: ``\texttt{slld SrcReg1, SrcReg2,
        DestReg}''.
    \item []\textbf{Semantics}: rd(pair) $\leftarrow$ rs1(pair) $>>$
      shift count register rs2.
    \end{itemize}
    Bits layout:
\begin{verbatim}
    Offsets      : 31       24 23       16  15        8   7        0
    Bit layout   :  XXXX  XXXX  XXXX  XXXX   XXXX  XXXX   XXXX  XXXX
    Insn Bits    :  10       1  0011  0        0           10        
    Destination  :    DD  DDD                                       
    Source 1     :                     SSS   SS
    Source 2     :                                           S  SSSS
    Unused (0)   :                              U  UUUU   UU        
    Final layout :  10DD  DDD1  0011  0SSS   SS0U  UUUU   U10I  IIII
\end{verbatim}

    This will need another macro that sets bits 5 and 6. Let's call it
    \texttt{OP\_AJIT\_BITS\_5\_AND\_6}.   Hence the  SPARC bit  layout of  this
    instruction is:

    \begin{tabular}[h]{lclcl}
      Macro to set  &=& \texttt{F5(x, y, z)} &in& \texttt{sparc.h}     \\
      Macro to reset  &=& \texttt{INVF5(x, y, z)} &in& \texttt{sparc.h}     \\
      x &=& 0x2      &in& \texttt{OP(x)  /* ((x) \& 0x3)  $<<$ 30 */} \\
      y &=& 0x26     &in& \texttt{OP3(y) /* ((y) \& 0x3f) $<<$ 19 */} \\
      z &=& 0x0      &in& \texttt{F3I(z) /* ((z) \& 0x1)  $<<$ 13 */} \\
      a &=& 0x2      &in& \texttt{OP\_AJIT\_BITS\_5\_AND\_6(a) /* ((a) \& 0x3  $<<$ 6 */}
    \end{tabular}

    The AJIT bits (insn[6:5]) is  set or reset internally by \texttt{F5}
    (just  like  in  \texttt{F4}),  and   hence  there  are  only  three
    arguments.

  \item \textbf{SRAD}:\\
    \begin{center}
      \begin{tabular}[p]{|c|c|l|l|p{.35\textwidth}|}
        \hline
        \textbf{Start} & \textbf{End} & \textbf{Range} & \textbf{Meaning} &
                                                                            \textbf{New Meaning}\\
        \hline
        0 & 4 & 32 & Source register 2, rs2 & Register number \\
        \hline
        5 & 12 & -- & \textbf{unused} &
                                        \begin{minipage}[h]{1.0\linewidth}
                                          \begin{itemize}
                                          \item \textbf{Set bit 5 to 0.}
                                          \item \textbf{Use bit 6 to
                                              distinguish AJIT from
                                              SPARC V8.}
                                          \end{itemize}
                                        \end{minipage}
        \\
        \hline
        13 & 13 & 0,1 & The \textbf{i} bit & \textbf{Set i to ``0''} \\
        14 & 18 & 32 & Source register 1, rs1 & No change \\
        19 & 24 & 100101 & ``\textbf{op3}'' & No change \\
        25 & 29 & 32 & Destination register, rd & No change \\
        30 & 31 & 4 & Always ``10'' & No change \\
        \hline
      \end{tabular}
    \end{center}
    \begin{itemize}
    \item []\textbf{SRAD}: same as SRA, but with Instr[13]=0 (i=0),
      and Instr[5]=1.
    \item []\textbf{Syntax}: ``\texttt{slld SrcReg1, SrcReg2,
        DestReg}''.
    \item []\textbf{Semantics}: rd(pair) $\leftarrow$ rs1(pair) $>>$
      shift count register rs2 (with sign extension).
    \end{itemize}
    Bits layout:
\begin{verbatim}
    Offsets      : 31       24 23       16  15        8   7        0
    Bit layout   :  XXXX  XXXX  XXXX  XXXX   XXXX  XXXX   XXXX  XXXX
    Insn Bits    :  10       1  0011  1        0           10        
    Destination  :    DD  DDD                                       
    Source 1     :                     SSS   SS
    Source 2     :                                           S  SSSS
    Unused (0)   :                              U  UUUU   UU        
    Final layout :  10DD  DDD1  0011  1SSS   SS0U  UUUU   U10I  IIII
\end{verbatim}

    This will need another macro that sets bits 5 and 6. Let's call it
    \texttt{OP\_AJIT\_BITS\_5\_AND\_6}.   Hence the  SPARC bit  layout of  this
    instruction is:

    \begin{tabular}[h]{lclcl}
      Macro to set  &=& \texttt{F5(x, y, z)} &in& \texttt{sparc.h}     \\
      Macro to reset  &=& \texttt{INVF5(x, y, z)} &in& \texttt{sparc.h}     \\
      x &=& 0x2      &in& \texttt{OP(x)  /* ((x) \& 0x3)  $<<$ 30 */} \\
      y &=& 0x27     &in& \texttt{OP3(y) /* ((y) \& 0x3f) $<<$ 19 */} \\
      z &=& 0x0      &in& \texttt{F3I(z) /* ((z) \& 0x1)  $<<$ 13 */} \\
      a &=& 0x2      &in& \texttt{OP\_AJIT\_BITS\_5\_AND\_6(a) /* ((a) \& 0x3  $<<$ 6 */}
    \end{tabular}

    The AJIT bits (insn[6:5]) is  set or reset internally by \texttt{F5}
    (just  like  in  \texttt{F4}),  and   hence  there  are  only  three
    arguments.
  \end{enumerate}
\end{enumerate}

\subsubsection{Multiplication and division instructions:}
\label{sec:mul:div:insn:impl}
\begin{enumerate}
\item \textbf{UMULD}: Unsigned Integer Multiply AJIT, no immediate
  version (i.e. i is always 0).\\
	\textbf{NOTE:} The \emph{suggested} mnemonic ``umuld'' conflicts with a mnemonic of the same name for another sparc architecture (other than v8).   Hence we change it to: ``\textbf{umuldaj}'' in the implementation, but not in the documentation below.

 This conflict occurs despite forcing the GNU assembler to assemble for v8 only via the command line switch ``-Av8''! It appears that forcing the assembler to use v8 is not universally applied throughout the assembler code. 
  \begin{center}
    \begin{tabular}[p]{|c|c|l|l|l|}
      \hline
      \textbf{Start} & \textbf{End} & \textbf{Range} & \textbf{Meaning} &
                                                                          \textbf{New Meaning}\\
      \hline
      0 & 4 & 32 & Source register 2, rs2 & No change \\
      5 & 12 & -- & \textbf{unused} & \textbf{Set bit 5 to ``1''} \\
      13 & 13 & 0,1 & The \textbf{i} bit & \textbf{Set i to ``0''} \\
      14 & 18 & 32 & Source register 1, rs1 & No change \\
      19 & 24 & 001010 & ``\textbf{op3}'' & No change \\
      25 & 29 & 32 & Destination register, rd & No change \\
      30 & 31 & 4 & Always ``10'' & No change \\
      \hline
    \end{tabular}
  \end{center}
  \begin{itemize}
  \item []\textbf{UMULD}: same as UMUL, but with Instr[13]=0 (i=0), and
    Instr[5]=1.
  \item []\textbf{Syntax}: ``\texttt{umuld  SrcReg1, SrcReg2, DestReg}''.
  \item []\textbf{Semantics}: rd(pair) $\leftarrow$ rs1(pair) * rs2(pair).
  \end{itemize}
  Bits layout:
\begin{verbatim}
    Offsets      : 31       24 23       16  15        8   7        0
    Bit layout   :  XXXX  XXXX  XXXX  XXXX   XXXX  XXXX   XXXX  XXXX
    Insn Bits    :  10       0  0101  0        0            1       
    Destination  :    DD  DDD                                       
    Source 1     :                     SSS   SS
    Source 2     :                                           S  SSSS
    Unused (0)   :                              U  UUUU   UU        
    Final layout :  10DD  DDD0  0101  0SSS   SS0U  UUUU   UU1S  SSSS
\end{verbatim}

  Hence the SPARC bit layout of this instruction is:

  \begin{tabular}[h]{lclcl}
    Macro to set  &=& \texttt{F4(x, y, z)} &in& \texttt{sparc.h}     \\
    Macro to reset  &=& \texttt{INVF4(x, y, z)} &in& \texttt{sparc.h}     \\
    x &=& 0x2      &in& \texttt{OP(x)  /* ((x) \& 0x3)  $<<$ 30 */} \\
    y &=& 0x0A     &in& \texttt{OP3(y) /* ((y) \& 0x3f) $<<$ 19 */} \\
    z &=& 0x0      &in& \texttt{F3I(z) /* ((z) \& 0x1)  $<<$ 13 */} \\
    a &=& 0x1      &in& \texttt{OP\_AJIT\_BIT(a) /* ((a) \& 0x1)  $<<$ 5 */}
  \end{tabular}

  The AJIT bit  (insn[5]) is set internally by  \texttt{F4}, and hence
  there are only three arguments.

\item \textbf{UMULDCC}:\\
  \begin{center}
    \begin{tabular}[p]{|c|c|l|l|l|}
      \hline
      \textbf{Start} & \textbf{End} & \textbf{Range} & \textbf{Meaning} &
                                                                          \textbf{New Meaning}\\
      \hline
      0 & 4 & 32 & Source register 2, rs2 & No change \\
      5 & 12 & -- & \textbf{unused} & \textbf{Set bit 5 to ``1''} \\
      13 & 13 & 0,1 & The \textbf{i} bit & \textbf{Set i to ``0''} \\
      14 & 18 & 32 & Source register 1, rs1 & No change \\
      19 & 24 & 011010 & ``\textbf{op3}'' & No change \\
      25 & 29 & 32 & Destination register, rd & No change \\
      30 & 31 & 4 & Always ``10'' & No change \\
      \hline
    \end{tabular}
  \end{center}
  New addition:
  \begin{itemize}
  \item []\textbf{UMULDCC}: same as UMULCC, but with Instr[13]=0 (i=0), and
    Instr[5]=1.
  \item []\textbf{Syntax}: ``\texttt{umuldcc  SrcReg1, SrcReg2, DestReg}''.
  \item []\textbf{Semantics}: rd(pair) $\leftarrow$ rs1(pair) * rs2(pair), set Z
  \end{itemize}
  Bits layout:
\begin{verbatim}
    Offsets      : 31       24 23       16  15        8   7        0
    Bit layout   :  XXXX  XXXX  XXXX  XXXX   XXXX  XXXX   XXXX  XXXX
    Insn Bits    :  10       0  1101  0        0            1       
    Destination  :    DD  DDD                                       
    Source 1     :                     SSS   SS
    Source 2     :                                           S  SSSS
    Unused (0)   :                              U  UUUU   UU        
    Final layout :  10DD  DDD0  1101  0SSS   SS0U  UUUU   UU1S  SSSS
\end{verbatim}

  Hence the SPARC bit layout of this instruction is:

  \begin{tabular}[h]{lclcl}
    Macro to set  &=& \texttt{F4(x, y, z)} &in& \texttt{sparc.h}     \\
    Macro to reset  &=& \texttt{INVF4(x, y, z)} &in& \texttt{sparc.h}     \\
    x &=& 0x2      &in& \texttt{OP(x)  /* ((x) \& 0x3)  $<<$ 30 */} \\
    y &=& 0x1A     &in& \texttt{OP3(y) /* ((y) \& 0x3f) $<<$ 19 */} \\
    z &=& 0x0      &in& \texttt{F3I(z) /* ((z) \& 0x1)  $<<$ 13 */} \\
    a &=& 0x1      &in& \texttt{OP\_AJIT\_BIT(a) /* ((a) \& 0x1)  $<<$ 5 */}
  \end{tabular}

  The AJIT bit  (insn[5]) is set internally by  \texttt{F4}, and hence
  there are only three arguments.

\item \textbf{SMULD}: Unsigned Integer Multiply AJIT, no immediate
  version (i.e. i is always 0).\\
  \begin{center}
    \begin{tabular}[p]{|c|c|l|l|l|}
      \hline
      \textbf{Start} & \textbf{End} & \textbf{Range} & \textbf{Meaning} &
                                                                          \textbf{New Meaning}\\
      \hline
      0 & 4 & 32 & Source register 2, rs2 & No change \\
      5 & 12 & -- & \textbf{unused} & \textbf{Set bit 5 to ``1''} \\
      13 & 13 & 0,1 & The \textbf{i} bit & \textbf{Set i to ``0''} \\
      14 & 18 & 32 & Source register 1, rs1 & No change \\
      19 & 24 & 001011 & ``\textbf{op3}'' & No change \\
      25 & 29 & 32 & Destination register, rd & No change \\
      30 & 31 & 4 & Always ``10'' & No change \\
      \hline
    \end{tabular}
  \end{center}
  \begin{itemize}
  \item []\textbf{SMULD}: same as SMUL, but with Instr[13]=0 (i=0), and
    Instr[5]=1.
  \item []\textbf{Syntax}: ``\texttt{smuld  SrcReg1, SrcReg2, DestReg}''.
  \item []\textbf{Semantics}: rd(pair) $\leftarrow$ rs1(pair) *
    rs2(pair) (signed).
  \end{itemize}
  Bits layout:
\begin{verbatim}
    Offsets      : 31       24 23       16  15        8   7        0
    Bit layout   :  XXXX  XXXX  XXXX  XXXX   XXXX  XXXX   XXXX  XXXX
    Insn Bits    :  10       0  0101  1        0            1       
    Destination  :    DD  DDD                                       
    Source 1     :                     SSS   SS
    Source 2     :                                           S  SSSS
    Unused (0)   :                              U  UUUU   UU        
    Final layout :  10DD  DDD0  0101  1SSS   SS0U  UUUU   UU1S  SSSS
\end{verbatim}

  Hence the SPARC bit layout of this instruction is:

  \begin{tabular}[h]{lclcl}
    Macro to set  &=& \texttt{F4(x, y, z)} &in& \texttt{sparc.h}     \\
    Macro to reset  &=& \texttt{INVF4(x, y, z)} &in& \texttt{sparc.h}     \\
    x &=& 0x2      &in& \texttt{OP(x)  /* ((x) \& 0x3)  $<<$ 30 */} \\
    y &=& 0x0B     &in& \texttt{OP3(y) /* ((y) \& 0x3f) $<<$ 19 */} \\
    z &=& 0x0      &in& \texttt{F3I(z) /* ((z) \& 0x1)  $<<$ 13 */} \\
    a &=& 0x1      &in& \texttt{OP\_AJIT\_BIT(a) /* ((a) \& 0x1)  $<<$ 5 */}
  \end{tabular}

  The AJIT bit  (insn[5]) is set internally by  \texttt{F4}, and hence
  there are only three arguments.

\item \textbf{SMULDCC}:\\
  \begin{center}
    \begin{tabular}[p]{|c|c|l|l|l|}
      \hline
      \textbf{Start} & \textbf{End} & \textbf{Range} & \textbf{Meaning} &
                                                                          \textbf{New Meaning}\\
      \hline
      0 & 4 & 32 & Source register 2, rs2 & No change \\
      5 & 12 & -- & \textbf{unused} & \textbf{Set bit 5 to ``1''} \\
      13 & 13 & 0,1 & The \textbf{i} bit & \textbf{Set i to ``0''} \\
      14 & 18 & 32 & Source register 1, rs1 & No change \\
      19 & 24 & 011011 & ``\textbf{op3}'' & No change \\
      25 & 29 & 32 & Destination register, rd & No change \\
      30 & 31 & 4 & Always ``10'' & No change \\
      \hline
    \end{tabular}
  \end{center}
  New addition:
  \begin{itemize}
  \item []\textbf{SMULDCC}: same as SMULCC, but with Instr[13]=0 (i=0), and
    Instr[5]=1.
  \item []\textbf{Syntax}: ``\texttt{smuldcc  SrcReg1, SrcReg2, DestReg}''.
  \item []\textbf{Semantics}: rd(pair) $\leftarrow$ rs1(pair) *
    rs2(pair) (signed), set Z,N,O
  \end{itemize}
  Bits layout:
\begin{verbatim}
    Offsets      : 31       24 23       16  15        8   7        0
    Bit layout   :  XXXX  XXXX  XXXX  XXXX   XXXX  XXXX   XXXX  XXXX
    Insn Bits    :  10       0  1101  1        0            1       
    Destination  :    DD  DDD                                       
    Source 1     :                     SSS   SS
    Source 2     :                                           S  SSSS
    Unused (0)   :                              U  UUUU   UU        
    Final layout :  10DD  DDD0  1101  1SSS   SS0U  UUUU   UU1S  SSSS
\end{verbatim}

  Hence the SPARC bit layout of this instruction is:

  \begin{tabular}[h]{lclcl}
    Macro to set  &=& \texttt{F4(x, y, z)} &in& \texttt{sparc.h}     \\
    Macro to reset  &=& \texttt{INVF4(x, y, z)} &in& \texttt{sparc.h}     \\
    x &=& 0x2      &in& \texttt{OP(x)  /* ((x) \& 0x3)  $<<$ 30 */} \\
    y &=& 0x1B     &in& \texttt{OP3(y) /* ((y) \& 0x3f) $<<$ 19 */} \\
    z &=& 0x0      &in& \texttt{F3I(z) /* ((z) \& 0x1)  $<<$ 13 */} \\
    a &=& 0x1      &in& \texttt{OP\_AJIT\_BIT(a) /* ((a) \& 0x1)  $<<$ 5 */}
  \end{tabular}

  The AJIT bit  (insn[5]) is set internally by  \texttt{F4}, and hence
  there are only three arguments.

\item \textbf{UDIVD}:\\
  \begin{center}
    \begin{tabular}[p]{|c|c|l|l|l|}
      \hline
      \textbf{Start} & \textbf{End} & \textbf{Range} & \textbf{Meaning} &
                                                                          \textbf{New Meaning}\\
      \hline
      0 & 4 & 32 & Source register 2, rs2 & No change \\
      5 & 12 & -- & \textbf{unused} & \textbf{Set bit 5 to ``1''} \\
      13 & 13 & 0,1 & The \textbf{i} bit & \textbf{Set i to ``0''} \\
      14 & 18 & 32 & Source register 1, rs1 & No change \\
      19 & 24 & 001110 & ``\textbf{op3}'' & No change \\
      25 & 29 & 32 & Destination register, rd & No change \\
      30 & 31 & 4 & Always ``10'' & No change \\
      \hline
    \end{tabular}
  \end{center}
  New addition:
  \begin{itemize}
  \item []\textbf{UDIVD}: same as UDIV, but with Instr[13]=0 (i=0), and
    Instr[5]=1.
  \item []\textbf{Syntax}: ``\texttt{udivd  SrcReg1, SrcReg2, DestReg}''.
  \item []\textbf{Semantics}: rd(pair) $\leftarrow$ rs1(pair) / rs2(pair).
  \end{itemize}
  Bits layout:
\begin{verbatim}
    Offsets      : 31       24 23       16  15        8   7        0
    Bit layout   :  XXXX  XXXX  XXXX  XXXX   XXXX  XXXX   XXXX  XXXX
    Insn Bits    :  10       0  0111  0        0            1       
    Destination  :    DD  DDD                                       
    Source 1     :                     SSS   SS
    Source 2     :                                           S  SSSS
    Unused (0)   :                              U  UUUU   UU        
    Final layout :  10DD  DDD0  0111  0SSS   SS0U  UUUU   UU1S  SSSS
\end{verbatim}

  Hence the SPARC bit layout of this instruction is:

  \begin{tabular}[h]{lclcl}
    Macro to set  &=& \texttt{F4(x, y, z)} &in& \texttt{sparc.h}     \\
    Macro to reset  &=& \texttt{INVF4(x, y, z)} &in& \texttt{sparc.h}     \\
    x &=& 0x2      &in& \texttt{OP(x)  /* ((x) \& 0x3)  $<<$ 30 */} \\
    y &=& 0x0E     &in& \texttt{OP3(y) /* ((y) \& 0x3f) $<<$ 19 */} \\
    z &=& 0x0      &in& \texttt{F3I(z) /* ((z) \& 0x1)  $<<$ 13 */} \\
    a &=& 0x1      &in& \texttt{OP\_AJIT\_BIT(a) /* ((a) \& 0x1)  $<<$ 5 */}
  \end{tabular}

  The AJIT bit  (insn[5]) is set internally by  \texttt{F4}, and hence
  there are only three arguments.

\item \textbf{UDIVDCC}:\\
  \begin{center}
    \begin{tabular}[p]{|c|c|l|l|l|}
      \hline
      \textbf{Start} & \textbf{End} & \textbf{Range} & \textbf{Meaning} &
                                                                          \textbf{New Meaning}\\
      \hline
      0 & 4 & 32 & Source register 2, rs2 & No change \\
      5 & 12 & -- & \textbf{unused} & \textbf{Set bit 5 to ``1''} \\
      13 & 13 & 0,1 & The \textbf{i} bit & \textbf{Set i to ``0''} \\
      14 & 18 & 32 & Source register 1, rs1 & No change \\
      19 & 24 & 011110 & ``\textbf{op3}'' & No change \\
      25 & 29 & 32 & Destination register, rd & No change \\
      30 & 31 & 4 & Always ``10'' & No change \\
      \hline
    \end{tabular}
  \end{center}
  New addition:
  \begin{itemize}
  \item []\textbf{UDIVDCC}: same as UDIVCC, but with Instr[13]=0 (i=0), and
    Instr[5]=1.
  \item []\textbf{Syntax}: ``\texttt{udivdcc  SrcReg1, SrcReg2, DestReg}''.
  \item []\textbf{Semantics}: rd(pair) $\leftarrow$ rs1(pair) / rs2(pair), set Z,O
  \end{itemize}
  Bits layout:
\begin{verbatim}
    Offsets      : 31       24 23       16  15        8   7        0
    Bit layout   :  XXXX  XXXX  XXXX  XXXX   XXXX  XXXX   XXXX  XXXX
    Insn Bits    :  10       0  1111  0        0            1       
    Destination  :    DD  DDD                                       
    Source 1     :                     SSS   SS
    Source 2     :                                           S  SSSS
    Unused (0)   :                              U  UUUU   UU        
    Final layout :  10DD  DDD0  1111  0SSS   SS0U  UUUU   UU1S  SSSS
\end{verbatim}

  Hence the SPARC bit layout of this instruction is:

  \begin{tabular}[h]{lclcl}
    Macro to set  &=& \texttt{F4(x, y, z)} &in& \texttt{sparc.h}     \\
    Macro to reset  &=& \texttt{INVF4(x, y, z)} &in& \texttt{sparc.h}     \\
    x &=& 0x2      &in& \texttt{OP(x)  /* ((x) \& 0x3)  $<<$ 30 */} \\
    y &=& 0x1E     &in& \texttt{OP3(y) /* ((y) \& 0x3f) $<<$ 19 */} \\
    z &=& 0x0      &in& \texttt{F3I(z) /* ((z) \& 0x1)  $<<$ 13 */} \\
    a &=& 0x1      &in& \texttt{OP\_AJIT\_BIT(a) /* ((a) \& 0x1)  $<<$ 5 */}
  \end{tabular}

  The AJIT bit  (insn[5]) is set internally by  \texttt{F4}, and hence
  there are only three arguments.

\item \textbf{SDIVD}:\\
  \begin{center}
    \begin{tabular}[p]{|c|c|l|l|l|}
      \hline
      \textbf{Start} & \textbf{End} & \textbf{Range} & \textbf{Meaning} &
                                                                          \textbf{New Meaning}\\
      \hline
      0 & 4 & 32 & Source register 2, rs2 & No change \\
      5 & 12 & -- & \textbf{unused} & \textbf{Set bit 5 to ``1''} \\
      13 & 13 & 0,1 & The \textbf{i} bit & \textbf{Set i to ``0''} \\
      14 & 18 & 32 & Source register 1, rs1 & No change \\
      19 & 24 & 001111 & ``\textbf{op3}'' & No change \\
      25 & 29 & 32 & Destination register, rd & No change \\
      30 & 31 & 4 & Always ``10'' & No change \\
      \hline
    \end{tabular}
  \end{center}
  New addition:
  \begin{itemize}
  \item []\textbf{SDIVD}: same as SDIV, but with Instr[13]=0 (i=0), and
    Instr[5]=1.
  \item []\textbf{Syntax}: ``\texttt{sdivd  SrcReg1, SrcReg2, DestReg}''.
  \item []\textbf{Semantics}: rd(pair) $\leftarrow$ rs1(pair) /
    rs2(pair) (signed).
  \end{itemize}
  Bits layout:
\begin{verbatim}
    Offsets      : 31       24 23       16  15        8   7        0
    Bit layout   :  XXXX  XXXX  XXXX  XXXX   XXXX  XXXX   XXXX  XXXX
    Insn Bits    :  10       0  0111  1        0            1       
    Destination  :    DD  DDD                                       
    Source 1     :                     SSS   SS
    Source 2     :                                           S  SSSS
    Unused (0)   :                              U  UUUU   UU        
    Final layout :  10DD  DDD0  0111  1SSS   SS0U  UUUU   UU1S  SSSS
\end{verbatim}

  Hence the SPARC bit layout of this instruction is:

  \begin{tabular}[h]{lclcl}
    Macro to set  &=& \texttt{F4(x, y, z)} &in& \texttt{sparc.h}     \\
    Macro to reset  &=& \texttt{INVF4(x, y, z)} &in& \texttt{sparc.h}     \\
    x &=& 0x2      &in& \texttt{OP(x)  /* ((x) \& 0x3)  $<<$ 30 */} \\
    y &=& 0x0F     &in& \texttt{OP3(y) /* ((y) \& 0x3f) $<<$ 19 */} \\
    z &=& 0x0      &in& \texttt{F3I(z) /* ((z) \& 0x1)  $<<$ 13 */} \\
    a &=& 0x1      &in& \texttt{OP\_AJIT\_BIT(a) /* ((a) \& 0x1)  $<<$ 5 */}
  \end{tabular}

  The AJIT bit  (insn[5]) is set internally by  \texttt{F4}, and hence
  there are only three arguments.

\item \textbf{SDIVDCC}:\\
  \begin{center}
    \begin{tabular}[p]{|c|c|l|l|l|}
      \hline
      \textbf{Start} & \textbf{End} & \textbf{Range} & \textbf{Meaning} &
                                                                          \textbf{New Meaning}\\
      \hline
      0 & 4 & 32 & Source register 2, rs2 & No change \\
      5 & 12 & -- & \textbf{unused} & \textbf{Set bit 5 to ``1''} \\
      13 & 13 & 0,1 & The \textbf{i} bit & \textbf{Set i to ``0''} \\
      14 & 18 & 32 & Source register 1, rs1 & No change \\
      19 & 24 & 011111 & ``\textbf{op3}'' & No change \\
      25 & 29 & 32 & Destination register, rd & No change \\
      30 & 31 & 4 & Always ``10'' & No change \\
      \hline
    \end{tabular}
  \end{center}
  New addition:
  \begin{itemize}
  \item []\textbf{SDIVDCC}: same as SDIVCC, but with Instr[13]=0 (i=0), and
    Instr[5]=1.
  \item []\textbf{Syntax}: ``\texttt{sdivdcc  SrcReg1, SrcReg2, DestReg}''.
  \item []\textbf{Semantics}: rd(pair) $\leftarrow$ rs1(pair) /
    rs2(pair) (signed), set Z,N,O
  \end{itemize}
  Bits layout:
\begin{verbatim}
    Offsets      : 31       24 23       16  15        8   7        0
    Bit layout   :  XXXX  XXXX  XXXX  XXXX   XXXX  XXXX   XXXX  XXXX
    Insn Bits    :  10       0  1111  1        0            1       
    Destination  :    DD  DDD                                       
    Source 1     :                     SSS   SS
    Source 2     :                                           S  SSSS
    Unused (0)   :                              U  UUUU   UU        
    Final layout :  10DD  DDD0  1111  1SSS   SS0U  UUUU   UU1S  SSSS
\end{verbatim}

  Hence the SPARC bit layout of this instruction is:

  \begin{tabular}[h]{lclcl}
    Macro to set  &=& \texttt{F4(x, y, z)} &in& \texttt{sparc.h}     \\
    Macro to reset  &=& \texttt{INVF4(x, y, z)} &in& \texttt{sparc.h}     \\
    x &=& 0x2      &in& \texttt{OP(x)  /* ((x) \& 0x3)  $<<$ 30 */} \\
    y &=& 0x1F     &in& \texttt{OP3(y) /* ((y) \& 0x3f) $<<$ 19 */} \\
    z &=& 0x0      &in& \texttt{F3I(z) /* ((z) \& 0x1)  $<<$ 13 */} \\
    a &=& 0x1      &in& \texttt{OP\_AJIT\_BIT(a) /* ((a) \& 0x1)  $<<$ 5 */}
  \end{tabular}

  The AJIT bit  (insn[5]) is set internally by  \texttt{F4}, and hence
  there are only three arguments.
\end{enumerate}

%%% Local Variables:
%%% mode: latex
%%% TeX-master: t
%%% End:

\subsubsection{64 Bit Logical Instructions:}
\label{sec:64:bit:logical:insn:impl}

No immediate mode, i.e. insn[5] $\equiv$ i = 0, always.

\begin{enumerate}
\item \textbf{ORD}:\\
  \begin{center}
    \begin{tabular}[p]{|c|c|l|l|l|}
      \hline
      \textbf{Start} & \textbf{End} & \textbf{Range} & \textbf{Meaning} &
                                                                          \textbf{New Meaning}\\
      \hline
      0 & 4 & 32 & Source register 2, rs2 & No change \\
      5 & 12 & -- & \textbf{unused} & \textbf{Set bit 5 to ``1''} \\
      13 & 13 & 0,1 & The \textbf{i} bit & \textbf{Set i to ``0''} \\
      14 & 18 & 32 & Source register 1, rs1 & No change \\
      19 & 24 & 000010 & ``\textbf{op3}'' & No change \\
      25 & 29 & 32 & Destination register, rd & No change \\
      30 & 31 & 4 & Always ``10'' & No change \\
      \hline
    \end{tabular}
  \end{center}
  \begin{itemize}
  \item []\textbf{ORD}: same as OR, but with Instr[13]=0 (i=0), and
    Instr[5]=1.
  \item []\textbf{Syntax}: ``\texttt{ord  SrcReg1, SrcReg2, DestReg}''.
  \item []\textbf{Semantics}: rd(pair) $\leftarrow$ rs1(pair) $\vert$ rs2(pair).
  \end{itemize}
  Bits layout:
\begin{verbatim}
    Offsets      : 31       24 23       16  15        8   7        0
    Bit layout   :  XXXX  XXXX  XXXX  XXXX   XXXX  XXXX   XXXX  XXXX
    Insn Bits    :  10       0  0001  0        0            1       
    Destination  :    DD  DDD                                       
    Source 1     :                     SSS   SS
    Source 2     :                                           S  SSSS
    Unused (0)   :                              U  UUUU   UU        
    Final layout :  10DD  DDD0  0001  0SSS   SS0U  UUUU   UU1S  SSSS
\end{verbatim}

  Hence the SPARC bit layout of this instruction is:

  \begin{tabular}[h]{lclcl}
    Macro to set  &=& \texttt{F4(x, y, z)} &in& \texttt{sparc.h}     \\
    Macro to reset  &=& \texttt{INVF4(x, y, z)} &in& \texttt{sparc.h}     \\
    x &=& 0x2      &in& \texttt{OP(x)  /* ((x) \& 0x3)  $<<$ 30 */} \\
    y &=& 0x02     &in& \texttt{OP3(y) /* ((y) \& 0x3f) $<<$ 19 */} \\
    z &=& 0x0      &in& \texttt{F3I(z) /* ((z) \& 0x1)  $<<$ 13 */} \\
    a &=& 0x1      &in& \texttt{OP\_AJIT\_BIT(a) /* ((a) \& 0x1)  $<<$ 5 */}
  \end{tabular}

  The AJIT bit  (insn[5]) is set internally by  \texttt{F4}, and hence
  there are only three arguments.

\item \textbf{ORDCC}:\\
  \begin{center}
    \begin{tabular}[p]{|c|c|l|l|l|}
      \hline
      \textbf{Start} & \textbf{End} & \textbf{Range} & \textbf{Meaning} &
                                                                          \textbf{New Meaning}\\
      \hline
      0 & 4 & 32 & Source register 2, rs2 & No change \\
      5 & 12 & -- & \textbf{unused} & \textbf{Set bit 5 to ``1''} \\
      13 & 13 & 0,1 & The \textbf{i} bit & \textbf{Set i to ``0''} \\
      14 & 18 & 32 & Source register 1, rs1 & No change \\
      19 & 24 & 010010 & ``\textbf{op3}'' & No change \\
      25 & 29 & 32 & Destination register, rd & No change \\
      30 & 31 & 4 & Always ``10'' & No change \\
      \hline
    \end{tabular}
  \end{center}
  \begin{itemize}
  \item []\textbf{ORDCC}: same as ORCC, but with Instr[13]=0 (i=0), and
    Instr[5]=1.
  \item []\textbf{Syntax}: ``\texttt{ordcc  SrcReg1, SrcReg2, DestReg}''.
  \item []\textbf{Semantics}: rd(pair) $\leftarrow$ rs1(pair) $\vert$
    rs2(pair), sets Z.
  \end{itemize}
  Bits layout:
\begin{verbatim}
    Offsets      : 31       24 23       16  15        8   7        0
    Bit layout   :  XXXX  XXXX  XXXX  XXXX   XXXX  XXXX   XXXX  XXXX
    Insn Bits    :  10       0  1001  0        0            1       
    Destination  :    DD  DDD                                       
    Source 1     :                     SSS   SS
    Source 2     :                                           S  SSSS
    Unused (0)   :                              U  UUUU   UU        
    Final layout :  10DD  DDD0  1001  0SSS   SS0U  UUUU   UU1S  SSSS
\end{verbatim}

  Hence the SPARC bit layout of this instruction is:

  \begin{tabular}[h]{lclcl}
    Macro to set  &=& \texttt{F4(x, y, z)} &in& \texttt{sparc.h}     \\
    Macro to reset  &=& \texttt{INVF4(x, y, z)} &in& \texttt{sparc.h}     \\
    x &=& 0x2      &in& \texttt{OP(x)  /* ((x) \& 0x3)  $<<$ 30 */} \\
    y &=& 0x12     &in& \texttt{OP3(y) /* ((y) \& 0x3f) $<<$ 19 */} \\
    z &=& 0x0      &in& \texttt{F3I(z) /* ((z) \& 0x1)  $<<$ 13 */} \\
    a &=& 0x1      &in& \texttt{OP\_AJIT\_BIT(a) /* ((a) \& 0x1)  $<<$ 5 */}
  \end{tabular}

  The AJIT bit  (insn[5]) is set internally by  \texttt{F4}, and hence
  there are only three arguments.

\item \textbf{ORDN}:\\
  \begin{center}
    \begin{tabular}[p]{|c|c|l|l|l|}
      \hline
      \textbf{Start} & \textbf{End} & \textbf{Range} & \textbf{Meaning} &
                                                                          \textbf{New Meaning}\\
      \hline
      0 & 4 & 32 & Source register 2, rs2 & No change \\
      5 & 12 & -- & \textbf{unused} & \textbf{Set bit 5 to ``1''} \\
      13 & 13 & 0,1 & The \textbf{i} bit & \textbf{Set i to ``0''} \\
      14 & 18 & 32 & Source register 1, rs1 & No change \\
      19 & 24 & 000110 & ``\textbf{op3}'' & No change \\
      25 & 29 & 32 & Destination register, rd & No change \\
      30 & 31 & 4 & Always ``10'' & No change \\
      \hline
    \end{tabular}
  \end{center}
  \begin{itemize}
  \item []\textbf{ORDN}: same as ORN, but with Instr[13]=0 (i=0), and
    Instr[5]=1.
  \item []\textbf{Syntax}: ``\texttt{ordn  SrcReg1, SrcReg2, DestReg}''.
  \item []\textbf{Semantics}: rd(pair) $\leftarrow$ rs1(pair) $\vert$ ($\sim$rs2(pair)).
  \end{itemize}
  Bits layout:
\begin{verbatim}
    Offsets      : 31       24 23       16  15        8   7        0
    Bit layout   :  XXXX  XXXX  XXXX  XXXX   XXXX  XXXX   XXXX  XXXX
    Insn Bits    :  10       0  0011  0        0            1       
    Destination  :    DD  DDD                                       
    Source 1     :                     SSS   SS
    Source 2     :                                           S  SSSS
    Unused (0)   :                              U  UUUU   UU        
    Final layout :  10DD  DDD0  0011  0SSS   SS0U  UUUU   UU1S  SSSS
\end{verbatim}

  Hence the SPARC bit layout of this instruction is:

  \begin{tabular}[h]{lclcl}
    Macro to set  &=& \texttt{F4(x, y, z)} &in& \texttt{sparc.h}     \\
    Macro to reset  &=& \texttt{INVF4(x, y, z)} &in& \texttt{sparc.h}     \\
    x &=& 0x2      &in& \texttt{OP(x)  /* ((x) \& 0x3)  $<<$ 30 */} \\
    y &=& 0x06     &in& \texttt{OP3(y) /* ((y) \& 0x3f) $<<$ 19 */} \\
    z &=& 0x0      &in& \texttt{F3I(z) /* ((z) \& 0x1)  $<<$ 13 */} \\
    a &=& 0x1      &in& \texttt{OP\_AJIT\_BIT(a) /* ((a) \& 0x1)  $<<$ 5 */}
  \end{tabular}

  The AJIT bit  (insn[5]) is set internally by  \texttt{F4}, and hence
  there are only three arguments.

\item \textbf{ORDNCC}:\\
  \begin{center}
    \begin{tabular}[p]{|c|c|l|l|l|}
      \hline
      \textbf{Start} & \textbf{End} & \textbf{Range} & \textbf{Meaning} &
                                                                          \textbf{New Meaning}\\
      \hline
      0 & 4 & 32 & Source register 2, rs2 & No change \\
      5 & 12 & -- & \textbf{unused} & \textbf{Set bit 5 to ``1''} \\
      13 & 13 & 0,1 & The \textbf{i} bit & \textbf{Set i to ``0''} \\
      14 & 18 & 32 & Source register 1, rs1 & No change \\
      19 & 24 & 010110 & ``\textbf{op3}'' & No change \\
      25 & 29 & 32 & Destination register, rd & No change \\
      30 & 31 & 4 & Always ``10'' & No change \\
      \hline
    \end{tabular}
  \end{center}
  \begin{itemize}
  \item []\textbf{ORDNCC}: same as ORN, but with Instr[13]=0 (i=0), and
    Instr[5]=1.
  \item []\textbf{Syntax}: ``\texttt{ordncc  SrcReg1, SrcReg2, DestReg}''.
  \item []\textbf{Semantics}: rd(pair) $\leftarrow$ rs1(pair) $\vert$
    ($\sim$rs2(pair)), sets Z.
  \end{itemize}
  Bits layout:
\begin{verbatim}
    Offsets      : 31       24 23       16  15        8   7        0
    Bit layout   :  XXXX  XXXX  XXXX  XXXX   XXXX  XXXX   XXXX  XXXX
    Insn Bits    :  10       0  1011  0        0            1       
    Destination  :    DD  DDD                                       
    Source 1     :                     SSS   SS
    Source 2     :                                           S  SSSS
    Unused (0)   :                              U  UUUU   UU        
    Final layout :  10DD  DDD0  0011  0SSS   SS0U  UUUU   UU1S  SSSS
\end{verbatim}

  Hence the SPARC bit layout of this instruction is:

  \begin{tabular}[h]{lclcl}
    Macro to set  &=& \texttt{F4(x, y, z)} &in& \texttt{sparc.h}     \\
    Macro to reset  &=& \texttt{INVF4(x, y, z)} &in& \texttt{sparc.h}     \\
    x &=& 0x2      &in& \texttt{OP(x)  /* ((x) \& 0x3)  $<<$ 30 */} \\
    y &=& 0x16     &in& \texttt{OP3(y) /* ((y) \& 0x3f) $<<$ 19 */} \\
    z &=& 0x0      &in& \texttt{F3I(z) /* ((z) \& 0x1)  $<<$ 13 */} \\
    a &=& 0x1      &in& \texttt{OP\_AJIT\_BIT(a) /* ((a) \& 0x1)  $<<$ 5 */}
  \end{tabular}

  The AJIT bit  (insn[5]) is set internally by  \texttt{F4}, and hence
  there are only three arguments.

\item \textbf{XORDCC}:\\
  \begin{center}
    \begin{tabular}[p]{|c|c|l|l|l|}
      \hline
      \textbf{Start} & \textbf{End} & \textbf{Range} & \textbf{Meaning} &
                                                                          \textbf{New Meaning}\\
      \hline
      0 & 4 & 32 & Source register 2, rs2 & No change \\
      5 & 12 & -- & \textbf{unused} & \textbf{Set bit 5 to ``1''} \\
      13 & 13 & 0,1 & The \textbf{i} bit & \textbf{Set i to ``0''} \\
      14 & 18 & 32 & Source register 1, rs1 & No change \\
      19 & 24 & 010011 & ``\textbf{op3}'' & No change \\
      25 & 29 & 32 & Destination register, rd & No change \\
      30 & 31 & 4 & Always ``10'' & No change \\
      \hline
    \end{tabular}
  \end{center}
  \begin{itemize}
  \item []\textbf{XORDCC}: same as XORCC, but with Instr[13]=0 (i=0), and
    Instr[5]=1.
  \item []\textbf{Syntax}: ``\texttt{xordcc  SrcReg1, SrcReg2, DestReg}''.
  \item []\textbf{Semantics}: rd(pair) $\leftarrow$ rs1(pair) $\hat{~}$
    rs2(pair), sets Z.
  \end{itemize}
  Bits layout:
\begin{verbatim}
    Offsets      : 31       24 23       16  15        8   7        0
    Bit layout   :  XXXX  XXXX  XXXX  XXXX   XXXX  XXXX   XXXX  XXXX
    Insn Bits    :  10       0  1001  1        0            1       
    Destination  :    DD  DDD                                       
    Source 1     :                     SSS   SS
    Source 2     :                                           S  SSSS
    Unused (0)   :                              U  UUUU   UU        
    Final layout :  10DD  DDD0  1001  1SSS   SS0U  UUUU   UU1S  SSSS
\end{verbatim}

  Hence the SPARC bit layout of this instruction is:

  \begin{tabular}[h]{lclcl}
    Macro to set  &=& \texttt{F4(x, y, z)} &in& \texttt{sparc.h}     \\
    Macro to reset  &=& \texttt{INVF4(x, y, z)} &in& \texttt{sparc.h}     \\
    x &=& 0x2      &in& \texttt{OP(x)  /* ((x) \& 0x3)  $<<$ 30 */} \\
    y &=& 0x13     &in& \texttt{OP3(y) /* ((y) \& 0x3f) $<<$ 19 */} \\
    z &=& 0x0      &in& \texttt{F3I(z) /* ((z) \& 0x1)  $<<$ 13 */} \\
    a &=& 0x1      &in& \texttt{OP\_AJIT\_BIT(a) /* ((a) \& 0x1)  $<<$ 5 */}
  \end{tabular}

  The AJIT bit  (insn[5]) is set internally by  \texttt{F4}, and hence
  there are only three arguments.

\item \textbf{XNORD}:\\
  \begin{center}
    \begin{tabular}[p]{|c|c|l|l|l|}
      \hline
      \textbf{Start} & \textbf{End} & \textbf{Range} & \textbf{Meaning} &
                                                                          \textbf{New Meaning}\\
      \hline
      0 & 4 & 32 & Source register 2, rs2 & No change \\
      5 & 12 & -- & \textbf{unused} & \textbf{Set bit 5 to ``1''} \\
      13 & 13 & 0,1 & The \textbf{i} bit & \textbf{Set i to ``0''} \\
      14 & 18 & 32 & Source register 1, rs1 & No change \\
      19 & 24 & 000111 & ``\textbf{op3}'' & No change \\
      25 & 29 & 32 & Destination register, rd & No change \\
      30 & 31 & 4 & Always ``10'' & No change \\
      \hline
    \end{tabular}
  \end{center}
  \begin{itemize}
  \item []\textbf{XNORD}: same as XNOR, but with Instr[13]=0 (i=0), and
    Instr[5]=1.
  \item []\textbf{Syntax}: ``\texttt{xnordcc  SrcReg1, SrcReg2, DestReg}''.
  \item []\textbf{Semantics}: rd(pair) $\leftarrow$ rs1(pair) $\hat{~}$
    rs2(pair).
  \end{itemize}
  Bits layout:
\begin{verbatim}
    Offsets      : 31       24 23       16  15        8   7        0
    Bit layout   :  XXXX  XXXX  XXXX  XXXX   XXXX  XXXX   XXXX  XXXX
    Insn Bits    :  10       0  0011  1        0            1       
    Destination  :    DD  DDD                                       
    Source 1     :                     SSS   SS
    Source 2     :                                           S  SSSS
    Unused (0)   :                              U  UUUU   UU        
    Final layout :  10DD  DDD0  0011  1SSS   SS0U  UUUU   UU1S  SSSS
\end{verbatim}

  Hence the SPARC bit layout of this instruction is:

  \begin{tabular}[h]{lclcl}
    Macro to set  &=& \texttt{F4(x, y, z)} &in& \texttt{sparc.h}     \\
    Macro to reset  &=& \texttt{INVF4(x, y, z)} &in& \texttt{sparc.h}     \\
    x &=& 0x2      &in& \texttt{OP(x)  /* ((x) \& 0x3)  $<<$ 30 */} \\
    y &=& 0x07     &in& \texttt{OP3(y) /* ((y) \& 0x3f) $<<$ 19 */} \\
    z &=& 0x0      &in& \texttt{F3I(z) /* ((z) \& 0x1)  $<<$ 13 */} \\
    a &=& 0x1      &in& \texttt{OP\_AJIT\_BIT(a) /* ((a) \& 0x1)  $<<$ 5 */}
  \end{tabular}

  The AJIT bit  (insn[5]) is set internally by  \texttt{F4}, and hence
  there are only three arguments.
  
\item \textbf{XNORDCC}:\\
  \begin{center}
    \begin{tabular}[p]{|c|c|l|l|l|}
      \hline
      \textbf{Start} & \textbf{End} & \textbf{Range} & \textbf{Meaning} &
                                                                          \textbf{New Meaning}\\
      \hline
      0 & 4 & 32 & Source register 2, rs2 & No change \\
      5 & 12 & -- & \textbf{unused} & \textbf{Set bit 5 to ``1''} \\
      13 & 13 & 0,1 & The \textbf{i} bit & \textbf{Set i to ``0''} \\
      14 & 18 & 32 & Source register 1, rs1 & No change \\
      19 & 24 & 000111 & ``\textbf{op3}'' & No change \\
      25 & 29 & 32 & Destination register, rd & No change \\
      30 & 31 & 4 & Always ``10'' & No change \\
      \hline
    \end{tabular}
  \end{center}
  \begin{itemize}
  \item []\textbf{XNORDCC}: same as XNORD, but with Instr[13]=0 (i=0), and
    Instr[5]=1.
  \item []\textbf{Syntax}: ``\texttt{xnordcc  SrcReg1, SrcReg2, DestReg}''.
  \item []\textbf{Semantics}: rd(pair) $\leftarrow$ rs1(pair) $\hat{~}$
    rs2(pair), sets Z.
  \end{itemize}
  Bits layout:
\begin{verbatim}
    Offsets      : 31       24 23       16  15        8   7        0
    Bit layout   :  XXXX  XXXX  XXXX  XXXX   XXXX  XXXX   XXXX  XXXX
    Insn Bits    :  10       0  0011  1        0            1       
    Destination  :    DD  DDD                                       
    Source 1     :                     SSS   SS
    Source 2     :                                           S  SSSS
    Unused (0)   :                              U  UUUU   UU        
    Final layout :  10DD  DDD0  0011  1SSS   SS0U  UUUU   UU1S  SSSS
\end{verbatim}

  Hence the SPARC bit layout of this instruction is:

  \begin{tabular}[h]{lclcl}
    Macro to set  &=& \texttt{F4(x, y, z)} &in& \texttt{sparc.h}     \\
    Macro to reset  &=& \texttt{INVF4(x, y, z)} &in& \texttt{sparc.h}     \\
    x &=& 0x2      &in& \texttt{OP(x)  /* ((x) \& 0x3)  $<<$ 30 */} \\
    y &=& 0x07     &in& \texttt{OP3(y) /* ((y) \& 0x3f) $<<$ 19 */} \\
    z &=& 0x0      &in& \texttt{F3I(z) /* ((z) \& 0x1)  $<<$ 13 */} \\
    a &=& 0x1      &in& \texttt{OP\_AJIT\_BIT(a) /* ((a) \& 0x1)  $<<$ 5 */}
  \end{tabular}

  The AJIT bit  (insn[5]) is set internally by  \texttt{F4}, and hence
  there are only three arguments.
  
\item \textbf{ANDD}:\\
  \begin{center}
    \begin{tabular}[p]{|c|c|l|l|l|}
      \hline
      \textbf{Start} & \textbf{End} & \textbf{Range} & \textbf{Meaning} &
                                                                          \textbf{New Meaning}\\
      \hline
      0 & 4 & 32 & Source register 2, rs2 & No change \\
      5 & 12 & -- & \textbf{unused} & \textbf{Set bit 5 to ``1''} \\
      13 & 13 & 0,1 & The \textbf{i} bit & \textbf{Set i to ``0''} \\
      14 & 18 & 32 & Source register 1, rs1 & No change \\
      19 & 24 & 000001 & ``\textbf{op3}'' & No change \\
      25 & 29 & 32 & Destination register, rd & No change \\
      30 & 31 & 4 & Always ``10'' & No change \\
      \hline
    \end{tabular}
  \end{center}
  \begin{itemize}
  \item []\textbf{ANDD}: same as AND, but with Instr[13]=0 (i=0), and
    Instr[5]=1.
  \item []\textbf{Syntax}: ``\texttt{andd  SrcReg1, SrcReg2, DestReg}''.
  \item []\textbf{Semantics}: rd(pair) $\leftarrow$ rs1(pair) $\cdot$ rs2(pair).
  \end{itemize}
  Bits layout:
\begin{verbatim}
    Offsets      : 31       24 23       16  15        8   7        0
    Bit layout   :  XXXX  XXXX  XXXX  XXXX   XXXX  XXXX   XXXX  XXXX
    Insn Bits    :  10       0  0000  1        0            1       
    Destination  :    DD  DDD                                       
    Source 1     :                     SSS   SS
    Source 2     :                                           S  SSSS
    Unused (0)   :                              U  UUUU   UU        
    Final layout :  10DD  DDD0  0000  1SSS   SS0U  UUUU   UU1S  SSSS
\end{verbatim}

  Hence the SPARC bit layout of this instruction is:

  \begin{tabular}[h]{lclcl}
    Macro to set  &=& \texttt{F4(x, y, z)} &in& \texttt{sparc.h}     \\
    Macro to reset  &=& \texttt{INVF4(x, y, z)} &in& \texttt{sparc.h}     \\
    x &=& 0x2      &in& \texttt{OP(x)  /* ((x) \& 0x3)  $<<$ 30 */} \\
    y &=& 0x01     &in& \texttt{OP3(y) /* ((y) \& 0x3f) $<<$ 19 */} \\
    z &=& 0x0      &in& \texttt{F3I(z) /* ((z) \& 0x1)  $<<$ 13 */} \\
    a &=& 0x1      &in& \texttt{OP\_AJIT\_BIT(a) /* ((a) \& 0x1)  $<<$ 5 */}
  \end{tabular}

  The AJIT bit  (insn[5]) is set internally by  \texttt{F4}, and hence
  there are only three arguments.

\item \textbf{ANDDCC}:\\
  \begin{center}
    \begin{tabular}[p]{|c|c|l|l|l|}
      \hline
      \textbf{Start} & \textbf{End} & \textbf{Range} & \textbf{Meaning} &
                                                                          \textbf{New Meaning}\\
      \hline
      0 & 4 & 32 & Source register 2, rs2 & No change \\
      5 & 12 & -- & \textbf{unused} & \textbf{Set bit 5 to ``1''} \\
      13 & 13 & 0,1 & The \textbf{i} bit & \textbf{Set i to ``0''} \\
      14 & 18 & 32 & Source register 1, rs1 & No change \\
      19 & 24 & 010001 & ``\textbf{op3}'' & No change \\
      25 & 29 & 32 & Destination register, rd & No change \\
      30 & 31 & 4 & Always ``10'' & No change \\
      \hline
    \end{tabular}
  \end{center}
  \begin{itemize}
  \item []\textbf{ANDDCC}: same as ANDCC, but with Instr[13]=0 (i=0), and
    Instr[5]=1.
  \item []\textbf{Syntax}: ``\texttt{anddcc  SrcReg1, SrcReg2, DestReg}''.
  \item []\textbf{Semantics}: rd(pair) $\leftarrow$ rs1(pair) $\cdot$
    rs2(pair), sets Z.
  \end{itemize}
  Bits layout:
\begin{verbatim}
    Offsets      : 31       24 23       16  15        8   7        0
    Bit layout   :  XXXX  XXXX  XXXX  XXXX   XXXX  XXXX   XXXX  XXXX
    Insn Bits    :  10       0  1000  1        0            1       
    Destination  :    DD  DDD                                       
    Source 1     :                     SSS   SS
    Source 2     :                                           S  SSSS
    Unused (0)   :                              U  UUUU   UU        
    Final layout :  10DD  DDD0  1000  1SSS   SS0U  UUUU   UU1S  SSSS
\end{verbatim}

  Hence the SPARC bit layout of this instruction is:

  \begin{tabular}[h]{lclcl}
    Macro to set  &=& \texttt{F4(x, y, z)} &in& \texttt{sparc.h}     \\
    Macro to reset  &=& \texttt{INVF4(x, y, z)} &in& \texttt{sparc.h}     \\
    x &=& 0x2      &in& \texttt{OP(x)  /* ((x) \& 0x3)  $<<$ 30 */} \\
    y &=& 0x11     &in& \texttt{OP3(y) /* ((y) \& 0x3f) $<<$ 19 */} \\
    z &=& 0x0      &in& \texttt{F3I(z) /* ((z) \& 0x1)  $<<$ 13 */} \\
    a &=& 0x1      &in& \texttt{OP\_AJIT\_BIT(a) /* ((a) \& 0x1)  $<<$ 5 */}
  \end{tabular}

  The AJIT bit  (insn[5]) is set internally by  \texttt{F4}, and hence
  there are only three arguments.

\item \textbf{ANDDN}:\\
  \begin{center}
    \begin{tabular}[p]{|c|c|l|l|l|}
      \hline
      \textbf{Start} & \textbf{End} & \textbf{Range} & \textbf{Meaning} &
                                                                          \textbf{New Meaning}\\
      \hline
      0 & 4 & 32 & Source register 2, rs2 & No change \\
      5 & 12 & -- & \textbf{unused} & \textbf{Set bit 5 to ``1''} \\
      13 & 13 & 0,1 & The \textbf{i} bit & \textbf{Set i to ``0''} \\
      14 & 18 & 32 & Source register 1, rs1 & No change \\
      19 & 24 & 000101 & ``\textbf{op3}'' & No change \\
      25 & 29 & 32 & Destination register, rd & No change \\
      30 & 31 & 4 & Always ``10'' & No change \\
      \hline
    \end{tabular}
  \end{center}
  \begin{itemize}
  \item []\textbf{ANDDN}: same as ANDN, but with Instr[13]=0 (i=0), and
    Instr[5]=1.
  \item []\textbf{Syntax}: ``\texttt{anddn  SrcReg1, SrcReg2, DestReg}''.
  \item []\textbf{Semantics}: rd(pair) $\leftarrow$ rs1(pair) $\cdot$ ($\sim$rs2(pair)).
  \end{itemize}
  Bits layout:
\begin{verbatim}
    Offsets      : 31       24 23       16  15        8   7        0
    Bit layout   :  XXXX  XXXX  XXXX  XXXX   XXXX  XXXX   XXXX  XXXX
    Insn Bits    :  10       0  0010  1        0            1       
    Destination  :    DD  DDD                                       
    Source 1     :                     SSS   SS
    Source 2     :                                           S  SSSS
    Unused (0)   :                              U  UUUU   UU        
    Final layout :  10DD  DDD0  0010  1SSS   SS0U  UUUU   UU1S  SSSS
\end{verbatim}

  Hence the SPARC bit layout of this instruction is:

  \begin{tabular}[h]{lclcl}
    Macro to set  &=& \texttt{F4(x, y, z)} &in& \texttt{sparc.h}     \\
    Macro to reset  &=& \texttt{INVF4(x, y, z)} &in& \texttt{sparc.h}     \\
    x &=& 0x2      &in& \texttt{OP(x)  /* ((x) \& 0x3)  $<<$ 30 */} \\
    y &=& 0x05     &in& \texttt{OP3(y) /* ((y) \& 0x3f) $<<$ 19 */} \\
    z &=& 0x0      &in& \texttt{F3I(z) /* ((z) \& 0x1)  $<<$ 13 */} \\
    a &=& 0x1      &in& \texttt{OP\_AJIT\_BIT(a) /* ((a) \& 0x1)  $<<$ 5 */}
  \end{tabular}

  The AJIT bit  (insn[5]) is set internally by  \texttt{F4}, and hence
  there are only three arguments.

\item \textbf{ANDDNCC}:\\
  \begin{center}
    \begin{tabular}[p]{|c|c|l|l|l|}
      \hline
      \textbf{Start} & \textbf{End} & \textbf{Range} & \textbf{Meaning} &
                                                                          \textbf{New Meaning}\\
      \hline
      0 & 4 & 32 & Source register 2, rs2 & No change \\
      5 & 12 & -- & \textbf{unused} & \textbf{Set bit 5 to ``1''} \\
      13 & 13 & 0,1 & The \textbf{i} bit & \textbf{Set i to ``0''} \\
      14 & 18 & 32 & Source register 1, rs1 & No change \\
      19 & 24 & 010101 & ``\textbf{op3}'' & No change \\
      25 & 29 & 32 & Destination register, rd & No change \\
      30 & 31 & 4 & Always ``10'' & No change \\
      \hline
    \end{tabular}
  \end{center}
  \begin{itemize}
  \item []\textbf{ANDDNCC}: same as ANDN, but with Instr[13]=0 (i=0), and
    Instr[5]=1.
  \item []\textbf{Syntax}: ``\texttt{anddncc  SrcReg1, SrcReg2, DestReg}''.
  \item []\textbf{Semantics}: rd(pair) $\leftarrow$ rs1(pair) $\cdot$
    ($\sim$rs2(pair)), sets Z.
  \end{itemize}
  Bits layout:
\begin{verbatim}
    Offsets      : 31       24 23       16  15        8   7        0
    Bit layout   :  XXXX  XXXX  XXXX  XXXX   XXXX  XXXX   XXXX  XXXX
    Insn Bits    :  10       0  1010  1        0            1       
    Destination  :    DD  DDD                                       
    Source 1     :                     SSS   SS
    Source 2     :                                           S  SSSS
    Unused (0)   :                              U  UUUU   UU        
    Final layout :  10DD  DDD0  0010  1SSS   SS0U  UUUU   UU1S  SSSS
\end{verbatim}

  Hence the SPARC bit layout of this instruction is:

  \begin{tabular}[h]{lclcl}
    Macro to set  &=& \texttt{F4(x, y, z)} &in& \texttt{sparc.h}     \\
    Macro to reset  &=& \texttt{INVF4(x, y, z)} &in& \texttt{sparc.h}     \\
    x &=& 0x2      &in& \texttt{OP(x)  /* ((x) \& 0x3)  $<<$ 30 */} \\
    y &=& 0x15     &in& \texttt{OP3(y) /* ((y) \& 0x3f) $<<$ 19 */} \\
    z &=& 0x0      &in& \texttt{F3I(z) /* ((z) \& 0x1)  $<<$ 13 */} \\
    a &=& 0x1      &in& \texttt{OP\_AJIT\_BIT(a) /* ((a) \& 0x1)  $<<$ 5 */}
  \end{tabular}

  The AJIT bit  (insn[5]) is set internally by  \texttt{F4}, and hence
  there are only three arguments.

\end{enumerate}

%%% Local Variables:
%%% mode: latex
%%% TeX-master: t
%%% End:

\subsubsection{Integer Unit Extensions Summary}
\label{sec:int:unit:extns:summary}

\begin{itemize}
  % \subsubsection{Addition and subtraction instructions:}
\label{sec:add:sub:insn:impl}
\begin{enumerate}
\item \textbf{ADDD}:\\
  \begin{center}
    \begin{tabular}[p]{|c|c|l|l|l|}
      \hline
      \textbf{Start} & \textbf{End} & \textbf{Range} & \textbf{Meaning} &
                                                                          \textbf{New Meaning}\\
      \hline
      0 & 4 & 32 & Source register 2, rs2 & No change \\
      5 & 12 & -- & \textbf{unused} & \textbf{Set bit 5 to ``1''} \\
      13 & 13 & 0,1 & The \textbf{i} bit & \textbf{Set i to ``0''} \\
      14 & 18 & 32 & Source register 1, rs1 & No change \\
      19 & 24 & 000000 & ``\textbf{op3}'' & No change \\
      25 & 29 & 32 & Destination register, rd & No change \\
      30 & 31 & 4 & Always ``10'' & No change \\
      \hline
    \end{tabular}
  \end{center}
  \begin{itemize}
  \item []\textbf{ADDD}: same as ADD, but with Instr[13]=0 (i=0), and
    Instr[5]=1.
  \item []\textbf{Syntax}: ``\texttt{addd  SrcReg1, SrcReg2, DestReg}''.
  \item []\textbf{Semantics}: rd(pair) $\leftarrow$ rs1(pair) + rs2(pair).
  \end{itemize}
  Bits layout:
\begin{verbatim}
    Offsets      : 31       24 23       16  15        8   7        0
    Bit layout   :  XXXX  XXXX  XXXX  XXXX   XXXX  XXXX   XXXX  XXXX
    Insn Bits    :  10       0  0000  0        0            1       
    Destination  :    DD  DDD                                       
    Source 1     :                     SSS   SS
    Source 2     :                                           S  SSSS
    Unused (0)   :                              U  UUUU   UU        
    Final layout :  10DD  DDD0  0000  0SSS   SS0U  UUUU   UU1S  SSSS
\end{verbatim}

  Hence the SPARC bit layout of this instruction is:

  \begin{tabular}[h]{lclcl}
    Macro to set  &=& \texttt{F4(x, y, z)} &in& \texttt{sparc.h}     \\
    Macro to reset  &=& \texttt{INVF4(x, y, z)} &in& \texttt{sparc.h}     \\
    x &=& 0x2      &in& \texttt{OP(x)  /* ((x) \& 0x3)  $<<$ 30 */} \\
    y &=& 0x00     &in& \texttt{OP3(y) /* ((y) \& 0x3f) $<<$ 19 */} \\
    z &=& 0x0      &in& \texttt{F3I(z) /* ((z) \& 0x1)  $<<$ 13 */} \\
    a &=& 0x1      &in& \texttt{OP\_AJIT\_BIT(a) /* ((a) \& 0x1)  $<<$ 5 */}
  \end{tabular}

  The AJIT bit  (insn[5]) is set internally by  \texttt{F4}, and hence
  there are only three arguments.

\item \textbf{ADDDCC}:\\
  \begin{center}
    \begin{tabular}[p]{|c|c|l|l|l|}
      \hline
      \textbf{Start} & \textbf{End} & \textbf{Range} & \textbf{Meaning} &
                                                                          \textbf{New Meaning}\\
      \hline
      0 & 4 & 32 & Source register 2, rs2 & No change \\
      5 & 12 & -- & \textbf{unused} & \textbf{Set bit 5 to ``1''} \\
      13 & 13 & 0,1 & The \textbf{i} bit & \textbf{Set i to ``0''} \\
      14 & 18 & 32 & Source register 1, rs1 & No change \\
      19 & 24 & 010000 & ``\textbf{op3}'' & No change \\
      25 & 29 & 32 & Destination register, rd & No change \\
      30 & 31 & 4 & Always ``10'' & No change \\
      \hline
    \end{tabular}
  \end{center}
  New addition:
  \begin{itemize}
  \item []\textbf{ADDDCC}: same as ADDCC, but with Instr[13]=0 (i=0), and
    Instr[5]=1.
  \item []\textbf{Syntax}: ``\texttt{adddcc  SrcReg1, SrcReg2, DestReg}''.
  \item []\textbf{Semantics}: rd(pair) $\leftarrow$ rs1(pair) + rs2(pair), set Z,N
  \end{itemize}
  Bits layout:
\begin{verbatim}
    Offsets      : 31       24 23       16  15        8   7        0
    Bit layout   :  XXXX  XXXX  XXXX  XXXX   XXXX  XXXX   XXXX  XXXX
    Insn Bits    :  10       0  1000  0        0            1       
    Destination  :    DD  DDD                                       
    Source 1     :                     SSS   SS
    Source 2     :                                           S  SSSS
    Unused (0)   :                              U  UUUU   UU        
    Final layout :  10DD  DDD0  1000  0SSS   SS0U  UUUU   UU1S  SSSS
\end{verbatim}

  Hence the SPARC bit layout of this instruction is:

  \begin{tabular}[h]{lclcl}
    Macro to set  &=& \texttt{F4(x, y, z)} &in& \texttt{sparc.h}     \\
    Macro to reset  &=& \texttt{INVF4(x, y, z)} &in& \texttt{sparc.h}     \\
    x &=& 0x2      &in& \texttt{OP(x)  /* ((x) \& 0x3)  $<<$ 30 */} \\
    y &=& 0x10     &in& \texttt{OP3(y) /* ((y) \& 0x3f) $<<$ 19 */} \\
    z &=& 0x0      &in& \texttt{F3I(z) /* ((z) \& 0x1)  $<<$ 13 */} \\
    a &=& 0x1      &in& \texttt{OP\_AJIT\_BIT(a) /* ((a) \& 0x1)  $<<$ 5 */}
  \end{tabular}

  The AJIT bit  (insn[5]) is set internally by  \texttt{F4}, and hence
  there are only three arguments.

\item \textbf{SUBD}:\\
  \begin{center}
    \begin{tabular}[p]{|c|c|l|l|l|}
      \hline
      \textbf{Start} & \textbf{End} & \textbf{Range} & \textbf{Meaning} &
                                                                          \textbf{New Meaning}\\
      \hline
      0 & 4 & 32 & Source register 2, rs2 & No change \\
      5 & 12 & -- & \textbf{unused} & \textbf{Set bit 5 to ``1''} \\
      13 & 13 & 0,1 & The \textbf{i} bit & \textbf{Set i to ``0''} \\
      14 & 18 & 32 & Source register 1, rs1 & No change \\
      19 & 24 & 000100 & ``\textbf{op3}'' & No change \\
      25 & 29 & 32 & Destination register, rd & No change \\
      30 & 31 & 4 & Always ``10'' & No change \\
      \hline
    \end{tabular}
  \end{center}
  New addition:
  \begin{itemize}
  \item []\textbf{SUBD}: same as SUB, but with Instr[13]=0 (i=0), and
    Instr[5]=1.
  \item []\textbf{Syntax}: ``\texttt{subd  SrcReg1, SrcReg2, DestReg}''.
  \item []\textbf{Semantics}: rd(pair) $\leftarrow$ rs1(pair) - rs2(pair).
  \end{itemize}
  Bits layout:
\begin{verbatim}
    Offsets      : 31       24 23       16  15        8   7        0
    Bit layout   :  XXXX  XXXX  XXXX  XXXX   XXXX  XXXX   XXXX  XXXX
    Insn Bits    :  10       0  0010  0        0            1       
    Destination  :    DD  DDD                                       
    Source 1     :                     SSS   SS
    Source 2     :                                           S  SSSS
    Unused (0)   :                              U  UUUU   UU        
    Final layout :  10DD  DDD0  0010  0SSS   SS0U  UUUU   UU1S  SSSS
\end{verbatim}

  Hence the SPARC bit layout of this instruction is:

  \begin{tabular}[h]{lclcl}
    Macro to set  &=& \texttt{F4(x, y, z)} &in& \texttt{sparc.h}     \\
    Macro to reset  &=& \texttt{INVF4(x, y, z)} &in& \texttt{sparc.h}     \\
    x &=& 0x2      &in& \texttt{OP(x)  /* ((x) \& 0x3)  $<<$ 30 */} \\
    y &=& 0x04     &in& \texttt{OP3(y) /* ((y) \& 0x3f) $<<$ 19 */} \\
    z &=& 0x0      &in& \texttt{F3I(z) /* ((z) \& 0x1)  $<<$ 13 */} \\
    a &=& 0x1      &in& \texttt{OP\_AJIT\_BIT(a) /* ((a) \& 0x1)  $<<$ 5 */}
  \end{tabular}

  The AJIT bit  (insn[5]) is set internally by  \texttt{F4}, and hence
  there are only three arguments.

\item \textbf{SUBDCC}:\\
  \begin{center}
    \begin{tabular}[p]{|c|c|l|l|l|}
      \hline
      \textbf{Start} & \textbf{End} & \textbf{Range} & \textbf{Meaning} &
                                                                          \textbf{New Meaning}\\
      \hline
      0 & 4 & 32 & Source register 2, rs2 & No change \\
      5 & 12 & -- & \textbf{unused} & \textbf{Set bit 5 to ``1''} \\
      13 & 13 & 0,1 & The \textbf{i} bit & \textbf{Set i to ``0''} \\
      14 & 18 & 32 & Source register 1, rs1 & No change \\
      19 & 24 & 010100 & ``\textbf{op3}'' & No change \\
      25 & 29 & 32 & Destination register, rd & No change \\
      30 & 31 & 4 & Always ``10'' & No change \\
      \hline
    \end{tabular}
  \end{center}
  New addition:
  \begin{itemize}
  \item []\textbf{SUBDCC}: same as SUBCC, but with Instr[13]=0 (i=0), and
    Instr[5]=1.
  \item []\textbf{Syntax}: ``\texttt{subdcc  SrcReg1, SrcReg2, DestReg}''.
  \item []\textbf{Semantics}: rd(pair) $\leftarrow$ rs1(pair) - rs2(pair), set Z,N
  \end{itemize}
  Bits layout:
\begin{verbatim}
    Offsets      : 31       24 23       16  15        8   7        0
    Bit layout   :  XXXX  XXXX  XXXX  XXXX   XXXX  XXXX   XXXX  XXXX
    Insn Bits    :  10       0  1010  0        0            1       
    Destination  :    DD  DDD                                       
    Source 1     :                     SSS   SS
    Source 2     :                                           S  SSSS
    Unused (0)   :                              U  UUUU   UU        
    Final layout :  10DD  DDD0  1010  0SSS   SS0U  UUUU   UU1S  SSSS
\end{verbatim}

  Hence the SPARC bit layout of this instruction is:

  \begin{tabular}[h]{lclcl}
    Macro to set  &=& \texttt{F4(x, y, z)} &in& \texttt{sparc.h}     \\
    Macro to reset  &=& \texttt{INVF4(x, y, z)} &in& \texttt{sparc.h}     \\
    x &=& 0x2      &in& \texttt{OP(x)  /* ((x) \& 0x3)  $<<$ 30 */} \\
    y &=& 0x14     &in& \texttt{OP3(y) /* ((y) \& 0x3f) $<<$ 19 */} \\
    z &=& 0x0      &in& \texttt{F3I(z) /* ((z) \& 0x1)  $<<$ 13 */} \\
    a &=& 0x1      &in& \texttt{OP\_AJIT\_BIT(a) /* ((a) \& 0x1)  $<<$ 5 */}
  \end{tabular}

  The AJIT bit  (insn[5]) is set internally by  \texttt{F4}, and hence
  there are only three arguments.
\end{enumerate}

\item {Addition and subtraction instructions:}\\
  \begin{enumerate}
  \item \textbf{ADDD}:\\
    \begin{tabular}[h]{lclcl}
      Macro to set  &=& \texttt{F4(x, y, z)} &in& \texttt{sparc.h}     \\
      Macro to reset  &=& \texttt{INVF4(x, y, z)} &in& \texttt{sparc.h}     \\
      x &=& 0x2      &in& \texttt{OP(x)  /* ((x) \& 0x3)  $<<$ 30 */} \\
      y &=& 0x00     &in& \texttt{OP3(y) /* ((y) \& 0x3f) $<<$ 19 */} \\
      z &=& 0x0      &in& \texttt{F3I(z) /* ((z) \& 0x1)  $<<$ 13 */} \\
      a &=& 0x1      &in& \texttt{OP\_AJIT\_BIT(a) /* ((a) \& 0x1)  $<<$ 5 */}
    \end{tabular}

    The AJIT bit  (insn[5]) is set internally by  \texttt{F4}, and hence
    there are only three arguments.

  \item \textbf{ADDDCC}:\\
    \begin{tabular}[h]{lclcl}
      Macro to set  &=& \texttt{F4(x, y, z)} &in& \texttt{sparc.h}     \\
      Macro to reset  &=& \texttt{INVF4(x, y, z)} &in& \texttt{sparc.h}     \\
      x &=& 0x2      &in& \texttt{OP(x)  /* ((x) \& 0x3)  $<<$ 30 */} \\
      y &=& 0x10     &in& \texttt{OP3(y) /* ((y) \& 0x3f) $<<$ 19 */} \\
      z &=& 0x0      &in& \texttt{F3I(z) /* ((z) \& 0x1)  $<<$ 13 */} \\
      a &=& 0x1      &in& \texttt{OP\_AJIT\_BIT(a) /* ((a) \& 0x1)  $<<$ 5 */}
    \end{tabular}

    The AJIT bit  (insn[5]) is set internally by  \texttt{F4}, and hence
    there are only three arguments.

  \item \textbf{SUBD}:\\
    \begin{tabular}[h]{lclcl}
      Macro to set  &=& \texttt{F4(x, y, z)} &in& \texttt{sparc.h}     \\
      Macro to reset  &=& \texttt{INVF4(x, y, z)} &in& \texttt{sparc.h}     \\
      x &=& 0x2      &in& \texttt{OP(x)  /* ((x) \& 0x3)  $<<$ 30 */} \\
      y &=& 0x04     &in& \texttt{OP3(y) /* ((y) \& 0x3f) $<<$ 19 */} \\
      z &=& 0x0      &in& \texttt{F3I(z) /* ((z) \& 0x1)  $<<$ 13 */} \\
      a &=& 0x1      &in& \texttt{OP\_AJIT\_BIT(a) /* ((a) \& 0x1)  $<<$ 5 */}
    \end{tabular}

    The AJIT bit  (insn[5]) is set internally by  \texttt{F4}, and hence
    there are only three arguments.

  \item \textbf{SUBDCC}:\\
    \begin{tabular}[h]{lclcl}
      Macro to set  &=& \texttt{F4(x, y, z)} &in& \texttt{sparc.h}     \\
      Macro to reset  &=& \texttt{INVF4(x, y, z)} &in& \texttt{sparc.h}     \\
      x &=& 0x2      &in& \texttt{OP(x)  /* ((x) \& 0x3)  $<<$ 30 */} \\
      y &=& 0x14     &in& \texttt{OP3(y) /* ((y) \& 0x3f) $<<$ 19 */} \\
      z &=& 0x0      &in& \texttt{F3I(z) /* ((z) \& 0x1)  $<<$ 13 */} \\
      a &=& 0x1      &in& \texttt{OP\_AJIT\_BIT(a) /* ((a) \& 0x1)  $<<$ 5 */}
    \end{tabular}

    The AJIT bit  (insn[5]) is set internally by  \texttt{F4}, and hence
    there are only three arguments.
  \end{enumerate}

  % \subsubsection{Multiplication and division instructions:}
\label{sec:mul:div:insn:impl}
\begin{enumerate}
\item \textbf{UMULD}: Unsigned Integer Multiply AJIT, no immediate
  version (i.e. i is always 0).\\
	\textbf{NOTE:} The \emph{suggested} mnemonic ``umuld'' conflicts with a mnemonic of the same name for another sparc architecture (other than v8).   Hence we change it to: ``\textbf{umuldaj}'' in the implementation, but not in the documentation below.

 This conflict occurs despite forcing the GNU assembler to assemble for v8 only via the command line switch ``-Av8''! It appears that forcing the assembler to use v8 is not universally applied throughout the assembler code. 
  \begin{center}
    \begin{tabular}[p]{|c|c|l|l|l|}
      \hline
      \textbf{Start} & \textbf{End} & \textbf{Range} & \textbf{Meaning} &
                                                                          \textbf{New Meaning}\\
      \hline
      0 & 4 & 32 & Source register 2, rs2 & No change \\
      5 & 12 & -- & \textbf{unused} & \textbf{Set bit 5 to ``1''} \\
      13 & 13 & 0,1 & The \textbf{i} bit & \textbf{Set i to ``0''} \\
      14 & 18 & 32 & Source register 1, rs1 & No change \\
      19 & 24 & 001010 & ``\textbf{op3}'' & No change \\
      25 & 29 & 32 & Destination register, rd & No change \\
      30 & 31 & 4 & Always ``10'' & No change \\
      \hline
    \end{tabular}
  \end{center}
  \begin{itemize}
  \item []\textbf{UMULD}: same as UMUL, but with Instr[13]=0 (i=0), and
    Instr[5]=1.
  \item []\textbf{Syntax}: ``\texttt{umuld  SrcReg1, SrcReg2, DestReg}''.
  \item []\textbf{Semantics}: rd(pair) $\leftarrow$ rs1(pair) * rs2(pair).
  \end{itemize}
  Bits layout:
\begin{verbatim}
    Offsets      : 31       24 23       16  15        8   7        0
    Bit layout   :  XXXX  XXXX  XXXX  XXXX   XXXX  XXXX   XXXX  XXXX
    Insn Bits    :  10       0  0101  0        0            1       
    Destination  :    DD  DDD                                       
    Source 1     :                     SSS   SS
    Source 2     :                                           S  SSSS
    Unused (0)   :                              U  UUUU   UU        
    Final layout :  10DD  DDD0  0101  0SSS   SS0U  UUUU   UU1S  SSSS
\end{verbatim}

  Hence the SPARC bit layout of this instruction is:

  \begin{tabular}[h]{lclcl}
    Macro to set  &=& \texttt{F4(x, y, z)} &in& \texttt{sparc.h}     \\
    Macro to reset  &=& \texttt{INVF4(x, y, z)} &in& \texttt{sparc.h}     \\
    x &=& 0x2      &in& \texttt{OP(x)  /* ((x) \& 0x3)  $<<$ 30 */} \\
    y &=& 0x0A     &in& \texttt{OP3(y) /* ((y) \& 0x3f) $<<$ 19 */} \\
    z &=& 0x0      &in& \texttt{F3I(z) /* ((z) \& 0x1)  $<<$ 13 */} \\
    a &=& 0x1      &in& \texttt{OP\_AJIT\_BIT(a) /* ((a) \& 0x1)  $<<$ 5 */}
  \end{tabular}

  The AJIT bit  (insn[5]) is set internally by  \texttt{F4}, and hence
  there are only three arguments.

\item \textbf{UMULDCC}:\\
  \begin{center}
    \begin{tabular}[p]{|c|c|l|l|l|}
      \hline
      \textbf{Start} & \textbf{End} & \textbf{Range} & \textbf{Meaning} &
                                                                          \textbf{New Meaning}\\
      \hline
      0 & 4 & 32 & Source register 2, rs2 & No change \\
      5 & 12 & -- & \textbf{unused} & \textbf{Set bit 5 to ``1''} \\
      13 & 13 & 0,1 & The \textbf{i} bit & \textbf{Set i to ``0''} \\
      14 & 18 & 32 & Source register 1, rs1 & No change \\
      19 & 24 & 011010 & ``\textbf{op3}'' & No change \\
      25 & 29 & 32 & Destination register, rd & No change \\
      30 & 31 & 4 & Always ``10'' & No change \\
      \hline
    \end{tabular}
  \end{center}
  New addition:
  \begin{itemize}
  \item []\textbf{UMULDCC}: same as UMULCC, but with Instr[13]=0 (i=0), and
    Instr[5]=1.
  \item []\textbf{Syntax}: ``\texttt{umuldcc  SrcReg1, SrcReg2, DestReg}''.
  \item []\textbf{Semantics}: rd(pair) $\leftarrow$ rs1(pair) * rs2(pair), set Z
  \end{itemize}
  Bits layout:
\begin{verbatim}
    Offsets      : 31       24 23       16  15        8   7        0
    Bit layout   :  XXXX  XXXX  XXXX  XXXX   XXXX  XXXX   XXXX  XXXX
    Insn Bits    :  10       0  1101  0        0            1       
    Destination  :    DD  DDD                                       
    Source 1     :                     SSS   SS
    Source 2     :                                           S  SSSS
    Unused (0)   :                              U  UUUU   UU        
    Final layout :  10DD  DDD0  1101  0SSS   SS0U  UUUU   UU1S  SSSS
\end{verbatim}

  Hence the SPARC bit layout of this instruction is:

  \begin{tabular}[h]{lclcl}
    Macro to set  &=& \texttt{F4(x, y, z)} &in& \texttt{sparc.h}     \\
    Macro to reset  &=& \texttt{INVF4(x, y, z)} &in& \texttt{sparc.h}     \\
    x &=& 0x2      &in& \texttt{OP(x)  /* ((x) \& 0x3)  $<<$ 30 */} \\
    y &=& 0x1A     &in& \texttt{OP3(y) /* ((y) \& 0x3f) $<<$ 19 */} \\
    z &=& 0x0      &in& \texttt{F3I(z) /* ((z) \& 0x1)  $<<$ 13 */} \\
    a &=& 0x1      &in& \texttt{OP\_AJIT\_BIT(a) /* ((a) \& 0x1)  $<<$ 5 */}
  \end{tabular}

  The AJIT bit  (insn[5]) is set internally by  \texttt{F4}, and hence
  there are only three arguments.

\item \textbf{SMULD}: Unsigned Integer Multiply AJIT, no immediate
  version (i.e. i is always 0).\\
  \begin{center}
    \begin{tabular}[p]{|c|c|l|l|l|}
      \hline
      \textbf{Start} & \textbf{End} & \textbf{Range} & \textbf{Meaning} &
                                                                          \textbf{New Meaning}\\
      \hline
      0 & 4 & 32 & Source register 2, rs2 & No change \\
      5 & 12 & -- & \textbf{unused} & \textbf{Set bit 5 to ``1''} \\
      13 & 13 & 0,1 & The \textbf{i} bit & \textbf{Set i to ``0''} \\
      14 & 18 & 32 & Source register 1, rs1 & No change \\
      19 & 24 & 001011 & ``\textbf{op3}'' & No change \\
      25 & 29 & 32 & Destination register, rd & No change \\
      30 & 31 & 4 & Always ``10'' & No change \\
      \hline
    \end{tabular}
  \end{center}
  \begin{itemize}
  \item []\textbf{SMULD}: same as SMUL, but with Instr[13]=0 (i=0), and
    Instr[5]=1.
  \item []\textbf{Syntax}: ``\texttt{smuld  SrcReg1, SrcReg2, DestReg}''.
  \item []\textbf{Semantics}: rd(pair) $\leftarrow$ rs1(pair) *
    rs2(pair) (signed).
  \end{itemize}
  Bits layout:
\begin{verbatim}
    Offsets      : 31       24 23       16  15        8   7        0
    Bit layout   :  XXXX  XXXX  XXXX  XXXX   XXXX  XXXX   XXXX  XXXX
    Insn Bits    :  10       0  0101  1        0            1       
    Destination  :    DD  DDD                                       
    Source 1     :                     SSS   SS
    Source 2     :                                           S  SSSS
    Unused (0)   :                              U  UUUU   UU        
    Final layout :  10DD  DDD0  0101  1SSS   SS0U  UUUU   UU1S  SSSS
\end{verbatim}

  Hence the SPARC bit layout of this instruction is:

  \begin{tabular}[h]{lclcl}
    Macro to set  &=& \texttt{F4(x, y, z)} &in& \texttt{sparc.h}     \\
    Macro to reset  &=& \texttt{INVF4(x, y, z)} &in& \texttt{sparc.h}     \\
    x &=& 0x2      &in& \texttt{OP(x)  /* ((x) \& 0x3)  $<<$ 30 */} \\
    y &=& 0x0B     &in& \texttt{OP3(y) /* ((y) \& 0x3f) $<<$ 19 */} \\
    z &=& 0x0      &in& \texttt{F3I(z) /* ((z) \& 0x1)  $<<$ 13 */} \\
    a &=& 0x1      &in& \texttt{OP\_AJIT\_BIT(a) /* ((a) \& 0x1)  $<<$ 5 */}
  \end{tabular}

  The AJIT bit  (insn[5]) is set internally by  \texttt{F4}, and hence
  there are only three arguments.

\item \textbf{SMULDCC}:\\
  \begin{center}
    \begin{tabular}[p]{|c|c|l|l|l|}
      \hline
      \textbf{Start} & \textbf{End} & \textbf{Range} & \textbf{Meaning} &
                                                                          \textbf{New Meaning}\\
      \hline
      0 & 4 & 32 & Source register 2, rs2 & No change \\
      5 & 12 & -- & \textbf{unused} & \textbf{Set bit 5 to ``1''} \\
      13 & 13 & 0,1 & The \textbf{i} bit & \textbf{Set i to ``0''} \\
      14 & 18 & 32 & Source register 1, rs1 & No change \\
      19 & 24 & 011011 & ``\textbf{op3}'' & No change \\
      25 & 29 & 32 & Destination register, rd & No change \\
      30 & 31 & 4 & Always ``10'' & No change \\
      \hline
    \end{tabular}
  \end{center}
  New addition:
  \begin{itemize}
  \item []\textbf{SMULDCC}: same as SMULCC, but with Instr[13]=0 (i=0), and
    Instr[5]=1.
  \item []\textbf{Syntax}: ``\texttt{smuldcc  SrcReg1, SrcReg2, DestReg}''.
  \item []\textbf{Semantics}: rd(pair) $\leftarrow$ rs1(pair) *
    rs2(pair) (signed), set Z,N,O
  \end{itemize}
  Bits layout:
\begin{verbatim}
    Offsets      : 31       24 23       16  15        8   7        0
    Bit layout   :  XXXX  XXXX  XXXX  XXXX   XXXX  XXXX   XXXX  XXXX
    Insn Bits    :  10       0  1101  1        0            1       
    Destination  :    DD  DDD                                       
    Source 1     :                     SSS   SS
    Source 2     :                                           S  SSSS
    Unused (0)   :                              U  UUUU   UU        
    Final layout :  10DD  DDD0  1101  1SSS   SS0U  UUUU   UU1S  SSSS
\end{verbatim}

  Hence the SPARC bit layout of this instruction is:

  \begin{tabular}[h]{lclcl}
    Macro to set  &=& \texttt{F4(x, y, z)} &in& \texttt{sparc.h}     \\
    Macro to reset  &=& \texttt{INVF4(x, y, z)} &in& \texttt{sparc.h}     \\
    x &=& 0x2      &in& \texttt{OP(x)  /* ((x) \& 0x3)  $<<$ 30 */} \\
    y &=& 0x1B     &in& \texttt{OP3(y) /* ((y) \& 0x3f) $<<$ 19 */} \\
    z &=& 0x0      &in& \texttt{F3I(z) /* ((z) \& 0x1)  $<<$ 13 */} \\
    a &=& 0x1      &in& \texttt{OP\_AJIT\_BIT(a) /* ((a) \& 0x1)  $<<$ 5 */}
  \end{tabular}

  The AJIT bit  (insn[5]) is set internally by  \texttt{F4}, and hence
  there are only three arguments.

\item \textbf{UDIVD}:\\
  \begin{center}
    \begin{tabular}[p]{|c|c|l|l|l|}
      \hline
      \textbf{Start} & \textbf{End} & \textbf{Range} & \textbf{Meaning} &
                                                                          \textbf{New Meaning}\\
      \hline
      0 & 4 & 32 & Source register 2, rs2 & No change \\
      5 & 12 & -- & \textbf{unused} & \textbf{Set bit 5 to ``1''} \\
      13 & 13 & 0,1 & The \textbf{i} bit & \textbf{Set i to ``0''} \\
      14 & 18 & 32 & Source register 1, rs1 & No change \\
      19 & 24 & 001110 & ``\textbf{op3}'' & No change \\
      25 & 29 & 32 & Destination register, rd & No change \\
      30 & 31 & 4 & Always ``10'' & No change \\
      \hline
    \end{tabular}
  \end{center}
  New addition:
  \begin{itemize}
  \item []\textbf{UDIVD}: same as UDIV, but with Instr[13]=0 (i=0), and
    Instr[5]=1.
  \item []\textbf{Syntax}: ``\texttt{udivd  SrcReg1, SrcReg2, DestReg}''.
  \item []\textbf{Semantics}: rd(pair) $\leftarrow$ rs1(pair) / rs2(pair).
  \end{itemize}
  Bits layout:
\begin{verbatim}
    Offsets      : 31       24 23       16  15        8   7        0
    Bit layout   :  XXXX  XXXX  XXXX  XXXX   XXXX  XXXX   XXXX  XXXX
    Insn Bits    :  10       0  0111  0        0            1       
    Destination  :    DD  DDD                                       
    Source 1     :                     SSS   SS
    Source 2     :                                           S  SSSS
    Unused (0)   :                              U  UUUU   UU        
    Final layout :  10DD  DDD0  0111  0SSS   SS0U  UUUU   UU1S  SSSS
\end{verbatim}

  Hence the SPARC bit layout of this instruction is:

  \begin{tabular}[h]{lclcl}
    Macro to set  &=& \texttt{F4(x, y, z)} &in& \texttt{sparc.h}     \\
    Macro to reset  &=& \texttt{INVF4(x, y, z)} &in& \texttt{sparc.h}     \\
    x &=& 0x2      &in& \texttt{OP(x)  /* ((x) \& 0x3)  $<<$ 30 */} \\
    y &=& 0x0E     &in& \texttt{OP3(y) /* ((y) \& 0x3f) $<<$ 19 */} \\
    z &=& 0x0      &in& \texttt{F3I(z) /* ((z) \& 0x1)  $<<$ 13 */} \\
    a &=& 0x1      &in& \texttt{OP\_AJIT\_BIT(a) /* ((a) \& 0x1)  $<<$ 5 */}
  \end{tabular}

  The AJIT bit  (insn[5]) is set internally by  \texttt{F4}, and hence
  there are only three arguments.

\item \textbf{UDIVDCC}:\\
  \begin{center}
    \begin{tabular}[p]{|c|c|l|l|l|}
      \hline
      \textbf{Start} & \textbf{End} & \textbf{Range} & \textbf{Meaning} &
                                                                          \textbf{New Meaning}\\
      \hline
      0 & 4 & 32 & Source register 2, rs2 & No change \\
      5 & 12 & -- & \textbf{unused} & \textbf{Set bit 5 to ``1''} \\
      13 & 13 & 0,1 & The \textbf{i} bit & \textbf{Set i to ``0''} \\
      14 & 18 & 32 & Source register 1, rs1 & No change \\
      19 & 24 & 011110 & ``\textbf{op3}'' & No change \\
      25 & 29 & 32 & Destination register, rd & No change \\
      30 & 31 & 4 & Always ``10'' & No change \\
      \hline
    \end{tabular}
  \end{center}
  New addition:
  \begin{itemize}
  \item []\textbf{UDIVDCC}: same as UDIVCC, but with Instr[13]=0 (i=0), and
    Instr[5]=1.
  \item []\textbf{Syntax}: ``\texttt{udivdcc  SrcReg1, SrcReg2, DestReg}''.
  \item []\textbf{Semantics}: rd(pair) $\leftarrow$ rs1(pair) / rs2(pair), set Z,O
  \end{itemize}
  Bits layout:
\begin{verbatim}
    Offsets      : 31       24 23       16  15        8   7        0
    Bit layout   :  XXXX  XXXX  XXXX  XXXX   XXXX  XXXX   XXXX  XXXX
    Insn Bits    :  10       0  1111  0        0            1       
    Destination  :    DD  DDD                                       
    Source 1     :                     SSS   SS
    Source 2     :                                           S  SSSS
    Unused (0)   :                              U  UUUU   UU        
    Final layout :  10DD  DDD0  1111  0SSS   SS0U  UUUU   UU1S  SSSS
\end{verbatim}

  Hence the SPARC bit layout of this instruction is:

  \begin{tabular}[h]{lclcl}
    Macro to set  &=& \texttt{F4(x, y, z)} &in& \texttt{sparc.h}     \\
    Macro to reset  &=& \texttt{INVF4(x, y, z)} &in& \texttt{sparc.h}     \\
    x &=& 0x2      &in& \texttt{OP(x)  /* ((x) \& 0x3)  $<<$ 30 */} \\
    y &=& 0x1E     &in& \texttt{OP3(y) /* ((y) \& 0x3f) $<<$ 19 */} \\
    z &=& 0x0      &in& \texttt{F3I(z) /* ((z) \& 0x1)  $<<$ 13 */} \\
    a &=& 0x1      &in& \texttt{OP\_AJIT\_BIT(a) /* ((a) \& 0x1)  $<<$ 5 */}
  \end{tabular}

  The AJIT bit  (insn[5]) is set internally by  \texttt{F4}, and hence
  there are only three arguments.

\item \textbf{SDIVD}:\\
  \begin{center}
    \begin{tabular}[p]{|c|c|l|l|l|}
      \hline
      \textbf{Start} & \textbf{End} & \textbf{Range} & \textbf{Meaning} &
                                                                          \textbf{New Meaning}\\
      \hline
      0 & 4 & 32 & Source register 2, rs2 & No change \\
      5 & 12 & -- & \textbf{unused} & \textbf{Set bit 5 to ``1''} \\
      13 & 13 & 0,1 & The \textbf{i} bit & \textbf{Set i to ``0''} \\
      14 & 18 & 32 & Source register 1, rs1 & No change \\
      19 & 24 & 001111 & ``\textbf{op3}'' & No change \\
      25 & 29 & 32 & Destination register, rd & No change \\
      30 & 31 & 4 & Always ``10'' & No change \\
      \hline
    \end{tabular}
  \end{center}
  New addition:
  \begin{itemize}
  \item []\textbf{SDIVD}: same as SDIV, but with Instr[13]=0 (i=0), and
    Instr[5]=1.
  \item []\textbf{Syntax}: ``\texttt{sdivd  SrcReg1, SrcReg2, DestReg}''.
  \item []\textbf{Semantics}: rd(pair) $\leftarrow$ rs1(pair) /
    rs2(pair) (signed).
  \end{itemize}
  Bits layout:
\begin{verbatim}
    Offsets      : 31       24 23       16  15        8   7        0
    Bit layout   :  XXXX  XXXX  XXXX  XXXX   XXXX  XXXX   XXXX  XXXX
    Insn Bits    :  10       0  0111  1        0            1       
    Destination  :    DD  DDD                                       
    Source 1     :                     SSS   SS
    Source 2     :                                           S  SSSS
    Unused (0)   :                              U  UUUU   UU        
    Final layout :  10DD  DDD0  0111  1SSS   SS0U  UUUU   UU1S  SSSS
\end{verbatim}

  Hence the SPARC bit layout of this instruction is:

  \begin{tabular}[h]{lclcl}
    Macro to set  &=& \texttt{F4(x, y, z)} &in& \texttt{sparc.h}     \\
    Macro to reset  &=& \texttt{INVF4(x, y, z)} &in& \texttt{sparc.h}     \\
    x &=& 0x2      &in& \texttt{OP(x)  /* ((x) \& 0x3)  $<<$ 30 */} \\
    y &=& 0x0F     &in& \texttt{OP3(y) /* ((y) \& 0x3f) $<<$ 19 */} \\
    z &=& 0x0      &in& \texttt{F3I(z) /* ((z) \& 0x1)  $<<$ 13 */} \\
    a &=& 0x1      &in& \texttt{OP\_AJIT\_BIT(a) /* ((a) \& 0x1)  $<<$ 5 */}
  \end{tabular}

  The AJIT bit  (insn[5]) is set internally by  \texttt{F4}, and hence
  there are only three arguments.

\item \textbf{SDIVDCC}:\\
  \begin{center}
    \begin{tabular}[p]{|c|c|l|l|l|}
      \hline
      \textbf{Start} & \textbf{End} & \textbf{Range} & \textbf{Meaning} &
                                                                          \textbf{New Meaning}\\
      \hline
      0 & 4 & 32 & Source register 2, rs2 & No change \\
      5 & 12 & -- & \textbf{unused} & \textbf{Set bit 5 to ``1''} \\
      13 & 13 & 0,1 & The \textbf{i} bit & \textbf{Set i to ``0''} \\
      14 & 18 & 32 & Source register 1, rs1 & No change \\
      19 & 24 & 011111 & ``\textbf{op3}'' & No change \\
      25 & 29 & 32 & Destination register, rd & No change \\
      30 & 31 & 4 & Always ``10'' & No change \\
      \hline
    \end{tabular}
  \end{center}
  New addition:
  \begin{itemize}
  \item []\textbf{SDIVDCC}: same as SDIVCC, but with Instr[13]=0 (i=0), and
    Instr[5]=1.
  \item []\textbf{Syntax}: ``\texttt{sdivdcc  SrcReg1, SrcReg2, DestReg}''.
  \item []\textbf{Semantics}: rd(pair) $\leftarrow$ rs1(pair) /
    rs2(pair) (signed), set Z,N,O
  \end{itemize}
  Bits layout:
\begin{verbatim}
    Offsets      : 31       24 23       16  15        8   7        0
    Bit layout   :  XXXX  XXXX  XXXX  XXXX   XXXX  XXXX   XXXX  XXXX
    Insn Bits    :  10       0  1111  1        0            1       
    Destination  :    DD  DDD                                       
    Source 1     :                     SSS   SS
    Source 2     :                                           S  SSSS
    Unused (0)   :                              U  UUUU   UU        
    Final layout :  10DD  DDD0  1111  1SSS   SS0U  UUUU   UU1S  SSSS
\end{verbatim}

  Hence the SPARC bit layout of this instruction is:

  \begin{tabular}[h]{lclcl}
    Macro to set  &=& \texttt{F4(x, y, z)} &in& \texttt{sparc.h}     \\
    Macro to reset  &=& \texttt{INVF4(x, y, z)} &in& \texttt{sparc.h}     \\
    x &=& 0x2      &in& \texttt{OP(x)  /* ((x) \& 0x3)  $<<$ 30 */} \\
    y &=& 0x1F     &in& \texttt{OP3(y) /* ((y) \& 0x3f) $<<$ 19 */} \\
    z &=& 0x0      &in& \texttt{F3I(z) /* ((z) \& 0x1)  $<<$ 13 */} \\
    a &=& 0x1      &in& \texttt{OP\_AJIT\_BIT(a) /* ((a) \& 0x1)  $<<$ 5 */}
  \end{tabular}

  The AJIT bit  (insn[5]) is set internally by  \texttt{F4}, and hence
  there are only three arguments.
\end{enumerate}

%%% Local Variables:
%%% mode: latex
%%% TeX-master: t
%%% End:

\item {Multiplication and division instructions:} \\
  \begin{enumerate}
  \item \textbf{UMULD}: Unsigned Integer Multiply AJIT, no immediate
    version (i.e. i is always 0).\\
    \begin{tabular}[h]{lclcl}
      Macro to set  &=& \texttt{F4(x, y, z)} &in& \texttt{sparc.h}     \\
      Macro to reset  &=& \texttt{INVF4(x, y, z)} &in& \texttt{sparc.h}     \\
      x &=& 0x2      &in& \texttt{OP(x)  /* ((x) \& 0x3)  $<<$ 30 */} \\
      y &=& 0x0A     &in& \texttt{OP3(y) /* ((y) \& 0x3f) $<<$ 19 */} \\
      z &=& 0x0      &in& \texttt{F3I(z) /* ((z) \& 0x1)  $<<$ 13 */} \\
      a &=& 0x1      &in& \texttt{OP\_AJIT\_BIT(a) /* ((a) \& 0x1)  $<<$ 5 */}
    \end{tabular}

    The AJIT bit  (insn[5]) is set internally by  \texttt{F4}, and hence
    there are only three arguments.

  \item \textbf{UMULDCC}:\\
    \begin{tabular}[h]{lclcl}
      Macro to set  &=& \texttt{F4(x, y, z)} &in& \texttt{sparc.h}     \\
      Macro to reset  &=& \texttt{INVF4(x, y, z)} &in& \texttt{sparc.h}     \\
      x &=& 0x2      &in& \texttt{OP(x)  /* ((x) \& 0x3)  $<<$ 30 */} \\
      y &=& 0x1A     &in& \texttt{OP3(y) /* ((y) \& 0x3f) $<<$ 19 */} \\
      z &=& 0x0      &in& \texttt{F3I(z) /* ((z) \& 0x1)  $<<$ 13 */} \\
      a &=& 0x1      &in& \texttt{OP\_AJIT\_BIT(a) /* ((a) \& 0x1)  $<<$ 5 */}
    \end{tabular}

    The AJIT bit  (insn[5]) is set internally by  \texttt{F4}, and hence
    there are only three arguments.

  \item \textbf{SMULD}: Unsigned Integer Multiply AJIT, no immediate
    version (i.e. i is always 0).\\
    \begin{tabular}[h]{lclcl}
      Macro to set  &=& \texttt{F4(x, y, z)} &in& \texttt{sparc.h}     \\
      Macro to reset  &=& \texttt{INVF4(x, y, z)} &in& \texttt{sparc.h}     \\
      x &=& 0x2      &in& \texttt{OP(x)  /* ((x) \& 0x3)  $<<$ 30 */} \\
      y &=& 0x0B     &in& \texttt{OP3(y) /* ((y) \& 0x3f) $<<$ 19 */} \\
      z &=& 0x0      &in& \texttt{F3I(z) /* ((z) \& 0x1)  $<<$ 13 */} \\
      a &=& 0x1      &in& \texttt{OP\_AJIT\_BIT(a) /* ((a) \& 0x1)  $<<$ 5 */}
    \end{tabular}

    The AJIT bit  (insn[5]) is set internally by  \texttt{F4}, and hence
    there are only three arguments.

  \item \textbf{SMULDCC}:\\
    \begin{tabular}[h]{lclcl}
      Macro to set  &=& \texttt{F4(x, y, z)} &in& \texttt{sparc.h}     \\
      Macro to reset  &=& \texttt{INVF4(x, y, z)} &in& \texttt{sparc.h}     \\
      x &=& 0x2      &in& \texttt{OP(x)  /* ((x) \& 0x3)  $<<$ 30 */} \\
      y &=& 0x1B     &in& \texttt{OP3(y) /* ((y) \& 0x3f) $<<$ 19 */} \\
      z &=& 0x0      &in& \texttt{F3I(z) /* ((z) \& 0x1)  $<<$ 13 */} \\
      a &=& 0x1      &in& \texttt{OP\_AJIT\_BIT(a) /* ((a) \& 0x1)  $<<$ 5 */}
    \end{tabular}

    The AJIT bit  (insn[5]) is set internally by  \texttt{F4}, and hence
    there are only three arguments.

  \item \textbf{UDIVD}:\\
    \begin{tabular}[h]{lclcl}
      Macro to set  &=& \texttt{F4(x, y, z)} &in& \texttt{sparc.h}     \\
      Macro to reset  &=& \texttt{INVF4(x, y, z)} &in& \texttt{sparc.h}     \\
      x &=& 0x2      &in& \texttt{OP(x)  /* ((x) \& 0x3)  $<<$ 30 */} \\
      y &=& 0x0E     &in& \texttt{OP3(y) /* ((y) \& 0x3f) $<<$ 19 */} \\
      z &=& 0x0      &in& \texttt{F3I(z) /* ((z) \& 0x1)  $<<$ 13 */} \\
      a &=& 0x1      &in& \texttt{OP\_AJIT\_BIT(a) /* ((a) \& 0x1)  $<<$ 5 */}
    \end{tabular}

    The AJIT bit  (insn[5]) is set internally by  \texttt{F4}, and hence
    there are only three arguments.

  \item \textbf{UDIVDCC}:\\
    \begin{tabular}[h]{lclcl}
      Macro to set  &=& \texttt{F4(x, y, z)} &in& \texttt{sparc.h}     \\
      Macro to reset  &=& \texttt{INVF4(x, y, z)} &in& \texttt{sparc.h}     \\
      x &=& 0x2      &in& \texttt{OP(x)  /* ((x) \& 0x3)  $<<$ 30 */} \\
      y &=& 0x1E     &in& \texttt{OP3(y) /* ((y) \& 0x3f) $<<$ 19 */} \\
      z &=& 0x0      &in& \texttt{F3I(z) /* ((z) \& 0x1)  $<<$ 13 */} \\
      a &=& 0x1      &in& \texttt{OP\_AJIT\_BIT(a) /* ((a) \& 0x1)  $<<$ 5 */}
    \end{tabular}

    The AJIT bit  (insn[5]) is set internally by  \texttt{F4}, and hence
    there are only three arguments.

  \item \textbf{SDIVD}:\\
    \begin{tabular}[h]{lclcl}
      Macro to set  &=& \texttt{F4(x, y, z)} &in& \texttt{sparc.h}     \\
      Macro to reset  &=& \texttt{INVF4(x, y, z)} &in& \texttt{sparc.h}     \\
      x &=& 0x2      &in& \texttt{OP(x)  /* ((x) \& 0x3)  $<<$ 30 */} \\
      y &=& 0x0F     &in& \texttt{OP3(y) /* ((y) \& 0x3f) $<<$ 19 */} \\
      z &=& 0x0      &in& \texttt{F3I(z) /* ((z) \& 0x1)  $<<$ 13 */} \\
      a &=& 0x1      &in& \texttt{OP\_AJIT\_BIT(a) /* ((a) \& 0x1)  $<<$ 5 */}
    \end{tabular}

    The AJIT bit  (insn[5]) is set internally by  \texttt{F4}, and hence
    there are only three arguments.

  \item \textbf{SDIVDCC}:\\
    \begin{tabular}[h]{lclcl}
      Macro to set  &=& \texttt{F4(x, y, z)} &in& \texttt{sparc.h}     \\
      Macro to reset  &=& \texttt{INVF4(x, y, z)} &in& \texttt{sparc.h}     \\
      x &=& 0x2      &in& \texttt{OP(x)  /* ((x) \& 0x3)  $<<$ 30 */} \\
      y &=& 0x1F     &in& \texttt{OP3(y) /* ((y) \& 0x3f) $<<$ 19 */} \\
      z &=& 0x0      &in& \texttt{F3I(z) /* ((z) \& 0x1)  $<<$ 13 */} \\
      a &=& 0x1      &in& \texttt{OP\_AJIT\_BIT(a) /* ((a) \& 0x1)  $<<$ 5 */}
    \end{tabular}

    The AJIT bit  (insn[5]) is set internally by  \texttt{F4}, and hence
    there are only three arguments.
  \end{enumerate}

  % \subsubsection{64 Bit Logical Instructions:}
\label{sec:64:bit:logical:insn:impl}

No immediate mode, i.e. insn[5] $\equiv$ i = 0, always.

\begin{enumerate}
\item \textbf{ORD}:\\
  \begin{center}
    \begin{tabular}[p]{|c|c|l|l|l|}
      \hline
      \textbf{Start} & \textbf{End} & \textbf{Range} & \textbf{Meaning} &
                                                                          \textbf{New Meaning}\\
      \hline
      0 & 4 & 32 & Source register 2, rs2 & No change \\
      5 & 12 & -- & \textbf{unused} & \textbf{Set bit 5 to ``1''} \\
      13 & 13 & 0,1 & The \textbf{i} bit & \textbf{Set i to ``0''} \\
      14 & 18 & 32 & Source register 1, rs1 & No change \\
      19 & 24 & 000010 & ``\textbf{op3}'' & No change \\
      25 & 29 & 32 & Destination register, rd & No change \\
      30 & 31 & 4 & Always ``10'' & No change \\
      \hline
    \end{tabular}
  \end{center}
  \begin{itemize}
  \item []\textbf{ORD}: same as OR, but with Instr[13]=0 (i=0), and
    Instr[5]=1.
  \item []\textbf{Syntax}: ``\texttt{ord  SrcReg1, SrcReg2, DestReg}''.
  \item []\textbf{Semantics}: rd(pair) $\leftarrow$ rs1(pair) $\vert$ rs2(pair).
  \end{itemize}
  Bits layout:
\begin{verbatim}
    Offsets      : 31       24 23       16  15        8   7        0
    Bit layout   :  XXXX  XXXX  XXXX  XXXX   XXXX  XXXX   XXXX  XXXX
    Insn Bits    :  10       0  0001  0        0            1       
    Destination  :    DD  DDD                                       
    Source 1     :                     SSS   SS
    Source 2     :                                           S  SSSS
    Unused (0)   :                              U  UUUU   UU        
    Final layout :  10DD  DDD0  0001  0SSS   SS0U  UUUU   UU1S  SSSS
\end{verbatim}

  Hence the SPARC bit layout of this instruction is:

  \begin{tabular}[h]{lclcl}
    Macro to set  &=& \texttt{F4(x, y, z)} &in& \texttt{sparc.h}     \\
    Macro to reset  &=& \texttt{INVF4(x, y, z)} &in& \texttt{sparc.h}     \\
    x &=& 0x2      &in& \texttt{OP(x)  /* ((x) \& 0x3)  $<<$ 30 */} \\
    y &=& 0x02     &in& \texttt{OP3(y) /* ((y) \& 0x3f) $<<$ 19 */} \\
    z &=& 0x0      &in& \texttt{F3I(z) /* ((z) \& 0x1)  $<<$ 13 */} \\
    a &=& 0x1      &in& \texttt{OP\_AJIT\_BIT(a) /* ((a) \& 0x1)  $<<$ 5 */}
  \end{tabular}

  The AJIT bit  (insn[5]) is set internally by  \texttt{F4}, and hence
  there are only three arguments.

\item \textbf{ORDCC}:\\
  \begin{center}
    \begin{tabular}[p]{|c|c|l|l|l|}
      \hline
      \textbf{Start} & \textbf{End} & \textbf{Range} & \textbf{Meaning} &
                                                                          \textbf{New Meaning}\\
      \hline
      0 & 4 & 32 & Source register 2, rs2 & No change \\
      5 & 12 & -- & \textbf{unused} & \textbf{Set bit 5 to ``1''} \\
      13 & 13 & 0,1 & The \textbf{i} bit & \textbf{Set i to ``0''} \\
      14 & 18 & 32 & Source register 1, rs1 & No change \\
      19 & 24 & 010010 & ``\textbf{op3}'' & No change \\
      25 & 29 & 32 & Destination register, rd & No change \\
      30 & 31 & 4 & Always ``10'' & No change \\
      \hline
    \end{tabular}
  \end{center}
  \begin{itemize}
  \item []\textbf{ORDCC}: same as ORCC, but with Instr[13]=0 (i=0), and
    Instr[5]=1.
  \item []\textbf{Syntax}: ``\texttt{ordcc  SrcReg1, SrcReg2, DestReg}''.
  \item []\textbf{Semantics}: rd(pair) $\leftarrow$ rs1(pair) $\vert$
    rs2(pair), sets Z.
  \end{itemize}
  Bits layout:
\begin{verbatim}
    Offsets      : 31       24 23       16  15        8   7        0
    Bit layout   :  XXXX  XXXX  XXXX  XXXX   XXXX  XXXX   XXXX  XXXX
    Insn Bits    :  10       0  1001  0        0            1       
    Destination  :    DD  DDD                                       
    Source 1     :                     SSS   SS
    Source 2     :                                           S  SSSS
    Unused (0)   :                              U  UUUU   UU        
    Final layout :  10DD  DDD0  1001  0SSS   SS0U  UUUU   UU1S  SSSS
\end{verbatim}

  Hence the SPARC bit layout of this instruction is:

  \begin{tabular}[h]{lclcl}
    Macro to set  &=& \texttt{F4(x, y, z)} &in& \texttt{sparc.h}     \\
    Macro to reset  &=& \texttt{INVF4(x, y, z)} &in& \texttt{sparc.h}     \\
    x &=& 0x2      &in& \texttt{OP(x)  /* ((x) \& 0x3)  $<<$ 30 */} \\
    y &=& 0x12     &in& \texttt{OP3(y) /* ((y) \& 0x3f) $<<$ 19 */} \\
    z &=& 0x0      &in& \texttt{F3I(z) /* ((z) \& 0x1)  $<<$ 13 */} \\
    a &=& 0x1      &in& \texttt{OP\_AJIT\_BIT(a) /* ((a) \& 0x1)  $<<$ 5 */}
  \end{tabular}

  The AJIT bit  (insn[5]) is set internally by  \texttt{F4}, and hence
  there are only three arguments.

\item \textbf{ORDN}:\\
  \begin{center}
    \begin{tabular}[p]{|c|c|l|l|l|}
      \hline
      \textbf{Start} & \textbf{End} & \textbf{Range} & \textbf{Meaning} &
                                                                          \textbf{New Meaning}\\
      \hline
      0 & 4 & 32 & Source register 2, rs2 & No change \\
      5 & 12 & -- & \textbf{unused} & \textbf{Set bit 5 to ``1''} \\
      13 & 13 & 0,1 & The \textbf{i} bit & \textbf{Set i to ``0''} \\
      14 & 18 & 32 & Source register 1, rs1 & No change \\
      19 & 24 & 000110 & ``\textbf{op3}'' & No change \\
      25 & 29 & 32 & Destination register, rd & No change \\
      30 & 31 & 4 & Always ``10'' & No change \\
      \hline
    \end{tabular}
  \end{center}
  \begin{itemize}
  \item []\textbf{ORDN}: same as ORN, but with Instr[13]=0 (i=0), and
    Instr[5]=1.
  \item []\textbf{Syntax}: ``\texttt{ordn  SrcReg1, SrcReg2, DestReg}''.
  \item []\textbf{Semantics}: rd(pair) $\leftarrow$ rs1(pair) $\vert$ ($\sim$rs2(pair)).
  \end{itemize}
  Bits layout:
\begin{verbatim}
    Offsets      : 31       24 23       16  15        8   7        0
    Bit layout   :  XXXX  XXXX  XXXX  XXXX   XXXX  XXXX   XXXX  XXXX
    Insn Bits    :  10       0  0011  0        0            1       
    Destination  :    DD  DDD                                       
    Source 1     :                     SSS   SS
    Source 2     :                                           S  SSSS
    Unused (0)   :                              U  UUUU   UU        
    Final layout :  10DD  DDD0  0011  0SSS   SS0U  UUUU   UU1S  SSSS
\end{verbatim}

  Hence the SPARC bit layout of this instruction is:

  \begin{tabular}[h]{lclcl}
    Macro to set  &=& \texttt{F4(x, y, z)} &in& \texttt{sparc.h}     \\
    Macro to reset  &=& \texttt{INVF4(x, y, z)} &in& \texttt{sparc.h}     \\
    x &=& 0x2      &in& \texttt{OP(x)  /* ((x) \& 0x3)  $<<$ 30 */} \\
    y &=& 0x06     &in& \texttt{OP3(y) /* ((y) \& 0x3f) $<<$ 19 */} \\
    z &=& 0x0      &in& \texttt{F3I(z) /* ((z) \& 0x1)  $<<$ 13 */} \\
    a &=& 0x1      &in& \texttt{OP\_AJIT\_BIT(a) /* ((a) \& 0x1)  $<<$ 5 */}
  \end{tabular}

  The AJIT bit  (insn[5]) is set internally by  \texttt{F4}, and hence
  there are only three arguments.

\item \textbf{ORDNCC}:\\
  \begin{center}
    \begin{tabular}[p]{|c|c|l|l|l|}
      \hline
      \textbf{Start} & \textbf{End} & \textbf{Range} & \textbf{Meaning} &
                                                                          \textbf{New Meaning}\\
      \hline
      0 & 4 & 32 & Source register 2, rs2 & No change \\
      5 & 12 & -- & \textbf{unused} & \textbf{Set bit 5 to ``1''} \\
      13 & 13 & 0,1 & The \textbf{i} bit & \textbf{Set i to ``0''} \\
      14 & 18 & 32 & Source register 1, rs1 & No change \\
      19 & 24 & 010110 & ``\textbf{op3}'' & No change \\
      25 & 29 & 32 & Destination register, rd & No change \\
      30 & 31 & 4 & Always ``10'' & No change \\
      \hline
    \end{tabular}
  \end{center}
  \begin{itemize}
  \item []\textbf{ORDNCC}: same as ORN, but with Instr[13]=0 (i=0), and
    Instr[5]=1.
  \item []\textbf{Syntax}: ``\texttt{ordncc  SrcReg1, SrcReg2, DestReg}''.
  \item []\textbf{Semantics}: rd(pair) $\leftarrow$ rs1(pair) $\vert$
    ($\sim$rs2(pair)), sets Z.
  \end{itemize}
  Bits layout:
\begin{verbatim}
    Offsets      : 31       24 23       16  15        8   7        0
    Bit layout   :  XXXX  XXXX  XXXX  XXXX   XXXX  XXXX   XXXX  XXXX
    Insn Bits    :  10       0  1011  0        0            1       
    Destination  :    DD  DDD                                       
    Source 1     :                     SSS   SS
    Source 2     :                                           S  SSSS
    Unused (0)   :                              U  UUUU   UU        
    Final layout :  10DD  DDD0  0011  0SSS   SS0U  UUUU   UU1S  SSSS
\end{verbatim}

  Hence the SPARC bit layout of this instruction is:

  \begin{tabular}[h]{lclcl}
    Macro to set  &=& \texttt{F4(x, y, z)} &in& \texttt{sparc.h}     \\
    Macro to reset  &=& \texttt{INVF4(x, y, z)} &in& \texttt{sparc.h}     \\
    x &=& 0x2      &in& \texttt{OP(x)  /* ((x) \& 0x3)  $<<$ 30 */} \\
    y &=& 0x16     &in& \texttt{OP3(y) /* ((y) \& 0x3f) $<<$ 19 */} \\
    z &=& 0x0      &in& \texttt{F3I(z) /* ((z) \& 0x1)  $<<$ 13 */} \\
    a &=& 0x1      &in& \texttt{OP\_AJIT\_BIT(a) /* ((a) \& 0x1)  $<<$ 5 */}
  \end{tabular}

  The AJIT bit  (insn[5]) is set internally by  \texttt{F4}, and hence
  there are only three arguments.

\item \textbf{XORDCC}:\\
  \begin{center}
    \begin{tabular}[p]{|c|c|l|l|l|}
      \hline
      \textbf{Start} & \textbf{End} & \textbf{Range} & \textbf{Meaning} &
                                                                          \textbf{New Meaning}\\
      \hline
      0 & 4 & 32 & Source register 2, rs2 & No change \\
      5 & 12 & -- & \textbf{unused} & \textbf{Set bit 5 to ``1''} \\
      13 & 13 & 0,1 & The \textbf{i} bit & \textbf{Set i to ``0''} \\
      14 & 18 & 32 & Source register 1, rs1 & No change \\
      19 & 24 & 010011 & ``\textbf{op3}'' & No change \\
      25 & 29 & 32 & Destination register, rd & No change \\
      30 & 31 & 4 & Always ``10'' & No change \\
      \hline
    \end{tabular}
  \end{center}
  \begin{itemize}
  \item []\textbf{XORDCC}: same as XORCC, but with Instr[13]=0 (i=0), and
    Instr[5]=1.
  \item []\textbf{Syntax}: ``\texttt{xordcc  SrcReg1, SrcReg2, DestReg}''.
  \item []\textbf{Semantics}: rd(pair) $\leftarrow$ rs1(pair) $\hat{~}$
    rs2(pair), sets Z.
  \end{itemize}
  Bits layout:
\begin{verbatim}
    Offsets      : 31       24 23       16  15        8   7        0
    Bit layout   :  XXXX  XXXX  XXXX  XXXX   XXXX  XXXX   XXXX  XXXX
    Insn Bits    :  10       0  1001  1        0            1       
    Destination  :    DD  DDD                                       
    Source 1     :                     SSS   SS
    Source 2     :                                           S  SSSS
    Unused (0)   :                              U  UUUU   UU        
    Final layout :  10DD  DDD0  1001  1SSS   SS0U  UUUU   UU1S  SSSS
\end{verbatim}

  Hence the SPARC bit layout of this instruction is:

  \begin{tabular}[h]{lclcl}
    Macro to set  &=& \texttt{F4(x, y, z)} &in& \texttt{sparc.h}     \\
    Macro to reset  &=& \texttt{INVF4(x, y, z)} &in& \texttt{sparc.h}     \\
    x &=& 0x2      &in& \texttt{OP(x)  /* ((x) \& 0x3)  $<<$ 30 */} \\
    y &=& 0x13     &in& \texttt{OP3(y) /* ((y) \& 0x3f) $<<$ 19 */} \\
    z &=& 0x0      &in& \texttt{F3I(z) /* ((z) \& 0x1)  $<<$ 13 */} \\
    a &=& 0x1      &in& \texttt{OP\_AJIT\_BIT(a) /* ((a) \& 0x1)  $<<$ 5 */}
  \end{tabular}

  The AJIT bit  (insn[5]) is set internally by  \texttt{F4}, and hence
  there are only three arguments.

\item \textbf{XNORD}:\\
  \begin{center}
    \begin{tabular}[p]{|c|c|l|l|l|}
      \hline
      \textbf{Start} & \textbf{End} & \textbf{Range} & \textbf{Meaning} &
                                                                          \textbf{New Meaning}\\
      \hline
      0 & 4 & 32 & Source register 2, rs2 & No change \\
      5 & 12 & -- & \textbf{unused} & \textbf{Set bit 5 to ``1''} \\
      13 & 13 & 0,1 & The \textbf{i} bit & \textbf{Set i to ``0''} \\
      14 & 18 & 32 & Source register 1, rs1 & No change \\
      19 & 24 & 000111 & ``\textbf{op3}'' & No change \\
      25 & 29 & 32 & Destination register, rd & No change \\
      30 & 31 & 4 & Always ``10'' & No change \\
      \hline
    \end{tabular}
  \end{center}
  \begin{itemize}
  \item []\textbf{XNORD}: same as XNOR, but with Instr[13]=0 (i=0), and
    Instr[5]=1.
  \item []\textbf{Syntax}: ``\texttt{xnordcc  SrcReg1, SrcReg2, DestReg}''.
  \item []\textbf{Semantics}: rd(pair) $\leftarrow$ rs1(pair) $\hat{~}$
    rs2(pair).
  \end{itemize}
  Bits layout:
\begin{verbatim}
    Offsets      : 31       24 23       16  15        8   7        0
    Bit layout   :  XXXX  XXXX  XXXX  XXXX   XXXX  XXXX   XXXX  XXXX
    Insn Bits    :  10       0  0011  1        0            1       
    Destination  :    DD  DDD                                       
    Source 1     :                     SSS   SS
    Source 2     :                                           S  SSSS
    Unused (0)   :                              U  UUUU   UU        
    Final layout :  10DD  DDD0  0011  1SSS   SS0U  UUUU   UU1S  SSSS
\end{verbatim}

  Hence the SPARC bit layout of this instruction is:

  \begin{tabular}[h]{lclcl}
    Macro to set  &=& \texttt{F4(x, y, z)} &in& \texttt{sparc.h}     \\
    Macro to reset  &=& \texttt{INVF4(x, y, z)} &in& \texttt{sparc.h}     \\
    x &=& 0x2      &in& \texttt{OP(x)  /* ((x) \& 0x3)  $<<$ 30 */} \\
    y &=& 0x07     &in& \texttt{OP3(y) /* ((y) \& 0x3f) $<<$ 19 */} \\
    z &=& 0x0      &in& \texttt{F3I(z) /* ((z) \& 0x1)  $<<$ 13 */} \\
    a &=& 0x1      &in& \texttt{OP\_AJIT\_BIT(a) /* ((a) \& 0x1)  $<<$ 5 */}
  \end{tabular}

  The AJIT bit  (insn[5]) is set internally by  \texttt{F4}, and hence
  there are only three arguments.
  
\item \textbf{XNORDCC}:\\
  \begin{center}
    \begin{tabular}[p]{|c|c|l|l|l|}
      \hline
      \textbf{Start} & \textbf{End} & \textbf{Range} & \textbf{Meaning} &
                                                                          \textbf{New Meaning}\\
      \hline
      0 & 4 & 32 & Source register 2, rs2 & No change \\
      5 & 12 & -- & \textbf{unused} & \textbf{Set bit 5 to ``1''} \\
      13 & 13 & 0,1 & The \textbf{i} bit & \textbf{Set i to ``0''} \\
      14 & 18 & 32 & Source register 1, rs1 & No change \\
      19 & 24 & 000111 & ``\textbf{op3}'' & No change \\
      25 & 29 & 32 & Destination register, rd & No change \\
      30 & 31 & 4 & Always ``10'' & No change \\
      \hline
    \end{tabular}
  \end{center}
  \begin{itemize}
  \item []\textbf{XNORDCC}: same as XNORD, but with Instr[13]=0 (i=0), and
    Instr[5]=1.
  \item []\textbf{Syntax}: ``\texttt{xnordcc  SrcReg1, SrcReg2, DestReg}''.
  \item []\textbf{Semantics}: rd(pair) $\leftarrow$ rs1(pair) $\hat{~}$
    rs2(pair), sets Z.
  \end{itemize}
  Bits layout:
\begin{verbatim}
    Offsets      : 31       24 23       16  15        8   7        0
    Bit layout   :  XXXX  XXXX  XXXX  XXXX   XXXX  XXXX   XXXX  XXXX
    Insn Bits    :  10       0  0011  1        0            1       
    Destination  :    DD  DDD                                       
    Source 1     :                     SSS   SS
    Source 2     :                                           S  SSSS
    Unused (0)   :                              U  UUUU   UU        
    Final layout :  10DD  DDD0  0011  1SSS   SS0U  UUUU   UU1S  SSSS
\end{verbatim}

  Hence the SPARC bit layout of this instruction is:

  \begin{tabular}[h]{lclcl}
    Macro to set  &=& \texttt{F4(x, y, z)} &in& \texttt{sparc.h}     \\
    Macro to reset  &=& \texttt{INVF4(x, y, z)} &in& \texttt{sparc.h}     \\
    x &=& 0x2      &in& \texttt{OP(x)  /* ((x) \& 0x3)  $<<$ 30 */} \\
    y &=& 0x07     &in& \texttt{OP3(y) /* ((y) \& 0x3f) $<<$ 19 */} \\
    z &=& 0x0      &in& \texttt{F3I(z) /* ((z) \& 0x1)  $<<$ 13 */} \\
    a &=& 0x1      &in& \texttt{OP\_AJIT\_BIT(a) /* ((a) \& 0x1)  $<<$ 5 */}
  \end{tabular}

  The AJIT bit  (insn[5]) is set internally by  \texttt{F4}, and hence
  there are only three arguments.
  
\item \textbf{ANDD}:\\
  \begin{center}
    \begin{tabular}[p]{|c|c|l|l|l|}
      \hline
      \textbf{Start} & \textbf{End} & \textbf{Range} & \textbf{Meaning} &
                                                                          \textbf{New Meaning}\\
      \hline
      0 & 4 & 32 & Source register 2, rs2 & No change \\
      5 & 12 & -- & \textbf{unused} & \textbf{Set bit 5 to ``1''} \\
      13 & 13 & 0,1 & The \textbf{i} bit & \textbf{Set i to ``0''} \\
      14 & 18 & 32 & Source register 1, rs1 & No change \\
      19 & 24 & 000001 & ``\textbf{op3}'' & No change \\
      25 & 29 & 32 & Destination register, rd & No change \\
      30 & 31 & 4 & Always ``10'' & No change \\
      \hline
    \end{tabular}
  \end{center}
  \begin{itemize}
  \item []\textbf{ANDD}: same as AND, but with Instr[13]=0 (i=0), and
    Instr[5]=1.
  \item []\textbf{Syntax}: ``\texttt{andd  SrcReg1, SrcReg2, DestReg}''.
  \item []\textbf{Semantics}: rd(pair) $\leftarrow$ rs1(pair) $\cdot$ rs2(pair).
  \end{itemize}
  Bits layout:
\begin{verbatim}
    Offsets      : 31       24 23       16  15        8   7        0
    Bit layout   :  XXXX  XXXX  XXXX  XXXX   XXXX  XXXX   XXXX  XXXX
    Insn Bits    :  10       0  0000  1        0            1       
    Destination  :    DD  DDD                                       
    Source 1     :                     SSS   SS
    Source 2     :                                           S  SSSS
    Unused (0)   :                              U  UUUU   UU        
    Final layout :  10DD  DDD0  0000  1SSS   SS0U  UUUU   UU1S  SSSS
\end{verbatim}

  Hence the SPARC bit layout of this instruction is:

  \begin{tabular}[h]{lclcl}
    Macro to set  &=& \texttt{F4(x, y, z)} &in& \texttt{sparc.h}     \\
    Macro to reset  &=& \texttt{INVF4(x, y, z)} &in& \texttt{sparc.h}     \\
    x &=& 0x2      &in& \texttt{OP(x)  /* ((x) \& 0x3)  $<<$ 30 */} \\
    y &=& 0x01     &in& \texttt{OP3(y) /* ((y) \& 0x3f) $<<$ 19 */} \\
    z &=& 0x0      &in& \texttt{F3I(z) /* ((z) \& 0x1)  $<<$ 13 */} \\
    a &=& 0x1      &in& \texttt{OP\_AJIT\_BIT(a) /* ((a) \& 0x1)  $<<$ 5 */}
  \end{tabular}

  The AJIT bit  (insn[5]) is set internally by  \texttt{F4}, and hence
  there are only three arguments.

\item \textbf{ANDDCC}:\\
  \begin{center}
    \begin{tabular}[p]{|c|c|l|l|l|}
      \hline
      \textbf{Start} & \textbf{End} & \textbf{Range} & \textbf{Meaning} &
                                                                          \textbf{New Meaning}\\
      \hline
      0 & 4 & 32 & Source register 2, rs2 & No change \\
      5 & 12 & -- & \textbf{unused} & \textbf{Set bit 5 to ``1''} \\
      13 & 13 & 0,1 & The \textbf{i} bit & \textbf{Set i to ``0''} \\
      14 & 18 & 32 & Source register 1, rs1 & No change \\
      19 & 24 & 010001 & ``\textbf{op3}'' & No change \\
      25 & 29 & 32 & Destination register, rd & No change \\
      30 & 31 & 4 & Always ``10'' & No change \\
      \hline
    \end{tabular}
  \end{center}
  \begin{itemize}
  \item []\textbf{ANDDCC}: same as ANDCC, but with Instr[13]=0 (i=0), and
    Instr[5]=1.
  \item []\textbf{Syntax}: ``\texttt{anddcc  SrcReg1, SrcReg2, DestReg}''.
  \item []\textbf{Semantics}: rd(pair) $\leftarrow$ rs1(pair) $\cdot$
    rs2(pair), sets Z.
  \end{itemize}
  Bits layout:
\begin{verbatim}
    Offsets      : 31       24 23       16  15        8   7        0
    Bit layout   :  XXXX  XXXX  XXXX  XXXX   XXXX  XXXX   XXXX  XXXX
    Insn Bits    :  10       0  1000  1        0            1       
    Destination  :    DD  DDD                                       
    Source 1     :                     SSS   SS
    Source 2     :                                           S  SSSS
    Unused (0)   :                              U  UUUU   UU        
    Final layout :  10DD  DDD0  1000  1SSS   SS0U  UUUU   UU1S  SSSS
\end{verbatim}

  Hence the SPARC bit layout of this instruction is:

  \begin{tabular}[h]{lclcl}
    Macro to set  &=& \texttt{F4(x, y, z)} &in& \texttt{sparc.h}     \\
    Macro to reset  &=& \texttt{INVF4(x, y, z)} &in& \texttt{sparc.h}     \\
    x &=& 0x2      &in& \texttt{OP(x)  /* ((x) \& 0x3)  $<<$ 30 */} \\
    y &=& 0x11     &in& \texttt{OP3(y) /* ((y) \& 0x3f) $<<$ 19 */} \\
    z &=& 0x0      &in& \texttt{F3I(z) /* ((z) \& 0x1)  $<<$ 13 */} \\
    a &=& 0x1      &in& \texttt{OP\_AJIT\_BIT(a) /* ((a) \& 0x1)  $<<$ 5 */}
  \end{tabular}

  The AJIT bit  (insn[5]) is set internally by  \texttt{F4}, and hence
  there are only three arguments.

\item \textbf{ANDDN}:\\
  \begin{center}
    \begin{tabular}[p]{|c|c|l|l|l|}
      \hline
      \textbf{Start} & \textbf{End} & \textbf{Range} & \textbf{Meaning} &
                                                                          \textbf{New Meaning}\\
      \hline
      0 & 4 & 32 & Source register 2, rs2 & No change \\
      5 & 12 & -- & \textbf{unused} & \textbf{Set bit 5 to ``1''} \\
      13 & 13 & 0,1 & The \textbf{i} bit & \textbf{Set i to ``0''} \\
      14 & 18 & 32 & Source register 1, rs1 & No change \\
      19 & 24 & 000101 & ``\textbf{op3}'' & No change \\
      25 & 29 & 32 & Destination register, rd & No change \\
      30 & 31 & 4 & Always ``10'' & No change \\
      \hline
    \end{tabular}
  \end{center}
  \begin{itemize}
  \item []\textbf{ANDDN}: same as ANDN, but with Instr[13]=0 (i=0), and
    Instr[5]=1.
  \item []\textbf{Syntax}: ``\texttt{anddn  SrcReg1, SrcReg2, DestReg}''.
  \item []\textbf{Semantics}: rd(pair) $\leftarrow$ rs1(pair) $\cdot$ ($\sim$rs2(pair)).
  \end{itemize}
  Bits layout:
\begin{verbatim}
    Offsets      : 31       24 23       16  15        8   7        0
    Bit layout   :  XXXX  XXXX  XXXX  XXXX   XXXX  XXXX   XXXX  XXXX
    Insn Bits    :  10       0  0010  1        0            1       
    Destination  :    DD  DDD                                       
    Source 1     :                     SSS   SS
    Source 2     :                                           S  SSSS
    Unused (0)   :                              U  UUUU   UU        
    Final layout :  10DD  DDD0  0010  1SSS   SS0U  UUUU   UU1S  SSSS
\end{verbatim}

  Hence the SPARC bit layout of this instruction is:

  \begin{tabular}[h]{lclcl}
    Macro to set  &=& \texttt{F4(x, y, z)} &in& \texttt{sparc.h}     \\
    Macro to reset  &=& \texttt{INVF4(x, y, z)} &in& \texttt{sparc.h}     \\
    x &=& 0x2      &in& \texttt{OP(x)  /* ((x) \& 0x3)  $<<$ 30 */} \\
    y &=& 0x05     &in& \texttt{OP3(y) /* ((y) \& 0x3f) $<<$ 19 */} \\
    z &=& 0x0      &in& \texttt{F3I(z) /* ((z) \& 0x1)  $<<$ 13 */} \\
    a &=& 0x1      &in& \texttt{OP\_AJIT\_BIT(a) /* ((a) \& 0x1)  $<<$ 5 */}
  \end{tabular}

  The AJIT bit  (insn[5]) is set internally by  \texttt{F4}, and hence
  there are only three arguments.

\item \textbf{ANDDNCC}:\\
  \begin{center}
    \begin{tabular}[p]{|c|c|l|l|l|}
      \hline
      \textbf{Start} & \textbf{End} & \textbf{Range} & \textbf{Meaning} &
                                                                          \textbf{New Meaning}\\
      \hline
      0 & 4 & 32 & Source register 2, rs2 & No change \\
      5 & 12 & -- & \textbf{unused} & \textbf{Set bit 5 to ``1''} \\
      13 & 13 & 0,1 & The \textbf{i} bit & \textbf{Set i to ``0''} \\
      14 & 18 & 32 & Source register 1, rs1 & No change \\
      19 & 24 & 010101 & ``\textbf{op3}'' & No change \\
      25 & 29 & 32 & Destination register, rd & No change \\
      30 & 31 & 4 & Always ``10'' & No change \\
      \hline
    \end{tabular}
  \end{center}
  \begin{itemize}
  \item []\textbf{ANDDNCC}: same as ANDN, but with Instr[13]=0 (i=0), and
    Instr[5]=1.
  \item []\textbf{Syntax}: ``\texttt{anddncc  SrcReg1, SrcReg2, DestReg}''.
  \item []\textbf{Semantics}: rd(pair) $\leftarrow$ rs1(pair) $\cdot$
    ($\sim$rs2(pair)), sets Z.
  \end{itemize}
  Bits layout:
\begin{verbatim}
    Offsets      : 31       24 23       16  15        8   7        0
    Bit layout   :  XXXX  XXXX  XXXX  XXXX   XXXX  XXXX   XXXX  XXXX
    Insn Bits    :  10       0  1010  1        0            1       
    Destination  :    DD  DDD                                       
    Source 1     :                     SSS   SS
    Source 2     :                                           S  SSSS
    Unused (0)   :                              U  UUUU   UU        
    Final layout :  10DD  DDD0  0010  1SSS   SS0U  UUUU   UU1S  SSSS
\end{verbatim}

  Hence the SPARC bit layout of this instruction is:

  \begin{tabular}[h]{lclcl}
    Macro to set  &=& \texttt{F4(x, y, z)} &in& \texttt{sparc.h}     \\
    Macro to reset  &=& \texttt{INVF4(x, y, z)} &in& \texttt{sparc.h}     \\
    x &=& 0x2      &in& \texttt{OP(x)  /* ((x) \& 0x3)  $<<$ 30 */} \\
    y &=& 0x15     &in& \texttt{OP3(y) /* ((y) \& 0x3f) $<<$ 19 */} \\
    z &=& 0x0      &in& \texttt{F3I(z) /* ((z) \& 0x1)  $<<$ 13 */} \\
    a &=& 0x1      &in& \texttt{OP\_AJIT\_BIT(a) /* ((a) \& 0x1)  $<<$ 5 */}
  \end{tabular}

  The AJIT bit  (insn[5]) is set internally by  \texttt{F4}, and hence
  there are only three arguments.

\end{enumerate}

%%% Local Variables:
%%% mode: latex
%%% TeX-master: t
%%% End:

\item {64 Bit Logical Instructions:}\\

  No immediate mode, i.e. insn[5] $\equiv$ i = 0, always.

  \begin{enumerate}
  \item \textbf{ORD}:\\
    \begin{tabular}[h]{lclcl}
      Macro to set  &=& \texttt{F4(x, y, z)} &in& \texttt{sparc.h}     \\
      Macro to reset  &=& \texttt{INVF4(x, y, z)} &in& \texttt{sparc.h}     \\
      x &=& 0x2      &in& \texttt{OP(x)  /* ((x) \& 0x3)  $<<$ 30 */} \\
      y &=& 0x02     &in& \texttt{OP3(y) /* ((y) \& 0x3f) $<<$ 19 */} \\
      z &=& 0x0      &in& \texttt{F3I(z) /* ((z) \& 0x1)  $<<$ 13 */} \\
      a &=& 0x1      &in& \texttt{OP\_AJIT\_BIT(a) /* ((a) \& 0x1)  $<<$ 5 */}
    \end{tabular}

    The AJIT bit  (insn[5]) is set internally by  \texttt{F4}, and hence
    there are only three arguments.

  \item \textbf{ORDCC}:\\
    \begin{tabular}[h]{lclcl}
      Macro to set  &=& \texttt{F4(x, y, z)} &in& \texttt{sparc.h}     \\
      Macro to reset  &=& \texttt{INVF4(x, y, z)} &in& \texttt{sparc.h}     \\
      x &=& 0x2      &in& \texttt{OP(x)  /* ((x) \& 0x3)  $<<$ 30 */} \\
      y &=& 0x12     &in& \texttt{OP3(y) /* ((y) \& 0x3f) $<<$ 19 */} \\
      z &=& 0x0      &in& \texttt{F3I(z) /* ((z) \& 0x1)  $<<$ 13 */} \\
      a &=& 0x1      &in& \texttt{OP\_AJIT\_BIT(a) /* ((a) \& 0x1)  $<<$ 5 */}
    \end{tabular}

    The AJIT bit  (insn[5]) is set internally by  \texttt{F4}, and hence
    there are only three arguments.

  \item \textbf{ORDN}:\\
    \begin{tabular}[h]{lclcl}
      Macro to set  &=& \texttt{F4(x, y, z)} &in& \texttt{sparc.h}     \\
      Macro to reset  &=& \texttt{INVF4(x, y, z)} &in& \texttt{sparc.h}     \\
      x &=& 0x2      &in& \texttt{OP(x)  /* ((x) \& 0x3)  $<<$ 30 */} \\
      y &=& 0x06     &in& \texttt{OP3(y) /* ((y) \& 0x3f) $<<$ 19 */} \\
      z &=& 0x0      &in& \texttt{F3I(z) /* ((z) \& 0x1)  $<<$ 13 */} \\
      a &=& 0x1      &in& \texttt{OP\_AJIT\_BIT(a) /* ((a) \& 0x1)  $<<$ 5 */}
    \end{tabular}

    The AJIT bit  (insn[5]) is set internally by  \texttt{F4}, and hence
    there are only three arguments.

  \item \textbf{ORDNCC}:\\
    \begin{tabular}[h]{lclcl}
      Macro to set  &=& \texttt{F4(x, y, z)} &in& \texttt{sparc.h}     \\
      Macro to reset  &=& \texttt{INVF4(x, y, z)} &in& \texttt{sparc.h}     \\
      x &=& 0x2      &in& \texttt{OP(x)  /* ((x) \& 0x3)  $<<$ 30 */} \\
      y &=& 0x16     &in& \texttt{OP3(y) /* ((y) \& 0x3f) $<<$ 19 */} \\
      z &=& 0x0      &in& \texttt{F3I(z) /* ((z) \& 0x1)  $<<$ 13 */} \\
      a &=& 0x1      &in& \texttt{OP\_AJIT\_BIT(a) /* ((a) \& 0x1)  $<<$ 5 */}
    \end{tabular}

    The AJIT bit  (insn[5]) is set internally by  \texttt{F4}, and hence
    there are only three arguments.

  \item \textbf{XORDCC}:\\
    \begin{tabular}[h]{lclcl}
      Macro to set  &=& \texttt{F4(x, y, z)} &in& \texttt{sparc.h}     \\
      Macro to reset  &=& \texttt{INVF4(x, y, z)} &in& \texttt{sparc.h}     \\
      x &=& 0x2      &in& \texttt{OP(x)  /* ((x) \& 0x3)  $<<$ 30 */} \\
      y &=& 0x13     &in& \texttt{OP3(y) /* ((y) \& 0x3f) $<<$ 19 */} \\
      z &=& 0x0      &in& \texttt{F3I(z) /* ((z) \& 0x1)  $<<$ 13 */} \\
      a &=& 0x1      &in& \texttt{OP\_AJIT\_BIT(a) /* ((a) \& 0x1)  $<<$ 5 */}
    \end{tabular}

    The AJIT bit  (insn[5]) is set internally by  \texttt{F4}, and hence
    there are only three arguments.

  \item \textbf{XNORD}:\\
    \begin{tabular}[h]{lclcl}
      Macro to set  &=& \texttt{F4(x, y, z)} &in& \texttt{sparc.h}     \\
      Macro to reset  &=& \texttt{INVF4(x, y, z)} &in& \texttt{sparc.h}     \\
      x &=& 0x2      &in& \texttt{OP(x)  /* ((x) \& 0x3)  $<<$ 30 */} \\
      y &=& 0x07     &in& \texttt{OP3(y) /* ((y) \& 0x3f) $<<$ 19 */} \\
      z &=& 0x0      &in& \texttt{F3I(z) /* ((z) \& 0x1)  $<<$ 13 */} \\
      a &=& 0x1      &in& \texttt{OP\_AJIT\_BIT(a) /* ((a) \& 0x1)  $<<$ 5 */}
    \end{tabular}

    The AJIT bit  (insn[5]) is set internally by  \texttt{F4}, and hence
    there are only three arguments.
    
  \item \textbf{XNORDCC}:\\
    \begin{tabular}[h]{lclcl}
      Macro to set  &=& \texttt{F4(x, y, z)} &in& \texttt{sparc.h}     \\
      Macro to reset  &=& \texttt{INVF4(x, y, z)} &in& \texttt{sparc.h}     \\
      x &=& 0x2      &in& \texttt{OP(x)  /* ((x) \& 0x3)  $<<$ 30 */} \\
      y &=& 0x07     &in& \texttt{OP3(y) /* ((y) \& 0x3f) $<<$ 19 */} \\
      z &=& 0x0      &in& \texttt{F3I(z) /* ((z) \& 0x1)  $<<$ 13 */} \\
      a &=& 0x1      &in& \texttt{OP\_AJIT\_BIT(a) /* ((a) \& 0x1)  $<<$ 5 */}
    \end{tabular}

    The AJIT bit  (insn[5]) is set internally by  \texttt{F4}, and hence
    there are only three arguments.
    
  \item \textbf{ANDD}:\\
    \begin{tabular}[h]{lclcl}
      Macro to set  &=& \texttt{F4(x, y, z)} &in& \texttt{sparc.h}     \\
      Macro to reset  &=& \texttt{INVF4(x, y, z)} &in& \texttt{sparc.h}     \\
      x &=& 0x2      &in& \texttt{OP(x)  /* ((x) \& 0x3)  $<<$ 30 */} \\
      y &=& 0x01     &in& \texttt{OP3(y) /* ((y) \& 0x3f) $<<$ 19 */} \\
      z &=& 0x0      &in& \texttt{F3I(z) /* ((z) \& 0x1)  $<<$ 13 */} \\
      a &=& 0x1      &in& \texttt{OP\_AJIT\_BIT(a) /* ((a) \& 0x1)  $<<$ 5 */}
    \end{tabular}

    The AJIT bit  (insn[5]) is set internally by  \texttt{F4}, and hence
    there are only three arguments.

  \item \textbf{ANDDCC}:\\
    \begin{tabular}[h]{lclcl}
      Macro to set  &=& \texttt{F4(x, y, z)} &in& \texttt{sparc.h}     \\
      Macro to reset  &=& \texttt{INVF4(x, y, z)} &in& \texttt{sparc.h}     \\
      x &=& 0x2      &in& \texttt{OP(x)  /* ((x) \& 0x3)  $<<$ 30 */} \\
      y &=& 0x11     &in& \texttt{OP3(y) /* ((y) \& 0x3f) $<<$ 19 */} \\
      z &=& 0x0      &in& \texttt{F3I(z) /* ((z) \& 0x1)  $<<$ 13 */} \\
      a &=& 0x1      &in& \texttt{OP\_AJIT\_BIT(a) /* ((a) \& 0x1)  $<<$ 5 */}
    \end{tabular}

    The AJIT bit  (insn[5]) is set internally by  \texttt{F4}, and hence
    there are only three arguments.

  \item \textbf{ANDDN}:\\
    \begin{tabular}[h]{lclcl}
      Macro to set  &=& \texttt{F4(x, y, z)} &in& \texttt{sparc.h}     \\
      Macro to reset  &=& \texttt{INVF4(x, y, z)} &in& \texttt{sparc.h}     \\
      x &=& 0x2      &in& \texttt{OP(x)  /* ((x) \& 0x3)  $<<$ 30 */} \\
      y &=& 0x05     &in& \texttt{OP3(y) /* ((y) \& 0x3f) $<<$ 19 */} \\
      z &=& 0x0      &in& \texttt{F3I(z) /* ((z) \& 0x1)  $<<$ 13 */} \\
      a &=& 0x1      &in& \texttt{OP\_AJIT\_BIT(a) /* ((a) \& 0x1)  $<<$ 5 */}
    \end{tabular}

    The AJIT bit  (insn[5]) is set internally by  \texttt{F4}, and hence
    there are only three arguments.

  \item \textbf{ANDDNCC}:\\
    \begin{tabular}[h]{lclcl}
      Macro to set  &=& \texttt{F4(x, y, z)} &in& \texttt{sparc.h}     \\
      Macro to reset  &=& \texttt{INVF4(x, y, z)} &in& \texttt{sparc.h}     \\
      x &=& 0x2      &in& \texttt{OP(x)  /* ((x) \& 0x3)  $<<$ 30 */} \\
      y &=& 0x15     &in& \texttt{OP3(y) /* ((y) \& 0x3f) $<<$ 19 */} \\
      z &=& 0x0      &in& \texttt{F3I(z) /* ((z) \& 0x1)  $<<$ 13 */} \\
      a &=& 0x1      &in& \texttt{OP\_AJIT\_BIT(a) /* ((a) \& 0x1)  $<<$ 5 */}
    \end{tabular}

    The AJIT bit  (insn[5]) is set internally by  \texttt{F4}, and hence
    there are only three arguments.

  \end{enumerate}
% \subsubsection{Shift instructions:}
\label{sec:shift:insn:impl}
The shift  family of instructions  of AJIT  may each be  considered to
have  two versions:  a direct  count version  and a  register indirect
count version.  In the direct count  version the shift count is a part
of the  instruction bits.   In the indirect  count version,  the shift
count is  found on the  register specified by  the bit pattern  in the
instruction  bits.   The direct  count  version  is specified  by  the
14$^{th}$  bit, i.e.  insn[13]  (bit  number 13  in  the  0 based  bit
numbering scheme), being set to 1.  If insn[13] is 0 then the register
indirect version is specified.

Similar to the addition and subtraction instructions, the shift family
of instructions of  SPARC V8 also do  not use bits from 5  to 12 (both
inclusive).  The AJIT processor uses bits  5 and 6.  In particular bit
6 is always 1.   Bit 5 may be used in the direct  version giving a set
of 6 bits  available for specifying the shift count.   The shift count
can have  a maximum  value of  64.  Bit  5 is  unused in  the register
indirect version, and is always 0 in that case.

These instructions  are therefore  worked out  below in  two different
sets: the direct and the register indirect ones.
\begin{enumerate}
\item The direct versions  are given by insn[13] = 1.  The 6 bit shift
  count  is directly  specified  in the  instruction bits.   Therefore
  insn[5:0] specify the  shift count.  insn[6] =  1, distinguishes the
  AJIT version from the SPARC V8 version.
  \begin{enumerate}
  \item \textbf{SLLD}:\\
    \begin{center}
      \begin{tabular}[p]{|c|c|l|p{.25\textwidth}|p{.3\textwidth}|}
        \hline
        \textbf{Start} & \textbf{End} & \textbf{Range} & \textbf{Meaning} & \textbf{New Meaning}\\
        \hline
        0 & 4 & 32 & Source register 2, rs2 & Lowest 5 bits of shift count \\
        \hline
        5 & 12 & -- & \textbf{Unused. Set to 0 by software.} &
                                        \begin{minipage}[h]{1.0\linewidth}
                                          \begin{itemize}
                                          \item \textbf{Use bit 5
                                              to specify the msb of
                                              shift count.}
                                          \item \textbf{Use bit 6 to
                                              distinguish AJIT from
                                              SPARC V8.}
                                          \item \textbf{Set bits 7:12
                                              to 0.}
                                          \end{itemize}
                                        \end{minipage}
        \\
        \hline
        13 & 13 & 0,1 & The \textbf{i} bit & \textbf{Set i to ``1''} \\
        14 & 18 & 32 & Source register 1, rs1 & No change \\
        19 & 24 & 100101 & ``\textbf{op3}'' & No change \\
        25 & 29 & 32 & Destination register, rd & No change \\
        30 & 31 & 4 & Always ``10'' & No change \\
        \hline
      \end{tabular}
    \end{center}
    \begin{itemize}
    \item []\textbf{SLLD}: same as SLL, but with Instr[13]=0 (i=0),
      and Instr[5]=1.
    \item []\textbf{Syntax}: ``\texttt{slld SrcReg1, 6BitShiftCnt,
        DestReg}''. \\
      (\textbf{Note:} In an assembly language program, when the second
      argument is a number, we have direct mode.  A register number is
      prefixed with  ``r'', and hence the  syntax itself distinguished
      between   direct  and   register   indirect   version  of   this
      instruction.)
    \item []\textbf{Semantics}: rd(pair) $\leftarrow$ rs1(pair) $<<$
      shift count.
    \end{itemize}
    Bits layout:
\begin{verbatim}
    Offsets      : 31       24 23       16  15        8   7        0
    Bit layout   :  XXXX  XXXX  XXXX  XXXX   XXXX  XXXX   XXXX  XXXX
    Insn Bits    :  10       1  0010  1        1           1        
    Destination  :    DD  DDD                                       
    Source 1     :                     SSS   SS
    Source 2     :                                           S  SSSS
    Unused (0)   :                              U  UUUU   UU        
    Final layout :  10DD  DDD1  0010  1SSS   SS1U  UUUU   U1II  IIII
\end{verbatim}

    This will need another macro that sets bits 5 and 6. Let's call it
    \texttt{OP\_AJIT\_BITS\_5\_AND\_6}.   Hence the  SPARC bit  layout of  this
    instruction is:

    \begin{tabular}[h]{lclcl}
      Macro to set  &=& \texttt{F5(x, y, z)} &in& \texttt{sparc.h}     \\
      Macro to reset  &=& \texttt{INVF5(x, y, z)} &in& \texttt{sparc.h}     \\
      x &=& 0x2      &in& \texttt{OP(x)  /* ((x) \& 0x3)  $<<$ 30 */} \\
      y &=& 0x25     &in& \texttt{OP3(y) /* ((y) \& 0x3f) $<<$ 19 */} \\
      z &=& 0x1      &in& \texttt{F3I(z) /* ((z) \& 0x1)  $<<$ 13 */} \\
      a &=& 0x2      &in& \texttt{OP\_AJIT\_BITS\_5\_AND\_6(a) /* ((a) \& 0x3  $<<$ 6 */}
    \end{tabular}

    The AJIT bits (insn[6:5]) is  set or reset internally by \texttt{F5}
    (just  like  in  \texttt{F4}),  and   hence  there  are  only  three
    arguments.

  \item \textbf{SRLD}:\\
    \begin{center}
      \begin{tabular}[p]{|c|c|l|l|p{.35\textwidth}|}
        \hline
        \textbf{Start} & \textbf{End} & \textbf{Range} & \textbf{Meaning} & \textbf{New Meaning}\\
        \hline
        0 & 4 & 32 & Source register 2, rs2 & Lowest 5 bits of shift count \\
        \hline
        5 & 12 & -- & \textbf{unused} &
                                        \begin{minipage}[h]{1.0\linewidth}
                                          \begin{itemize}
                                          \item \textbf{Use bit 5
                                              to specify the msb of
                                              shift count.}
                                          \item \textbf{Use bit 6 to
                                              distinguish AJIT from
                                              SPARC V8.}
                                          \end{itemize}
                                        \end{minipage}
        \\
        \hline
        13 & 13 & 0,1 & The \textbf{i} bit & \textbf{Set i to ``1''} \\
        14 & 18 & 32 & Source register 1, rs1 & No change \\
        19 & 24 & 100110 & ``\textbf{op3}'' & No change \\
        25 & 29 & 32 & Destination register, rd & No change \\
        30 & 31 & 4 & Always ``10'' & No change \\
        \hline
      \end{tabular}
    \end{center}
    \begin{itemize}
    \item []\textbf{SRLD}: same as SRL, but with Instr[13]=0 (i=0),
      and Instr[5]=1.
    \item []\textbf{Syntax}: ``\texttt{sral SrcReg1, 6BitShiftCnt,
        DestReg}''. \\
      (\textbf{Note:} In an assembly language program, when the second
      argument is a number, we have direct mode.  A register number is
      prefixed with  ``r'', and hence the  syntax itself distinguished
      between   direct  and   register   indirect   version  of   this
      instruction.)
    \item []\textbf{Semantics}: rd(pair) $\leftarrow$ rs1(pair) $>>$
      shift count.
    \end{itemize}
    Bits layout:
\begin{verbatim}
    Offsets      : 31       24 23       16  15        8   7        0
    Bit layout   :  XXXX  XXXX  XXXX  XXXX   XXXX  XXXX   XXXX  XXXX
    Insn Bits    :  10       1  0011  0        1           1        
    Destination  :    DD  DDD                                       
    Source 1     :                     SSS   SS
    Source 2     :                                           S  SSSS
    Unused (0)   :                              U  UUUU   UU        
    Final layout :  10DD  DDD1  0011  0SSS   SS1U  UUUU   U1II  IIII
\end{verbatim}

    This will need another macro that sets bits 5 and 6. Let's call it
    \texttt{OP\_AJIT\_BITS\_5\_AND\_6}.   Hence the  SPARC bit  layout of  this
    instruction is:

    \begin{tabular}[h]{lclcl}
      Macro to set  &=& \texttt{F5(x, y, z)} &in& \texttt{sparc.h}     \\
      Macro to reset  &=& \texttt{INVF5(x, y, z)} &in& \texttt{sparc.h}     \\
      x &=& 0x2      &in& \texttt{OP(x)  /* ((x) \& 0x3)  $<<$ 30 */} \\
      y &=& 0x26     &in& \texttt{OP3(y) /* ((y) \& 0x3f) $<<$ 19 */} \\
      z &=& 0x1      &in& \texttt{F3I(z) /* ((z) \& 0x1)  $<<$ 13 */} \\
      a &=& 0x2      &in& \texttt{OP\_AJIT\_BITS\_5\_AND\_6(a) /* ((a) \& 0x3  $<<$ 6 */}
    \end{tabular}

    The AJIT bits (insn[6:5]) is  set or reset internally by \texttt{F5}
    (just  like  in  \texttt{F4}),  and   hence  there  are  only  three
    arguments.
    
  \item \textbf{SRAD}:\\
    \begin{center}
      \begin{tabular}[p]{|c|c|l|l|p{.35\textwidth}|}
        \hline
        \textbf{Start} & \textbf{End} & \textbf{Range} & \textbf{Meaning} & \textbf{New Meaning}\\
        \hline
        0 & 4 & 32 & Source register 2, rs2 & Lowest 5 bits of shift count \\
        \hline
        5 & 12 & -- & \textbf{unused} &
                                        \begin{minipage}[h]{1.0\linewidth}
                                          \begin{itemize}
                                          \item \textbf{Use bit 5
                                              to specify the msb of
                                              shift count.}
                                          \item \textbf{Use bit 6 to
                                              distinguish AJIT from
                                              SPARC V8.}
                                          \end{itemize}
                                        \end{minipage}
        \\
        \hline
        13 & 13 & 0,1 & The \textbf{i} bit & \textbf{Set i to ``1''} \\
        14 & 18 & 32 & Source register 1, rs1 & No change \\
        19 & 24 & 100111 & ``\textbf{op3}'' & No change \\
        25 & 29 & 32 & Destination register, rd & No change \\
        30 & 31 & 4 & Always ``10'' & No change \\
        \hline
      \end{tabular}
    \end{center}
    \begin{itemize}
    \item []\textbf{SRAD}: same as SRA, but with Instr[13]=0 (i=0),
      and Instr[5]=1.
    \item []\textbf{Syntax}: ``\texttt{srad SrcReg1, 6BitShiftCnt,
        DestReg}''. \\
      (\textbf{Note:} In an assembly language program, when the second
      argument is a number, we have direct mode.  A register number is
      prefixed with  ``r'', and hence the  syntax itself distinguished
      between   direct  and   register   indirect   version  of   this
      instruction.)
    \item []\textbf{Semantics}: rd(pair) $\leftarrow$ rs1(pair) $>>$
      shift count (with sign extension).
    \end{itemize}
    Bits layout:
\begin{verbatim}
    Offsets      : 31       24 23       16  15        8   7        0
    Bit layout   :  XXXX  XXXX  XXXX  XXXX   XXXX  XXXX   XXXX  XXXX
    Insn Bits    :  10       1  0011  1        1           1        
    Destination  :    DD  DDD                                       
    Source 1     :                     SSS   SS
    Source 2     :                                           S  SSSS
    Unused (0)   :                              U  UUUU   UU        
    Final layout :  10DD  DDD1  0011  1SSS   SS1U  UUUU   U1II  IIII
\end{verbatim}

    This will need another macro that sets bits 5 and 6. Let's call it
    \texttt{OP\_AJIT\_BITS\_5\_AND\_6}.   Hence the  SPARC bit  layout of  this
    instruction is:

    \begin{tabular}[h]{lclcl}
      Macro to set  &=& \texttt{F5(x, y, z)} &in& \texttt{sparc.h}     \\
      Macro to reset  &=& \texttt{INVF5(x, y, z)} &in& \texttt{sparc.h}     \\
      x &=& 0x2      &in& \texttt{OP(x)  /* ((x) \& 0x3)  $<<$ 30 */} \\
      y &=& 0x27     &in& \texttt{OP3(y) /* ((y) \& 0x3f) $<<$ 19 */} \\
      z &=& 0x1      &in& \texttt{F3I(z) /* ((z) \& 0x1)  $<<$ 13 */} \\
      a &=& 0x2      &in& \texttt{OP\_AJIT\_BITS\_5\_AND\_6(a) /* ((a) \& 0x3  $<<$ 6 */}
    \end{tabular}

    The AJIT bits (insn[6:5]) is  set or reset internally by \texttt{F5}
    (just  like  in  \texttt{F4}),  and   hence  there  are  only  three
    arguments.

  \end{enumerate}
\item The register  indirect versions are given by insn[13]  = 0.  The
  shift count is indirectly specified in the 32 bit register specified
  in instruction bits.  Therefore  insn[4:0] specify the register that
  has the  shift count.  insn[6]  = 1, distinguishes the  AJIT version
  from the SPARC V8 version.  In this case, insn[5] = 0, necessarily.
  \begin{enumerate}
  \item \textbf{SLLD}:\\
    \begin{center}
      \begin{tabular}[p]{|c|c|l|l|p{.35\textwidth}|}
        \hline
        \textbf{Start} & \textbf{End} & \textbf{Range} & \textbf{Meaning} &
                                                                            \textbf{New Meaning}\\
        \hline
        0 & 4 & 32 & Source register 2, rs2 & Register number \\
        \hline
        5 & 12 & -- & \textbf{unused} &
                                        \begin{minipage}[h]{1.0\linewidth}
                                          \begin{itemize}
                                          \item \textbf{Set bit 5 to 0.}
                                          \item \textbf{Use bit 6 to
                                              distinguish AJIT from
                                              SPARC V8.}
                                          \end{itemize}
                                        \end{minipage}
        \\
        \hline
        13 & 13 & 0,1 & The \textbf{i} bit & \textbf{Set i to ``0''} \\
        14 & 18 & 32 & Source register 1, rs1 & No change \\
        19 & 24 & 100101 & ``\textbf{op3}'' & No change \\
        25 & 29 & 32 & Destination register, rd & No change \\
        30 & 31 & 4 & Always ``10'' & No change \\
        \hline
      \end{tabular}
    \end{center}
    \begin{itemize}
    \item []\textbf{SLLD}: same as SLL, but with Instr[13]=0 (i=0),
      and Instr[5]=1.
    \item []\textbf{Syntax}: ``\texttt{slld SrcReg1, SrcReg2,
        DestReg}''.
    \item []\textbf{Semantics}: rd(pair) $\leftarrow$ rs1(pair) $<<$
      shift count register rs2.
    \end{itemize}
    Bits layout:
\begin{verbatim}
    Offsets      : 31       24 23       16  15        8   7        0
    Bit layout   :  XXXX  XXXX  XXXX  XXXX   XXXX  XXXX   XXXX  XXXX
    Insn Bits    :  10       1  0010  1        0           10        
    Destination  :    DD  DDD                                       
    Source 1     :                     SSS   SS
    Source 2     :                                           S  SSSS
    Unused (0)   :                              U  UUUU   UU        
    Final layout :  10DD  DDD1  0010  1SSS   SS0U  UUUU   U10I  IIII
\end{verbatim}

    This will need another macro that sets bits 5 and 6. Let's call it
    \texttt{OP\_AJIT\_BITS\_5\_AND\_6}.   Hence the  SPARC bit  layout of  this
    instruction is:

    \begin{tabular}[h]{lclcl}
      Macro to set  &=& \texttt{F5(x, y, z)} &in& \texttt{sparc.h}     \\
      Macro to reset  &=& \texttt{INVF5(x, y, z)} &in& \texttt{sparc.h}     \\
      x &=& 0x2      &in& \texttt{OP(x)  /* ((x) \& 0x3)  $<<$ 30 */} \\
      y &=& 0x25     &in& \texttt{OP3(y) /* ((y) \& 0x3f) $<<$ 19 */} \\
      z &=& 0x0      &in& \texttt{F3I(z) /* ((z) \& 0x1)  $<<$ 13 */} \\
      a &=& 0x2      &in& \texttt{OP\_AJIT\_BITS\_5\_AND\_6(a) /* ((a) \& 0x3  $<<$ 6 */}
    \end{tabular}

    The AJIT bits (insn[6:5]) is  set or reset internally by \texttt{F5}
    (just  like  in  \texttt{F4}),  and   hence  there  are  only  three
    arguments.

  \item \textbf{SRLD}:\\
    \begin{center}
      \begin{tabular}[p]{|c|c|l|l|p{.35\textwidth}|}
        \hline
        \textbf{Start} & \textbf{End} & \textbf{Range} & \textbf{Meaning} &
                                                                            \textbf{New Meaning}\\
        \hline
        0 & 4 & 32 & Source register 2, rs2 & Register number \\
        \hline
        5 & 12 & -- & \textbf{unused} &
                                        \begin{minipage}[h]{1.0\linewidth}
                                          \begin{itemize}
                                          \item \textbf{Set bit 5 to 0.}
                                          \item \textbf{Use bit 6 to
                                              distinguish AJIT from
                                              SPARC V8.}
                                          \end{itemize}
                                        \end{minipage}
        \\
        \hline
        13 & 13 & 0,1 & The \textbf{i} bit & \textbf{Set i to ``0''} \\
        14 & 18 & 32 & Source register 1, rs1 & No change \\
        19 & 24 & 100110 & ``\textbf{op3}'' & No change \\
        25 & 29 & 32 & Destination register, rd & No change \\
        30 & 31 & 4 & Always ``10'' & No change \\
        \hline
      \end{tabular}
    \end{center}
    \begin{itemize}
    \item []\textbf{SRLD}: same as SRL, but with Instr[13]=0 (i=0),
      and Instr[5]=1.
    \item []\textbf{Syntax}: ``\texttt{slld SrcReg1, SrcReg2,
        DestReg}''.
    \item []\textbf{Semantics}: rd(pair) $\leftarrow$ rs1(pair) $>>$
      shift count register rs2.
    \end{itemize}
    Bits layout:
\begin{verbatim}
    Offsets      : 31       24 23       16  15        8   7        0
    Bit layout   :  XXXX  XXXX  XXXX  XXXX   XXXX  XXXX   XXXX  XXXX
    Insn Bits    :  10       1  0011  0        0           10        
    Destination  :    DD  DDD                                       
    Source 1     :                     SSS   SS
    Source 2     :                                           S  SSSS
    Unused (0)   :                              U  UUUU   UU        
    Final layout :  10DD  DDD1  0011  0SSS   SS0U  UUUU   U10I  IIII
\end{verbatim}

    This will need another macro that sets bits 5 and 6. Let's call it
    \texttt{OP\_AJIT\_BITS\_5\_AND\_6}.   Hence the  SPARC bit  layout of  this
    instruction is:

    \begin{tabular}[h]{lclcl}
      Macro to set  &=& \texttt{F5(x, y, z)} &in& \texttt{sparc.h}     \\
      Macro to reset  &=& \texttt{INVF5(x, y, z)} &in& \texttt{sparc.h}     \\
      x &=& 0x2      &in& \texttt{OP(x)  /* ((x) \& 0x3)  $<<$ 30 */} \\
      y &=& 0x26     &in& \texttt{OP3(y) /* ((y) \& 0x3f) $<<$ 19 */} \\
      z &=& 0x0      &in& \texttt{F3I(z) /* ((z) \& 0x1)  $<<$ 13 */} \\
      a &=& 0x2      &in& \texttt{OP\_AJIT\_BITS\_5\_AND\_6(a) /* ((a) \& 0x3  $<<$ 6 */}
    \end{tabular}

    The AJIT bits (insn[6:5]) is  set or reset internally by \texttt{F5}
    (just  like  in  \texttt{F4}),  and   hence  there  are  only  three
    arguments.

  \item \textbf{SRAD}:\\
    \begin{center}
      \begin{tabular}[p]{|c|c|l|l|p{.35\textwidth}|}
        \hline
        \textbf{Start} & \textbf{End} & \textbf{Range} & \textbf{Meaning} &
                                                                            \textbf{New Meaning}\\
        \hline
        0 & 4 & 32 & Source register 2, rs2 & Register number \\
        \hline
        5 & 12 & -- & \textbf{unused} &
                                        \begin{minipage}[h]{1.0\linewidth}
                                          \begin{itemize}
                                          \item \textbf{Set bit 5 to 0.}
                                          \item \textbf{Use bit 6 to
                                              distinguish AJIT from
                                              SPARC V8.}
                                          \end{itemize}
                                        \end{minipage}
        \\
        \hline
        13 & 13 & 0,1 & The \textbf{i} bit & \textbf{Set i to ``0''} \\
        14 & 18 & 32 & Source register 1, rs1 & No change \\
        19 & 24 & 100101 & ``\textbf{op3}'' & No change \\
        25 & 29 & 32 & Destination register, rd & No change \\
        30 & 31 & 4 & Always ``10'' & No change \\
        \hline
      \end{tabular}
    \end{center}
    \begin{itemize}
    \item []\textbf{SRAD}: same as SRA, but with Instr[13]=0 (i=0),
      and Instr[5]=1.
    \item []\textbf{Syntax}: ``\texttt{slld SrcReg1, SrcReg2,
        DestReg}''.
    \item []\textbf{Semantics}: rd(pair) $\leftarrow$ rs1(pair) $>>$
      shift count register rs2 (with sign extension).
    \end{itemize}
    Bits layout:
\begin{verbatim}
    Offsets      : 31       24 23       16  15        8   7        0
    Bit layout   :  XXXX  XXXX  XXXX  XXXX   XXXX  XXXX   XXXX  XXXX
    Insn Bits    :  10       1  0011  1        0           10        
    Destination  :    DD  DDD                                       
    Source 1     :                     SSS   SS
    Source 2     :                                           S  SSSS
    Unused (0)   :                              U  UUUU   UU        
    Final layout :  10DD  DDD1  0011  1SSS   SS0U  UUUU   U10I  IIII
\end{verbatim}

    This will need another macro that sets bits 5 and 6. Let's call it
    \texttt{OP\_AJIT\_BITS\_5\_AND\_6}.   Hence the  SPARC bit  layout of  this
    instruction is:

    \begin{tabular}[h]{lclcl}
      Macro to set  &=& \texttt{F5(x, y, z)} &in& \texttt{sparc.h}     \\
      Macro to reset  &=& \texttt{INVF5(x, y, z)} &in& \texttt{sparc.h}     \\
      x &=& 0x2      &in& \texttt{OP(x)  /* ((x) \& 0x3)  $<<$ 30 */} \\
      y &=& 0x27     &in& \texttt{OP3(y) /* ((y) \& 0x3f) $<<$ 19 */} \\
      z &=& 0x0      &in& \texttt{F3I(z) /* ((z) \& 0x1)  $<<$ 13 */} \\
      a &=& 0x2      &in& \texttt{OP\_AJIT\_BITS\_5\_AND\_6(a) /* ((a) \& 0x3  $<<$ 6 */}
    \end{tabular}

    The AJIT bits (insn[6:5]) is  set or reset internally by \texttt{F5}
    (just  like  in  \texttt{F4}),  and   hence  there  are  only  three
    arguments.
  \end{enumerate}
\end{enumerate}

\item {Shift instructions:} \\

  The shift  family of instructions  of AJIT  may each be  considered to
  have  two versions:  a direct  count version  and a  register indirect
  count version.  In the direct count  version the shift count is a part
  of the  instruction bits.   In the indirect  count version,  the shift
  count is  found on the  register specified by  the bit pattern  in the
  instruction  bits.   The direct  count  version  is specified  by  the
  14$^{th}$  bit, i.e.  insn[13]  (bit  number 13  in  the  0 based  bit
  numbering scheme), being set to 1.  If insn[13] is 0 then the register
  indirect version is specified.

  Similar to the addition and subtraction instructions, the shift family
  of instructions of  SPARC V8 also do  not use bits from 5  to 12 (both
  inclusive).  The AJIT processor uses bits  5 and 6.  In particular bit
  6 is always 1.   Bit 5 may be used in the direct  version giving a set
  of 6 bits  available for specifying the shift count.   The shift count
  can have  a maximum  value of  64.  Bit  5 is  unused in  the register
  indirect version, and is always 0 in that case.

  These instructions  are therefore  worked out  below in  two different
  sets: the direct and the register indirect ones.
  \begin{enumerate}
  \item The direct versions  are given by insn[13] = 1.  The 6 bit shift
    count  is directly  specified  in the  instruction bits.   Therefore
    insn[5:0] specify the  shift count.  insn[6] =  1, distinguishes the
    AJIT version from the SPARC V8 version.
    \begin{enumerate}
    \item \textbf{SLLD}:\\
      This will need another macro that sets bits 5 and 6. Let's call it
      \texttt{OP\_AJIT\_BIT\_2}.   Hence the  SPARC bit  layout of  this
      instruction is:

      \begin{tabular}[h]{lclcl}
        Macro to set  &=& \texttt{F5(x, y, z)} &in& \texttt{sparc.h}     \\
        Macro to reset  &=& \texttt{INVF5(x, y, z)} &in& \texttt{sparc.h}     \\
        x &=& 0x2      &in& \texttt{OP(x)  /* ((x) \& 0x3)  $<<$ 30 */} \\
        y &=& 0x25     &in& \texttt{OP3(y) /* ((y) \& 0x3f) $<<$ 19 */} \\
        z &=& 0x1      &in& \texttt{F3I(z) /* ((z) \& 0x1)  $<<$ 13 */} \\
        a &=& 0x2      &in& \texttt{OP\_AJIT\_BIT\_2(a) /* ((a) \& 0x3  $<<$ 6 */}
      \end{tabular}

      The AJIT bits (insn[6:5]) is  set or reset internally by \texttt{F5}
      (just  like  in  \texttt{F4}),  and   hence  there  are  only  three
      arguments.

    \item \textbf{SRLD}:\\
      This will need another macro that sets bits 5 and 6. Let's call it
      \texttt{OP\_AJIT\_BIT\_2}.   Hence the  SPARC bit  layout of  this
      instruction is:

      \begin{tabular}[h]{lclcl}
        Macro to set  &=& \texttt{F5(x, y, z)} &in& \texttt{sparc.h}     \\
        Macro to reset  &=& \texttt{INVF5(x, y, z)} &in& \texttt{sparc.h}     \\
        x &=& 0x2      &in& \texttt{OP(x)  /* ((x) \& 0x3)  $<<$ 30 */} \\
        y &=& 0x26     &in& \texttt{OP3(y) /* ((y) \& 0x3f) $<<$ 19 */} \\
        z &=& 0x1      &in& \texttt{F3I(z) /* ((z) \& 0x1)  $<<$ 13 */} \\
        a &=& 0x2      &in& \texttt{OP\_AJIT\_BIT\_2(a) /* ((a) \& 0x3  $<<$ 6 */}
      \end{tabular}

      The AJIT bits (insn[6:5]) is  set or reset internally by \texttt{F5}
      (just  like  in  \texttt{F4}),  and   hence  there  are  only  three
      arguments.
      
    \item \textbf{SRAD}:\\
      This will need another macro that sets bits 5 and 6. Let's call it
      \texttt{OP\_AJIT\_BIT\_2}.   Hence the  SPARC bit  layout of  this
      instruction is:

      \begin{tabular}[h]{lclcl}
        Macro to set  &=& \texttt{F5(x, y, z)} &in& \texttt{sparc.h}     \\
        Macro to reset  &=& \texttt{INVF5(x, y, z)} &in& \texttt{sparc.h}     \\
        x &=& 0x2      &in& \texttt{OP(x)  /* ((x) \& 0x3)  $<<$ 30 */} \\
        y &=& 0x27     &in& \texttt{OP3(y) /* ((y) \& 0x3f) $<<$ 19 */} \\
        z &=& 0x1      &in& \texttt{F3I(z) /* ((z) \& 0x1)  $<<$ 13 */} \\
        a &=& 0x2      &in& \texttt{OP\_AJIT\_BIT\_2(a) /* ((a) \& 0x3  $<<$ 6 */}
      \end{tabular}

      The AJIT bits (insn[6:5]) is  set or reset internally by \texttt{F5}
      (just  like  in  \texttt{F4}),  and   hence  there  are  only  three
      arguments.

    \end{enumerate}
  \item The register  indirect versions are given by insn[13]  = 0.  The
    shift count is indirectly specified in the 32 bit register specified
    in instruction bits.  Therefore  insn[4:0] specify the register that
    has the  shift count.  insn[6]  = 1, distinguishes the  AJIT version
    from the SPARC V8 version.  In this case, insn[5] = 0, necessarily.
    \begin{enumerate}
    \item \textbf{SLLD}:\\
      This will need another macro that sets bits 5 and 6. Let's call it
      \texttt{OP\_AJIT\_BIT\_2}.   Hence the  SPARC bit  layout of  this
      instruction is:

      \begin{tabular}[h]{lclcl}
        Macro to set  &=& \texttt{F5(x, y, z)} &in& \texttt{sparc.h}     \\
        Macro to reset  &=& \texttt{INVF5(x, y, z)} &in& \texttt{sparc.h}     \\
        x &=& 0x2      &in& \texttt{OP(x)  /* ((x) \& 0x3)  $<<$ 30 */} \\
        y &=& 0x25     &in& \texttt{OP3(y) /* ((y) \& 0x3f) $<<$ 19 */} \\
        z &=& 0x0      &in& \texttt{F3I(z) /* ((z) \& 0x1)  $<<$ 13 */} \\
        a &=& 0x2      &in& \texttt{OP\_AJIT\_BIT\_2(a) /* ((a) \& 0x3  $<<$ 6 */}
      \end{tabular}

      The AJIT bits (insn[6:5]) is  set or reset internally by \texttt{F5}
      (just  like  in  \texttt{F4}),  and   hence  there  are  only  three
      arguments.

    \item \textbf{SRLD}:\\
      This will need another macro that sets bits 5 and 6. Let's call it
      \texttt{OP\_AJIT\_BIT\_2}.   Hence the  SPARC bit  layout of  this
      instruction is:

      \begin{tabular}[h]{lclcl}
        Macro to set  &=& \texttt{F5(x, y, z)} &in& \texttt{sparc.h}     \\
        Macro to reset  &=& \texttt{INVF5(x, y, z)} &in& \texttt{sparc.h}     \\
        x &=& 0x2      &in& \texttt{OP(x)  /* ((x) \& 0x3)  $<<$ 30 */} \\
        y &=& 0x26     &in& \texttt{OP3(y) /* ((y) \& 0x3f) $<<$ 19 */} \\
        z &=& 0x0      &in& \texttt{F3I(z) /* ((z) \& 0x1)  $<<$ 13 */} \\
        a &=& 0x2      &in& \texttt{OP\_AJIT\_BIT\_2(a) /* ((a) \& 0x3  $<<$ 6 */}
      \end{tabular}

      The AJIT bits (insn[6:5]) is  set or reset internally by \texttt{F5}
      (just  like  in  \texttt{F4}),  and   hence  there  are  only  three
      arguments.

    \item \textbf{SRAD}:\\
      This will need another macro that sets bits 5 and 6. Let's call it
      \texttt{OP\_AJIT\_BIT\_2}.   Hence the  SPARC bit  layout of  this
      instruction is:

      \begin{tabular}[h]{lclcl}
        Macro to set  &=& \texttt{F5(x, y, z)} &in& \texttt{sparc.h}     \\
        Macro to reset  &=& \texttt{INVF5(x, y, z)} &in& \texttt{sparc.h}     \\
        x &=& 0x2      &in& \texttt{OP(x)  /* ((x) \& 0x3)  $<<$ 30 */} \\
        y &=& 0x27     &in& \texttt{OP3(y) /* ((y) \& 0x3f) $<<$ 19 */} \\
        z &=& 0x0      &in& \texttt{F3I(z) /* ((z) \& 0x1)  $<<$ 13 */} \\
        a &=& 0x2      &in& \texttt{OP\_AJIT\_BIT\_2(a) /* ((a) \& 0x3  $<<$ 6 */}
      \end{tabular}

      The AJIT bits (insn[6:5]) is  set or reset internally by \texttt{F5}
      (just  like  in  \texttt{F4}),  and   hence  there  are  only  three
      arguments.
    \end{enumerate}
  \end{enumerate}
\end{itemize}

%%% Local Variables:
%%% mode: latex
%%% TeX-master: t
%%% End:


\subsection{Integer-Unit Extensions: SIMD Instructions}
\label{sec:integer-unit-extns:simd-instructions:impl}

\subsubsection{SIMD I instructions:}
\label{sec:simd:1:insn:impl}

The  first   set  of  SIMD  instructions  are   the  three  arithmetic
instructions: add,  sub, and mul.  The ``mul''  instruction has signed
and unsigned variations.  Each of the three instructions have 8 bit (1
byte),  16 bit  (1 half  word) and  32 bit  (1 word)  versions.  These
versions  are encoded  as  shown in  table~\ref{tab:types:for:simd:1},
where the first column denotes the  bit numbers.  We list all the SIMD
I instructions version wise below.
\begin{table}[h]
  \centering
  \begin{tabular}[p]{|l|l|l|}
  \hline
  \textbf{987} & \textbf{Type} & \textbf{Example}\\
  \hline
  001 & Byte & e.g. VADDD8\\
  010 & Half-word (16-bits) & e.g. VADDD16\\
  100 & Word (32-bits) & e.g. VADDD32\\
  \hline
\end{tabular}
\caption{Data type encoding for SIMD I instructions.}
\label{tab:types:for:simd:1}
\end{table}
\begin{enumerate}
\item \textbf{8 bit} (\textbf{1 Byte})
  \begin{enumerate}
  \item \textbf{VADDD8}:\\
    \begin{center}
      \begin{tabular}[p]{|c|c|l|l|}
        \hline
        \textbf{Start} & \textbf{End} & \textbf{Range} & \textbf{Meaning} \\
        \hline
        0 & 4 & 32 & Source register 2, rs2 \\
        5 & 6 & 4 & \emph{Always} 2, i.e. insn[6:5] = 10$_b$ \\
        7 & 9 & 8 & \textbf{Data type} specifier:  \emph{Always} 0x1\\
        10 & 12 & -- & \textbf{unused} \\
        13 & 13 & 0,1 & The \textbf{i} bit. \emph{Always} 0. \\
        14 & 18 & 32 & Source register 1, rs1 \\
        19 & 24 & 000000 & ``\textbf{op3}'' \\
        25 & 29 & 32 & Destination register, rd \\
        30 & 31 & 4 & Always ``10'' \\
        \hline
      \end{tabular}
    \end{center}
    \begin{itemize}
    \item []\textbf{VADDD8}: same as  ADD, but with Instr[13]=0 (i=0),
      and  Instr[6:5]=2.  Bits Instr[9:7]  are  a  3-bit field,  which
      specify the data type
    \item []\textbf{Syntax}: ``\texttt{vaddd8 SrcReg1, SrcReg2,
        DestReg}''.
    \item []\textbf{Semantics}: \emph{not given}
    \end{itemize}
    Bits layout:
\begin{verbatim}
    Offsets      : 31       24 23       16  15        8   7        0
    Bit layout   :  XXXX  XXXX  XXXX  XXXX   XXXX  XXXX   XXXX  XXXX
    Insn Bits    :  10       0  0000  0        0     00   110       
    Destination  :    DD  DDD                                       
    Source 1     :                     SSS   SS
    Source 2     :                                           S  SSSS
    Unused (0)   :                              U  UU               
    Final layout :  10DD  DDD0  0000  0SSS   SS0U  UU00   110S  SSSS
    To match     :  ^^       ^  ^^^^  ^        ^     ^^   ^^^
    Bitfield name:  OP          OP3            i     9-   765
\end{verbatim}

    To  set  up  bits  5  and  6, we  use  an  already  defined  macro
    \texttt{OP\_AJIT\_BIT\_5\_AND\_6}.  The  value to be set  in these
    two bits is 0x2.   To set bits 7 through 9, we  define a new macro
    \texttt{OP\_AJIT\_BIT\_7\_THRU\_9}.  The value  set in these three
    bits  decides the  \emph{type}, byte,  half word  or word,  of the
    instruction.  For  \textbf{vaddd8} instruction,  bits 7  through 9
    are  set  to the  value  0x1.   Both  these macros  influence  the
    \emph{unused} bits of  the SPARC V8 architecture.  So  we define a
    macro \texttt{OP\_AJIT\_SET\_UNUSED} that uses the previous two to
    set these bits unused by the SPARC V8, but used by AJIT.

    \verb|#define OP_AJIT_BIT_7_THRU_9(x)   ((x) << 0x7)|

    \verb+#define OP_AJIT_SET_UNUSED        (OP_AJIT_BIT_5_AND_6(0x2) | \\+

    \verb+                                   OP_AJIT_BIT_7_THRU_9(0x1))+

    We can  now define the final  macro \texttt{F6(x, y, z,  b, a)} to
    set the match bits for this instruction.
\begin{verbatim}
#define OP_AJIT_BIT_5(x)          (((x) & 0x1) << 5)
#define F4(x, y, z, b)            (F3(x, y, z) | OP_AJIT_BIT_5(b))
#define OP_AJIT_BIT_5_AND_6(x)    (((x) & 0x3) << 6)
#define F5(x, y, z, b)            (F3(x, y, z) | OP_AJIT_BIT_5_AND_6 (b))
#define OP_AJIT_BIT_7_THRU_9(x)   (((x) & 0x3) << 9)
#define F6(x, y, z, b, a)         (F5 (x, y, z, b) | OP_AJIT_BIT_7_THRU_9(a))
\end{verbatim}
    Hence the SPARC bit layout of this instruction is:

  \begin{tabular}[h]{lclcl}
    Macro to set  &=& \texttt{F4(x, y, z)} &in& \texttt{sparc.h}     \\
    Macro to reset  &=& \texttt{INVF4(x, y, z)} &in& \texttt{sparc.h}     \\
    x &=& 0x2      &in& \texttt{OP(x)  /* ((x) \& 0x3)  $<<$ 30 */} \\
    y &=& 0x00     &in& \texttt{OP3(y) /* ((y) \& 0x3f) $<<$ 19 */} \\
    z &=& 0x0      &in& \texttt{F3I(z) /* ((z) \& 0x1)  $<<$ 13 */} \\
    a &=& 0x1      &in& \texttt{OP\_AJIT\_BIT(a) /* ((a) \& 0x1)  $<<$ 5 */}
  \end{tabular}

  The AJIT bit (insn[5]) is set internally by \texttt{F4}, and hence
  there are only three arguments.
\end{enumerate}
\item \textbf{1 Half word} (\textbf{16 bit})
\item \textbf{1 Word} (\textbf{32 bit})
\end{enumerate}



\subsection{Integer-Unit Extensions: SIMD Instructions II}
\label{sec:integer-unit-extns:simd-instructions:2:impl}

\subsection{Vector Floating Point Instructions}
\label{sec:vector-floating-point-instructions:impl}

\subsection{CSWAP instructions}
\label{sec:cswap-instructions:impl}


\chapter{Towards Assembler Extraction}
\label{chap:asm:extraction}

\section{Succinct ISA Descriptions}
\label{sec:sisad}

\textbf{A. M. Vichare}

ISA description languages seem to be  at least 20 years old problem as
of 2018.   Attempts like  MIMOLA or  LISA have  been made  to describe
processors and  generate system software through  them.  This document
records my  attempts to develop  such a  language afresh, but  for the
AJIT processor of IIT Bombay.  The benefit of hindsight should ideally
be employed in this design process.  I shall try to bring that in as a
parallel activity along side the attempts to a practical design.

\subsection{Instruction Set Design Study}
\label{sec:isa:design:study}

This is the  background work mainly of conceptual ideas,  and study of
some known examples.

\subsubsection{Basic Concepts of Instruction Set Design}
\label{sec:basic:isa:design:info}

From: Henn-Patt, CA-Quant.Approach. Ed.5, App.A:

\begin{itemize}
\item \textbf{Type of internal storage}:
  \begin{itemize}
  \item Stack: Operands are on the stack, and hence \emph{implicit} in
    the instruction.
  \item Accumulator: One of the  operands is in the \emph{accumulator}
    register, and hence implicit in the instruction.
  \item  Register-Memory:   Memory  \emph{can   be}  a  part   of  the
    instruction.
  \item  Register-Register: Memory  is  \textbf{never} a  part of  the
    instruction,   except   for    the   \emph{load-store}   pair   of
    instructions.
  \item Memory-Memory:  All operands  are in  the memory  and directly
    addressed as a  part of the instruction.  This is  an old style is
    not often found today ($\sim$ 2018).
  \item  Variations:  Dedicating  some   registers  for  some  special
    purposes  --  \textbf{extended   accumulator}  or  \textbf{special
      purpose registers}.
  \item  Number of  operands: This  depends  on the  type of  internal
    storage,  and  a  design   choice.   An  binary  instruction  (aka
    \emph{operation}) may explicitly take two data source operands and
    one result destination operand.  Or it may take only two operands,
    with one  of them being \textbf{both}  a data source and  a result
    destination operand.
  \end{itemize}
\item \textbf{Memory layout addressing}:
  \begin{itemize}
  \item Byte ordering: There are two ways to order a set of bytes of a
    multi-byte object  (e.g. 32 bit,  i.e. 4 byte integer).
    \begin{itemize}
    \item \textbf{Little Endian}: The  byte with the least significant
      bit can  be stored at  the smallest byte  address, or 
    \item \textbf{Big Endian}: The byte with the least significant bit
      can be stored at the largest byte address.
    \end{itemize}
  \item Alignment  needs: For  multibyte objects, an  architecture may
    need the components to be  aligned on suitable address boundaries.
    Or it may not need them to be so aligned!  If $k$ is the number of
    bytes of a  multibyte object, $a$ is the address  of the byte with
    the  least  significant  bit,  then  the  object  is  aligned  if:
    $a = n \times k$, where $n$  is a natural number.  The address $a$
    is an integral multiple of the object size $k$.
  \item Shifting needs: Consider  reading a \emph{single} byte aligned
    at a word address into a  \emph{64 bit} register.  A single 64 bit
    read, i.e. a  double word read, would be performed  on double word
    aligned address.  If  the word aligned byte  would \textbf{not} be
    double word aligned,  then the byte that is read  would not occupy
    the least  significant position in  the 64 bit register.   In such
    cases for correct  alignment, we will need to shift  the byte read
    in by 3  positions (calculate this ``3'') so that  it occupies the
    correct position in a 64 bit register.
  \end{itemize}
\item \textbf{Addressing Modes}: \\
  How do we address the primary memory?
  \begin{itemize}
  \item  \emph{Immediate}:  No  addressing  at  all.   The  argument/s
    (i.e. operand/s) is/are given as a part of the instruction.  There
    is a finite size, finite number of bits, and layout norms.
  \item \emph{Register Direct}: The  operand/s is/are available in one
    or more  registers.  Instead of  being placed in  the instruction,
    the operands are available in the register.
    \begin{itemize}
    \item \emph{PC Relative}: A  variant of register direct addressing
      where the  register to be used  is fixed as the  program counter
      (i.e.  the instruction pointer).
    \end{itemize}
  \item  \emph{Direct} or  \emph{Absolute}:  The  address is  provided
    directly as an  argument.  There could be  finite size definitions
    that could  be same as or  different from the size  of the address
    bus.
  \item \emph{Register Indirect}: The operand location is given in one
    or more registers.   The register size is expected to  be the same
    as the size of the address bus.
    \begin{itemize}
    \item \emph{Auto  Increment or  Decrement}: A variant  of register
      indirect where the  indirection value in the  register is either
      automatically incremented  or decremented.   Autoincrementing is
      useful for array  traversals with the base address  of the array
      in the  register, and  the array element  size as  the increment
      value.    Autodecrementing  is   similarly   useful  for   stack
      operations.
    \item   \emph{Displacement}:  A   variant  of   register  indirect
      addressing   mode,  the   operand  location   is  given   as  an
      \emph{offset}   (i.e.   \emph{displacement}),   relative  to   a
      register  indirect address.   The  memory location  is thus  the
      offset relative to the location given in a register.
    \item  \emph{Indexed}:   Another  variant  of   register  indirect
      addressing  where  the  operand   location  is  a  well  defined
      algebraic  relation of  values in  a few  registers.  Thus,  for
      example, the location  might be given as a  \emph{sum} of values
      in two registers where one  register has the ``base'' value, and
      the other has an ``index'' (i.e. an offset) relative to the base
      value.
    \item  \emph{Scaled}: Yet  another  variant  of register  indirect
      addressing where  the operand location  is again a  well defined
      algebraic relation of values in  a few registers.  The algebraic
      relation  is  a  displacement  relative to  a  ``base''  in  one
      register  and  an integral  scale  up  of ``index''  in  another
      register.
    \end{itemize}
  \item \emph{Memory  Indirect}: Adding one more  level of indirection
    to the register  indirect mode yields this mode.   The location of
    the operand is  now available at the memory location  given by the
    register indirect mode.
  \end{itemize}
  The immediate, displacement, and  register indirect addressing modes
  are predominantly used (about 75\% to 99\% of modes used).
\item \textbf{Types and Size of Operands}:
  \begin{itemize}
  \item  Some specifications  of  size have  standardized (e.g.   IEEE
    floating point), some have become conventional (e.g. 8 bit byte, 2
    byte half words, 4 byte words etc.), some are optionally supported
    by  the  processor  architecture   (e.g.   strings,  binary  coded
    decimal, packed  decimal).  Representation  is either  tagged (not
    used  much  today  $\sim$  2018), or  encoded  within  the  opcode
    (preferred method today).
  \item \emph{Standardised}: IEEE Floating  point -- single and double
    precision.  Single precision is 4 bytes, and double precision is 8
    bytes.
  \item \emph{Conventional}:
    \begin{center}
      \begin{tabular}[h]{|c|c|c|c|c|c|}
        \hline
        Quad Word   & Double word & Word & Half word & Byte & Bits \\
        \hline
        -- & -- & -- & -- & 1 & 8 \\
        -- & -- & -- & 1 & 2 & 16 \\
        -- & -- & 1 & 2 & 4 & 32 \\
        -- & 1 & 2 & 4 & 8 & 64 \\
        1 & 2 & 4 & 8 & 16 & 128 \\
        \hline
      \end{tabular}
    \end{center}
  \end{itemize}
\item \textbf{Operations in the Instruction Set}:\\
  Thumb rule: Simplest instructions are the most widely executed ones.
  \begin{center}
    \begin{tabular}[h]{|l|l|}
      \hline
      \textbf{Type} & \textbf{Description or Examples} \\
      \hline
      Arithmetic & Arithmetic operations on numbers: +, -, *, / etc. \\
      Logical & Logical: AND, OR, NOT \\
      Data Transfer & Load, Store, Move \\
      Control Flow & Branch, Loop, Jump, Procedure call and return,
                     Trap \\
      System & OS System call, Virtual memory management \\
      Floating point & Floating point +, -, *, / etc. \\
      Decimal & Decimal  +, -, *, / etc. \\
      String & String move, compare, search \\
      Graphics & Pixel and vertex operation, compression \&
                 decompression \\
      Signal Processing & FFT, MAC \\
      \hline
    \end{tabular}
  \end{center}

  It might be useful to classify at a little more higher level: 
  \begin{center}
    \begin{tabular}[h]{|l|l|}
      \hline
      \textbf{Class} & \textbf{Description or Examples} \\
      \hline
      Data Type based & Arithmetic, Logical, Floating point, Signal
                        Processing, Graphics, Decimal, String \\
      Data Transfer & All I/O \\
      System Control & Control flow, System management \\
      \hline
    \end{tabular}
  \end{center}
\item \textbf{Instructions for Control Flow}:\\
  \begin{itemize}
  \item No well defined convention for  naming, but we will follow the
    text referred at the beginning  of this section. Four main control
    flow instructions are usually offered.
    \begin{itemize}
    \item Jump: These are unconditional.
    \item Branch: These are conditional.
    \item Procedure call.
    \item Procedure return.
    \end{itemize}
  \item  It is  useful to  use \emph{PC-Relative}  addressing mode  to
    specify  the destination  address of  a control  flow instruction.
    This allows running the code independent  of where it is loaded --
    a   property   called  \emph{position   independence}.    Position
    independence may not always be  possible, especially if the target
    of control flow cannot be computed at compile time.  In such cases
    other addressing modes are  used.  Register indirect addressing is
    useful for:
    \begin{enumerate}
    \item Case analysis as in \emph{switch} statements.
    \item Virtual functions or methods,
    \item Higher order functions or function pointers, and
    \item Dynamically shared libraries.
    \end{enumerate}
  \item Condition code techniques: Three methods have been used --
    \begin{itemize}
    \item Condition codes register (aka  the flags register): A set of
      reserved special bits each  indicating some defined condition is
      set or  reset during  an operation.   The subsequent  branch can
      test  these  bits.   Typically, a  separate  branch  instruction
      exists for each condition code bit.
    \item  Condition  register:  No dedicated  register.   Instead  an
      arbitrary register can be designated as the ``flags'' register.
    \item Compare and  Branch: The comparison is a part  of the branch
      instruction itself.
    \end{itemize}
  \end{itemize}
\item \textbf{Encoding an Instruction Set}:
  \begin{itemize}
  \item Variable sized.
  \item Fixed width.
  \item Hybrid: Some size varying part and some fixed part.
  \end{itemize}
\end{itemize}

\subsubsection{Some Examples of Instruction Set Design Languages}
\label{sec:isa:design:lang:eg}

We will look at MIMOLA and LISA.

\subsection{Instruction Set Description and Generation}
\label{sec:describe:generate:isa}

We use  an ``engineering'' approach  to design and development  of the
language and its  processors for describing an ISA  and generating the
processing software.

\subsubsection{Basic Elements  of the Structure of  an Instruction Set
  Language}
\label{sec:isa:lang:struct}

\begin{itemize}
\item  Mnemonic:  A string  of  ``word''  characters.   A ``word''  is
  understood intuitively, and from the context.
\item Class: ISAs frequently  group instructions into \emph{groups} or
  \emph{classes} typically  based on the semantics.  Thus  we can have
  logical  instructions,  integer  arithmetic  instructions,  etc.  We
  capture the class in this field.
\item Bit pattern:  An instruction is expressed using  a set of binary
  digits, aka bits.  The key attributes are:
  \begin{itemize}
  \item Length: The total number of bits that make up the instruction.
    For  our architecture this  is a  constant with  value \textbf{32}
    bits.
  \item  Composition: An  instruction  bit pattern  is  composed of  a
    subsets of bits that describe  components of the bit pattern.  The
    various \emph{kinds} of subsets that may be needed are:
    \begin{itemize}
    \item 
    \end{itemize}
  \end{itemize}
\end{itemize}

\subsection{Instruction Set Generation}
\label{sec:generate:isa}

\subsubsection{Basic Elements of the ``Language'' to Describe the
  Instruction}
\label{sec:describe:isa:lang}

\begin{itemize}
\item   ``insn-mnemonic''  denotes   the   \textbf{mnemonic}  of   the
  instruction.
\item ``insn-bit-pattern'' denotes the  top level composite of the bit
  pattern of the given instruction.
  \begin{itemize}
  \item  ``length'' is a  field of  the bit  pattern that  records the
    total number of bits that make up the instruction.
  \item ``composition''  is a variable  length field that  records the
    composition of the bits pattern.
  \end{itemize}
\end{itemize}

\end{document}
