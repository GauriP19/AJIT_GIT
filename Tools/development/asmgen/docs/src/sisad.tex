\chapter{Towards Assembler Extraction}
\label{chap:asm:extraction}

\section{Succinct ISA Descriptions}
\label{sec:sisad}

\textbf{A. M. Vichare}

ISA description languages seem to be  at least 20 years old problem as
of 2018.   Attempts like  MIMOLA or  LISA have  been made  to describe
processors and  generate system software through  them.  This document
records my  attempts to develop  such a  language afresh, but  for the
AJIT processor of IIT Bombay.  The benefit of hindsight should ideally
be employed in this design process.  I shall try to bring that in as a
parallel activity along side the attempts to a practical design.

\subsection{Instruction Set Design Study}
\label{sec:isa:design:study}

This is the  background work mainly of conceptual ideas,  and study of
some known examples.

\subsubsection{Basic Concepts of Instruction Set Design}
\label{sec:basic:isa:design:info}

From: Henn-Patt, CA-Quant.Approach. Ed.5, App.A:

\begin{itemize}
\item \textbf{Type of internal storage}:
  \begin{itemize}
  \item Stack: Operands are on the stack, and hence \emph{implicit} in
    the instruction.
  \item Accumulator: One of the  operands is in the \emph{accumulator}
    register, and hence implicit in the instruction.
  \item  Register-Memory:   Memory  \emph{can   be}  a  part   of  the
    instruction.
  \item  Register-Register: Memory  is  \textbf{never} a  part of  the
    instruction,   except   for    the   \emph{load-store}   pair   of
    instructions.
  \item Memory-Memory:  All operands  are in  the memory  and directly
    addressed as a  part of the instruction.  This is  an old style is
    not often found today ($\sim$ 2018).
  \item  Variations:  Dedicating  some   registers  for  some  special
    purposes  --  \textbf{extended   accumulator}  or  \textbf{special
      purpose registers}.
  \item  Number of  operands: This  depends  on the  type of  internal
    storage,  and  a  design   choice.   An  binary  instruction  (aka
    \emph{operation}) may explicitly take two data source operands and
    one result destination operand.  Or it may take only two operands,
    with one  of them being \textbf{both}  a data source and  a result
    destination operand.
  \end{itemize}
\item \textbf{Memory layout addressing}:
  \begin{itemize}
  \item Byte ordering: There are two ways to order a set of bytes of a
    multi-byte object  (e.g. 32 bit,  i.e. 4 byte integer).
    \begin{itemize}
    \item \textbf{Little Endian}: The  byte with the least significant
      bit can  be stored at  the smallest byte  address, or 
    \item \textbf{Big Endian}: The byte with the least significant bit
      can be stored at the largest byte address.
    \end{itemize}
  \item Alignment  needs: For  multibyte objects, an  architecture may
    need the components to be  aligned on suitable address boundaries.
    Or it may not need them to be so aligned!  If $k$ is the number of
    bytes of a  multibyte object, $a$ is the address  of the byte with
    the  least  significant  bit,  then  the  object  is  aligned  if:
    $a = n \times k$, where $n$  is a natural number.  The address $a$
    is an integral multiple of the object size $k$.
  \item Shifting needs: Consider  reading a \emph{single} byte aligned
    at a word address into a  \emph{64 bit} register.  A single 64 bit
    read, i.e. a  double word read, would be performed  on double word
    aligned address.  If  the word aligned byte  would \textbf{not} be
    double word aligned,  then the byte that is read  would not occupy
    the least  significant position in  the 64 bit register.   In such
    cases for correct  alignment, we will need to shift  the byte read
    in by 3  positions (calculate this ``3'') so that  it occupies the
    correct position in a 64 bit register.
  \end{itemize}
\item \textbf{Addressing Modes}: \\
  How do we address the primary memory?
  \begin{itemize}
  \item  \emph{Immediate}:  No  addressing  at  all.   The  argument/s
    (i.e. operand/s) is/are given as a part of the instruction.  There
    is a finite size, finite number of bits, and layout norms.
  \item \emph{Register Direct}: The  operand/s is/are available in one
    or more  registers.  Instead of  being placed in  the instruction,
    the operands are available in the register.
    \begin{itemize}
    \item \emph{PC Relative}: A  variant of register direct addressing
      where the  register to be used  is fixed as the  program counter
      (i.e.  the instruction pointer).
    \end{itemize}
  \item  \emph{Direct} or  \emph{Absolute}:  The  address is  provided
    directly as an  argument.  There could be  finite size definitions
    that could  be same as or  different from the size  of the address
    bus.
  \item \emph{Register Indirect}: The operand location is given in one
    or more registers.   The register size is expected to  be the same
    as the size of the address bus.
    \begin{itemize}
    \item \emph{Auto  Increment or  Decrement}: A variant  of register
      indirect where the  indirection value in the  register is either
      automatically incremented  or decremented.   Autoincrementing is
      useful for array  traversals with the base address  of the array
      in the  register, and  the array element  size as  the increment
      value.    Autodecrementing  is   similarly   useful  for   stack
      operations.
    \item   \emph{Displacement}:  A   variant  of   register  indirect
      addressing   mode,  the   operand  location   is  given   as  an
      \emph{offset}   (i.e.   \emph{displacement}),   relative  to   a
      register  indirect address.   The  memory location  is thus  the
      offset relative to the location given in a register.
    \item  \emph{Indexed}:   Another  variant  of   register  indirect
      addressing  where  the  operand   location  is  a  well  defined
      algebraic  relation of  values in  a few  registers.  Thus,  for
      example, the location  might be given as a  \emph{sum} of values
      in two registers where one  register has the ``base'' value, and
      the other has an ``index'' (i.e. an offset) relative to the base
      value.
    \item  \emph{Scaled}: Yet  another  variant  of register  indirect
      addressing where  the operand location  is again a  well defined
      algebraic relation of values in  a few registers.  The algebraic
      relation  is  a  displacement  relative to  a  ``base''  in  one
      register  and  an integral  scale  up  of ``index''  in  another
      register.
    \end{itemize}
  \item \emph{Memory  Indirect}: Adding one more  level of indirection
    to the register  indirect mode yields this mode.   The location of
    the operand is  now available at the memory location  given by the
    register indirect mode.
  \end{itemize}
  The immediate, displacement, and  register indirect addressing modes
  are predominantly used (about 75\% to 99\% of modes used).
\item \textbf{Types and Size of Operands}:
  \begin{itemize}
  \item  Some specifications  of  size have  standardized (e.g.   IEEE
    floating point), some have become conventional (e.g. 8 bit byte, 2
    byte half words, 4 byte words etc.), some are optionally supported
    by  the  processor  architecture   (e.g.   strings,  binary  coded
    decimal, packed  decimal).  Representation  is either  tagged (not
    used  much  today  $\sim$  2018), or  encoded  within  the  opcode
    (preferred method today).
  \item \emph{Standardised}: IEEE Floating  point -- single and double
    precision.  Single precision is 4 bytes, and double precision is 8
    bytes.
  \item \emph{Conventional}:
    \begin{center}
      \begin{tabular}[h]{|c|c|c|c|c|c|}
        \hline
        Quad Word   & Double word & Word & Half word & Byte & Bits \\
        \hline
        -- & -- & -- & -- & 1 & 8 \\
        -- & -- & -- & 1 & 2 & 16 \\
        -- & -- & 1 & 2 & 4 & 32 \\
        -- & 1 & 2 & 4 & 8 & 64 \\
        1 & 2 & 4 & 8 & 16 & 128 \\
        \hline
      \end{tabular}
    \end{center}
  \end{itemize}
\item \textbf{Operations in the Instruction Set}:\\
  Thumb rule: Simplest instructions are the most widely executed ones.
  \begin{center}
    \begin{tabular}[h]{|l|l|}
      \hline
      \textbf{Type} & \textbf{Description or Examples} \\
      \hline
      Arithmetic & Arithmetic operations on numbers: +, -, *, / etc. \\
      Logical & Logical: AND, OR, NOT \\
      Data Transfer & Load, Store, Move \\
      Control Flow & Branch, Loop, Jump, Procedure call and return,
                     Trap \\
      System & OS System call, Virtual memory management \\
      Floating point & Floating point +, -, *, / etc. \\
      Decimal & Decimal  +, -, *, / etc. \\
      String & String move, compare, search \\
      Graphics & Pixel and vertex operation, compression \&
                 decompression \\
      Signal Processing & FFT, MAC \\
      \hline
    \end{tabular}
  \end{center}

  It might be useful to classify at a little more higher level: 
  \begin{center}
    \begin{tabular}[h]{|l|l|}
      \hline
      \textbf{Class} & \textbf{Description or Examples} \\
      \hline
      Data Type based & Arithmetic, Logical, Floating point, Signal
                        Processing, Graphics, Decimal, String \\
      Data Transfer & All I/O \\
      System Control & Control flow, System management \\
      \hline
    \end{tabular}
  \end{center}
\item \textbf{Instructions for Control Flow}:\\
  \begin{itemize}
  \item No well defined convention for  naming, but we will follow the
    text referred at the beginning  of this section. Four main control
    flow instructions are usually offered.
    \begin{itemize}
    \item Jump: These are unconditional.
    \item Branch: These are conditional.
    \item Procedure call.
    \item Procedure return.
    \end{itemize}
  \item  It is  useful to  use \emph{PC-Relative}  addressing mode  to
    specify  the destination  address of  a control  flow instruction.
    This allows running the code independent  of where it is loaded --
    a   property   called  \emph{position   independence}.    Position
    independence may not always be  possible, especially if the target
    of control flow cannot be computed at compile time.  In such cases
    other addressing modes are  used.  Register indirect addressing is
    useful for:
    \begin{enumerate}
    \item Case analysis as in \emph{switch} statements.
    \item Virtual functions or methods,
    \item Higher order functions or function pointers, and
    \item Dynamically shared libraries.
    \end{enumerate}
  \item Condition code techniques: Three methods have been used --
    \begin{itemize}
    \item Condition codes register (aka  the flags register): A set of
      reserved special bits each  indicating some defined condition is
      set or  reset during  an operation.   The subsequent  branch can
      test  these  bits.   Typically, a  separate  branch  instruction
      exists for each condition code bit.
    \item  Condition  register:  No dedicated  register.   Instead  an
      arbitrary register can be designated as the ``flags'' register.
    \item Compare and  Branch: The comparison is a part  of the branch
      instruction itself.
    \end{itemize}
  \end{itemize}
\item \textbf{Encoding an Instruction Set}:
  \begin{itemize}
  \item Variable sized.
  \item Fixed width.
  \item Hybrid: Some size varying part and some fixed part.
  \end{itemize}
\end{itemize}

\subsubsection{Some Examples of Instruction Set Design Languages}
\label{sec:isa:design:lang:eg}

We will look at MIMOLA and LISA.

\subsection{Instruction Set Description and Generation}
\label{sec:describe:generate:isa}

We use  an ``engineering'' approach  to design and development  of the
language and its  processors for describing an ISA  and generating the
processing software.

\subsubsection{Basic Elements  of the Structure of  an Instruction Set
  Language}
\label{sec:isa:lang:struct}

\begin{itemize}
\item  Mnemonic:  A string  of  ``word''  characters.   A ``word''  is
  understood intuitively, and from the context.
\item Class: ISAs frequently  group instructions into \emph{groups} or
  \emph{classes} typically  based on the semantics.  Thus  we can have
  logical  instructions,  integer  arithmetic  instructions,  etc.  We
  capture the class in this field.
\item Bit pattern:  An instruction is expressed using  a set of binary
  digits, aka bits.  The key attributes are:
  \begin{itemize}
  \item Length: The total number of bits that make up the instruction.
    For  our architecture this  is a  constant with  value \textbf{32}
    bits.
  \item  Composition: An  instruction  bit pattern  is  composed of  a
    subsets of bits that describe  components of the bit pattern.  The
    various \emph{kinds} of subsets that may be needed are:
    \begin{itemize}
    \item 
    \end{itemize}
  \end{itemize}
\end{itemize}

\subsection{Instruction Set Generation}
\label{sec:generate:isa}

\subsubsection{Basic Elements of the ``Language'' to Describe the
  Instruction}
\label{sec:describe:isa:lang}

\begin{itemize}
\item   ``insn-mnemonic''  denotes   the   \textbf{mnemonic}  of   the
  instruction.
\item ``insn-bit-pattern'' denotes the  top level composite of the bit
  pattern of the given instruction.
  \begin{itemize}
  \item  ``length'' is a  field of  the bit  pattern that  records the
    total number of bits that make up the instruction.
  \item ``composition''  is a variable  length field that  records the
    composition of the bits pattern.
  \end{itemize}
\end{itemize}